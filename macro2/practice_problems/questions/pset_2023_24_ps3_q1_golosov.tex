Wedges in agriculture

Consider a static, two sector economy, $A$ and $M$. There is a unit measure of agents, who have preferences $\ln C_A+\ln C_M$. Production technology in manufacturing is $Y_M=L_M$ and $Y_A=L_A^\alpha Z^{1-\alpha}$ where $L_M, L_A$ are labor supplied in $M$ and $A$, and $Z$ is land. Labor supply is inelastic, $L_A+L_M=1$. Land is also inelastic, $Z=1$.

Manufacturing technology is operated in the city, with perfectly competitive firms that pay wage $w$ to labor.

Agricultural technology is operated in the village. The village owns land in the communal property. That is, it divides land among all agents who live in the village (i.e. did not move into the city), and villagers grow agricultural output, trade it for manufacturing goods, and consume.

Initially, all agents are born in the village, and choose whether to stay in the village or to move in the city.

(a) Set up social planner's problem and derive the key optimality condition that characterizes optimal intersectoral allocation of resources. [Hint: write the condition in terms of ratios between marginal utilities, marginal product of labor, etc. Lecture 7 can be helpful.]

(b) Define competitive equilibrium in this economy. [Hint: normalize price for manufacturing goods to 1. Villagers equally divide income from agricultural production. Also describe the relationship between villagers' utility and city workers' utility in equilibrium.]

(c) Show that this equilibrium is inefficient so that there is a wedge relative to the social planner's allocation. [Hint: what does the relationship you establish for villagers' and city works' utility imply about the consumption allocations in equilibrium?]