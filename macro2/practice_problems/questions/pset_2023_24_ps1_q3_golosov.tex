(a) Prove that competitive equilibrium in the economy defined above is efficient (i.e., that competitive equilibrium allocation solves social planner's problem). What are the equilibrium dividends in this economy?

(b) Let $\left\{c_t^*, k_t^*\right\}_t$ be the solution to the social planner problem. Use these allocations to construct competitive equilibrium prices $\left\{r_t^{c e}, R_t^{c e}, w_t^{c e}\right\}_t$.

Hint (listed beside exercise in notes, not in pset):
Remember that for any constant return function $G(x_1, ..., x_n)$
we have $G(x_1, ..., x_n) = \sum_{i=1}^n G_ix_i$
where $G_i$ is the partial derivative of $G(x_1, ..., x_n)$ wrt $x_i$.

\deepersection{Part A Solution}

As our premise, recall several components of building a competitive equilibrium.

First the household/consumer problem:

\begin{align}
    \underset{\left\{c_t^{ce}, k_{t+1}^{ce}, b_{t+1}^{ce}\right\}_t}{\text{max }} \sum_{t=0}^\infty \beta^t u(c_t^{ce}) \\
\end{align}
s.t.
\begin{align}
    c_t^{ce} + k_{t+1}^{ce} + b_{t+1}^{ce} &\leq w_t^{ce} + R_t^{ce}b_t^{ce} + r_t^{ce}k_t^{ce} + (1 - \delta)k_t^{ce} + d_t^{ce} \quad \forall t \label{eq:pset_2023_24_ps1_q3_golosov_hh_budget_constraint} \\
    b_{t+1} &\text{ is bounded below}
\end{align}

where $c_t^{ce} \geq 0$, $k_{t+1}^{ce} \geq 0$, $k_0$ is given, $b_0 = 0$,
and, from the household's perspective, $w_t^[ce]$, $R_t^{ce}$, $r_t^{ce}$, and $d_t^{ce}$ are given.

Next, the firm problem:

\begin{align}
    d_t = \underset{\hat{k}_t^{ce}, \hat{l}_t^{ce}}{\text{max }} F(\hat{k}_t^{ce}, \hat{l}_t^{ce}) - w_t^{ce}\hat{l}_t^{ce} - r_t^{ce}\hat{k}_t^{ce} \label{eq:pset_2023_24_ps1_q3_golosov_firm_problem}
\end{align}

Then, our competitive equlibrium can be characterized as the 
sequence of prices, $\{r_t^{ce}, R_t^{ce}, w_t^{ce}\}_t$,
and allocations, $\{c_t^{ce}, k_{t}^{ce}, b_{t}^{ce}, d_t^{ce}, \hat{k}_t^{ce}, \hat{l}_t^{ce}\}_t$,
such that:

\begin{enumerate}
    \item $\{c_t^{ce}, k_{t}^{ce}, b_{t}^{ce}\}_t$ solves the household problem, taking $\{r_t^{ce}, R_t^{ce}, w_t^{ce}, d_t^{ce}\}_t$ as given.
    \item $\{d_t^{ce}, \hat{k}_t^{ce}, \hat{l}_t^{ce}\}_t$ solves the firm problem, taking $\{r_t^{ce}, w_t^{ce}\}_t$ as given.
    \item All markets clear, i.e., $k_t^{ce} = \hat{k}_t^{ce}$, $l_t^{ce} = \hat{l}_t^{ce}$, and $b_t^{ce} = 0$ $\forall t$.
\end{enumerate}

We will begin by looking at the Lagrangian for the household problem.
Note that the budget constraint holds with equality, given strictly 
increasing utility. 

\begin{align}
    \mathcal{L} = \sum_{t=0}^\infty \left[\beta^t u(c_t^{ce}) + \lambda_t\left[w_t^{ce} + R_t^{ce}b_t^{ce} + r_t^{ce}k_t^{ce} + (1 - \delta)k_t^{ce} + d_t^{ce} - c_t^{ce} - k_{t+1}^{ce} - b_{t+1}^{ce}\right]\right]
\end{align}

The FOCs are then:

\begin{align}
    &\frac{\partial \mathcal{L}}{\partial c_t^{ce}} = \beta^t u'(c_t^{ce}) - \lambda_t = 0 \quad \Rightarrow \beta^t u'(c_t^{ce}) =\lambda_t \label{eq:pset_2023_24_ps1_q3_golosov_foc1} \\
    &\frac{\partial \mathcal{L}}{\partial k_{t+1}^{ce}} = -\lambda_t + \lambda_{t+1}\left[r_{t+1}^{ce} + (1 - \delta)\right] = 0 \quad \Rightarrow \lambda_t = \lambda_{t+1}\left[r_{t+1}^{ce} + (1 - \delta)\right] \label{eq:pset_2023_24_ps1_q3_golosov_foc2} \\
    &\frac{\partial \mathcal{L}}{\partial b_{t+1}^{ce}} = -\lambda_t + \lambda_{t+1}R_{t+1}^{ce} = 0 \quad \Rightarrow \lambda_t = \lambda_{t+1}R_{t+1}^{ce} \label{eq:pset_2023_24_ps1_q3_golosov_foc3} \\
    &\frac{\partial \mathcal{L}}{\partial \lambda_t} = 0 \quad \Rightarrow w_t^{ce} + R_t^{ce}b_t^{ce} + r_t^{ce}k_t^{ce} + (1 - \delta)k_t^{ce} + d_t^{ce} - c_t^{ce} - k_{t+1}^{ce} - b_{t+1}^{ce} = 0 \label{eq:pset_2023_24_ps1_q3_golosov_foc4} \\
\end{align}

and the transversality conditions are:

\begin{align}
    \lim_{T \to \infty} \beta^Tu'(c_T)l_{T+1} \leq 0\\
    \lim_{T \to \infty} \beta^Tu'(c_T)b_{T+1} \leq 0
\end{align}

Note that from \eqref{eq:pset_2023_24_ps1_q3_golosov_foc2} and \eqref{eq:pset_2023_24_ps1_q3_golosov_foc3}, we have:

\begin{align}
    \lambda_t = \lambda_{t+1}\left[r_{t+1}^{ce} + (1 - \delta)\right] = \lambda_{t+1}R_{t+1}^{ce} \label{eq:pset_2023_24_ps1_q3_golosov_lambda} \\
    \Rightarrow R_{t+1}^{ce} = r_{t+1}^{ce} + (1 - \delta) \label{eq:pset_2023_24_ps1_q3_golosov_lambda2}
\end{align}


Moreover, consider that 

\begin{align}
    &\lambda_t = \beta^t u'(c_t^{ce}) && \text{by \eqref{eq:pset_2023_24_ps1_q3_golosov_foc1}} \\
    &\lambda_{t+1} = \beta^{t+1} u'(c_{t+1}^{ce}) && \text{by \eqref{eq:pset_2023_24_ps1_q3_golosov_foc1}} \\
\end{align}

Then

\begin{align}
    \frac{\lambda_{t+1}}{\lambda_t} = \frac{\beta^{t+1} u'(c_{t+1}^{ce})}{\beta^t u'(c_t^{ce})} = \beta \frac{u'(c_{t+1}^{ce})}{u'(c_t^{ce})} \label{eq:pset_2023_24_ps1_q3_golosov_lambda_ratio}
\end{align}

Additionally,

\begin{align}
    \frac{\lambda_{t+1}}{\lambda_t}&= \frac{\lambda_{t+1}}{\lambda_{t+1}R_{t+1}^{ce}} && \text{by \eqref{eq:pset_2023_24_ps1_q3_golosov_foc3}} \\
    &= \frac{1}{R_{t+1}^{ce}} \\
    &= \frac{1}{r_{t+1}^{ce} + (1 - \delta)} && \text{by \eqref{eq:pset_2023_24_ps1_q3_golosov_lambda2}} \\
\end{align}

Then,

\begin{align}
    \frac{1}{r_{t+1}^{ce} + (1 - \delta)} = \beta \frac{u'(c_{t+1}^{ce})}{u'(c_t^{ce})} \\
    \Rightarrow u'(c_t) = \beta u'(c_{t+1})[r_{t+1}^{ce} + (1 - \delta)] \label{eq:pset_2023_24_ps1_q3_golosov_euler}
\end{align}

which reflects our standard Euler equation.

Now, we return to the firm's problem in \eqref{eq:pset_2023_24_ps1_q3_golosov_firm_problem}. 

We have assumed the $F$ is continuous and differentiable
and corresponds to positive, diminishing marginal product,
and constant returns to scale in $l$ and $k$. 
Thus, we have that our problem is concave.

Moreover, we have assumed the existence of a representative firm, taking 
all firms to make identical decisions and maximize dividends paid 
to their owners.

Then, our FOCS yield:

\begin{align}
    F_k(\hat{k}_t^{ce}, \hat{l}_t^{ce}) = r_t^{ce} \label{eq:pset_2023_24_ps1_q3_golosov_firm_foc1} \\
    F_l(\hat{k}_t^{ce}, \hat{l}_t^{ce}) = w_t^{ce} \label{eq:pset_2023_24_ps1_q3_golosov_firm_foc2}
\end{align}

Then, by properties of CRS functions, we have 

\begin{align}
    F(\hat{k}_t^{ce}, \hat{l}_t^{ce}) &= \hat{k}_t^{ce}F_k(\hat{k}_t^{ce}, \hat{l}_t^{ce}) + \hat{l}_t^{ce}F_l(\hat{k}_t^{ce}, \hat{l}_t^{ce}) \\
    &= \hat{k}_t^{ce}r_t^{ce} + \hat{l}_t^{ce}w_t^{ce} \label{eq:pset_2023_24_ps1_q3_golosov_firm_foc3}
\end{align}

From this, we have $d_t = 0$, which is logical, since 
positive dividends would imply that firms should demand 
arbitrarily large amounts of capital and labor.

Finally, we return to the market clearing conditions and enforce

\begin{align}
    k_t^{ce} = \hat{k}_t^{ce} = k^* \\
    1 = l_t^{ce} = \hat{l}_t^{ce} = l^* \\
    b_t = 0
\end{align}

Again by CRS properties, we have 

\begin{align}
    r_t = f'(k_t)
\end{align}

where $k$ reflects the capital-labor ratio. This implies 
that $r_{t+1} = f'(k_{t+1})$. If we then plug this into our 
Euler equation, \eqref{eq:pset_2023_24_ps1_q3_golosov_euler}, we get:

\begin{align}
    u'(c_t) = \beta u'(c_{t+1})[f'(k_{t+1}) + (1 - \delta)] \label{eq:pset_2023_24_ps1_q3_golosov_euler2}
\end{align}

If we then enforce equality and substitute $f(k^*_{t}) = r^{*ce}_t k_t^* + w^{*ce}_t$, $d_t = 0$,
and $b^*_t = b^*_{t+1} = 0$ into our household budget constraint, \eqref{eq:pset_2023_24_ps1_q3_golosov_hh_budget_constraint},
we get:

\begin{align}
    c_t^* + k_{t+1}^* &= f(k^*_{t}) + (1 - \delta)k_t^* \\
\end{align}

Recall that the social planner's problem is:

\begin{align}
    \underset{\left\{c_t, k_t\right\}}{\text{max}} \sum_{t = 0}^\infty \beta^t u(c_t) \\
\end{align}
s.t.

\begin{align}
    c_t + k_{t+1} &\leq f(k_t) + (1 - \delta)k_t \quad \forall t
\end{align}

or, via the Lagrangian,

\begin{align}
    \max_{\left\{c_t, k_t\right\}} \sum_{t = 0}^\infty \beta^t u(c_t) + \sum_{t=0}^\infty \lambda_t\left[f(k_t) + (1 - \delta)k_t - c_t - k_{t+1}\right]
\end{align}

which, via the FOCs, yields: 

\begin{align}
    u'(c_t) = \beta u'(c_{t+1})[f'(k_{t+1}) + (1 - \delta)] 
\end{align}

Thus, the two maximization problems are the same, and, assuming based 
on the phrasing the of the question that the social planner's 
solution is taken to be efficient, the result of the 
competitive equilibrium is efficient.



\samesection{Part B Solution}

Given the solution to the social planner's problem, $\{c_t^*, k_t^*\}_t$,
we will construct the competitive equilibrium prices $\{r_t^{ce}, R_t^{ce}, w_t^{ce}\}_t$.

First, note that

\begin{align}
    r_t^{ce} &= f'(k_t^*) \label{eq:pset_2023_24_ps1_q3_golosov_r_ce} \\
    w_t^{ce} &= f(k_t^*) - f'(k_t^*)k_t^* = f(k_t^*) - r_t^{ce}k_t^*
\end{align}

The factor prices are equal to the marginal product. The 
wage has been defined via the CRS production function properties.
Moreover, we have:

\begin{align}
    R_t^{ce} &= r_t^{ce} + (1 - \delta) && \text{by \eqref{eq:pset_2023_24_ps1_q3_golosov_lambda2}} \\
    &= f'(k^*_{t}) + (1 - \delta)  && \text{by \eqref{eq:pset_2023_24_ps1_q3_golosov_r_ce}} \\
\end{align}

Similar to the final steps of Part (A), substituting these expressions
into the budget constraint and the Euler Equation for the competitive 
equilibrium connects the competitive equilibrium resulting from the 
firm/household problems to the solution to the social planner's problem.

Thus, $\{R_t^{ce}, r_t^{ce}, w_t^{ce}\}_t$ as described above are the prices that
characterize the competitive equilibrium.