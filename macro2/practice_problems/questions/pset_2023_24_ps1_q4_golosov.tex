(a) Show that any sequences of dividends $\left\{d_t\right\}_t$ consumers value as $\sum_{t=0}^{\infty} Q_t d_t$, where $Q_t=R_1^{-1} \times$ $\ldots \times R_t^{-1}$, that is consumers obtain the same utility for any two sequences $\left\{d_t^{\prime}\right\}_t,\left\{d_t^{\prime \prime}\right\}_t$ with the same present value $\sum_{t=0}^{\infty} Q_t d_t^{\prime}=\sum_{t=0}^{\infty} Q_t d_t^{\prime \prime}$

[Hint: Note that you can assume that NPG and TVC are satisfied. If useful, you can assume $\lim _{t \rightarrow \infty}\left(\prod_{j=0}^{t-1} R_j^{-1}\right) b_t \geq 0$.]

(b) Define a firm optimization problem in which firms own initial capital, make all investment decisions, and hire labor to maximize the present value stream of dividends $\sum_{t=0}^{\infty} Q_t d_t$.

(c) Define competitive equilibrium in this economy and show that it is efficient. How do dividends in this equilibrium compare to the dividends in the equilibrium that was set up in Definition 1?

\deepersection{Part A Solution}

First, consider the household's budget set:

\begin{align}
    A = &\{\{c_t\}_{t \geq 0} \mid \exists \{b_t\}_{t \geq 0} \text{ s.t. } \\
    &c_t + b_{t+1} \leq w_t + R_tb_t + d_t \text{ for all } t \geq 0 \label{eq:pset_2023_24_ps1_q4_golosov_cond1} \\
    &\text{ and } \underset{t \rightarrow \infty}{\text{lim}} (\prod_{j=0}^{t-1} R_j^{-1})b_t \geq 0\} \label{eq:pset_2023_24_ps1_q4_golosov_cond2}
\end{align}

Next, consider the lifetime budget constraint set:

\begin{align}
    B=\left\{\left\{c_t\right\}_{t \geq 0} \mid \sum_{t=0}^{\infty}\left(\prod_{j=0}^t R_j^{-1}\right) c_t \leq \sum_{t=0}^{\infty}\left(\prod_{j=0}^t R_j^{-1}\right) w_t+\sum_{t=0}^{\infty}\left(\prod_{j=0}^t R_j^{-1}\right) d_t\right\}
\end{align}

Our goal is to prove that sets $A$ and $B$ are equivalent.

First, we will show that $A \subseteq B$. 

Suppose $\{c_t\}_{t \geq 0} \in A$. 
Then, by repeatedly applying \eqref{eq:pset_2023_24_ps1_q4_golosov_cond1}, we have:

\begin{align}
    b_t &\leq R_{t-1}b_{t-1} + w_{t-1} + d_{t-1} - c_{t-1} \\
    &\leq R_{t-1}[R_{t-2}b_{t-2} + w_{t-2} + d_{t-2} - c_{t-2}] + w_{t-1} + d_{t-1} - c_{t-1} \\
    &= R_{t-1}R_{t-2}b_{t-2} + R_{t-1}(w_{t-2} + d_{t-2} - c_{t-2}) + w_{t-1} + d_{t-1} - c_{t-1} \\
    &\leq R_{t-1}R_{t-2}[R_{t-3}b_{t-3} + w_{t-3} + d_{t-3} - c_{t-3}] + R_{t-1}(w_{t-2} + d_{t-2} - c_{t-2}) + w_{t-1} + d_{t-1} - c_{t-1} \\
    &\vdots \\
    &\leq \left(\prod_{j=0}^{t-1} R_j\right) b_0+\sum_{s=0}^{t-2}\left(\prod_{j=0}^{t-2-s} R_{t-1-j}\right)\left[w_s+d_s-c_s\right]+\left(w_{t-1}+d_{t-1}-c_{t-1}\right)
\end{align}

That is,

\begin{align}
    b_t \leq \left(\prod_{j=0}^{t-1} R_j\right) b_0+\sum_{s=0}^{t-2}\left(\prod_{j=0}^{t-2-s} R_{t-1-j}\right)\left[w_s+d_s-c_s\right]+\left(w_{t-1}+d_{t-1}-c_{t-1}\right)
\end{align}

Multiplying each side by $(\prod_{j=0}^{t-1} R_j^{-1})$, we have:

\begin{align}
    \begin{aligned}
        \left(\prod_{j=0}^{t-1} R_j^{-1}\right) b_t & \leq b_0+\left(\prod_{j=0}^{t-1} R_j^{-1}\right) \sum_{s=0}^{t-2}\left(\prod_{j=0}^{t-2-s} R_{t-1-j}\right)\left[w_s+d_s-c_s\right]+\left(\prod_{j=0}^{t-1} R_j^{-1}\right)\left(w_{t-1}+d_{t-1}-c_{t-1}\right) \\
        & =b_0+\sum_{s=0}^{t-2}\left(\prod_{j=0}^s R_j^{-1}\right)\left[w_s+d_s-c_s\right]+\left(\prod_{j=0}^{t-1} R_j^{-1}\right)\left(w_{t-1}+d_{t-1}-c_{t-1}\right) \\
        & =b_0+\sum_{s=0}^{t-1}\left(\prod_{j=0}^s R_j^{-1}\right)\left[w_s+d_s-c_s\right] \\
        & = \sum_{s=0}^{t-1}\left(\prod_{j=0}^s R_j^{-1}\right)\left[w_s+d_s-c_s\right]
    \end{aligned}
\end{align}

where the last equality comes from $b_0=0$.

Combining this inequality with \eqref{eq:pset_2023_24_ps1_q4_golosov_cond2}, we have:

\begin{align}
    \sum_{s=0}^{\infty}\left(\prod_{j=0}^s R_j^{-1}\right)\left[w_s+d_s-c_s\right] \geq 0
\end{align}

which, if we distribute and re-arrange around the inequality sign, is equivalent to the 
requirement to be included in $B$. Thus, for any $\{c_t\}_{t \geq 0} \in A$, we have $\{c_t\}_{t \geq 0} \in B$, so $A \subseteq B$.

Next, we will show that $B \subseteq A$.

Suppose $\{c_t\}_{t \geq 0} \in B$. 

Define a sequence $\{b_t\}_{t \geq 0}$ as:

\begin{align}
    b_t = R_{t-1}b_{t-1} + w_{t-1} + d_{t-1} - c_{t-1}
\end{align}

Then we've met \eqref{eq:pset_2023_24_ps1_q4_golosov_cond1} by construction.

Furthermore, from the earlier direction, we have that 

\begin{align}
    \underset{t \rightarrow \infty}{\text{lim}} (\prod_{j=0}^{t-1} R_j^{-1})b_t = \sum_{s =0}^{\infty}( \prod_{j=0}^s R_j^{-1})[w_s + d_s - c_s]
\end{align}

Substituting this into our requirement for inclusion in set B immediately 
gives \eqref{eq:pset_2023_24_ps1_q4_golosov_cond2}, so $\{c_t\}_{t \geq 0} \in A$
and hence $B \subseteq A$.

Thus, since $A \subseteq B$ and $B \subseteq A$, we have $A = B$.

Thus, we can express the household optimization problem equivalently as
either 

\begin{align}
    \max_{\left\{c_t\right\}_{t \geq 0} \in A} \sum_{t \geq 0} \beta^t u\left(c_t\right)
\end{align}

or 

\begin{align}
    \max_{\left\{c_t\right\}_{t \geq 0} \in B} \sum_{t \geq 0} \beta^t u\left(c_t\right)
\end{align}

Finally, note that 
if we replace $\{d_t\}_t$ with $\{d_t^{\prime}\}_t$ such that 

\begin{align}
    \sum_{t=0}^{\infty}\left(\prod_{j=0}^t R_j^{-1}\right) d_t=\sum_{t=0}^{\infty}\left(\prod_{j=0}^t R_j^{-1}\right) d'_t
\end{align}

$B$ does not change.

\samesection{Part B Solution}

\begin{align}
    \max_{l_t,k_{t+1}} \sum_{t \geq 0} Q_td_t, \\
    \text{with } d_t = F(k_t, l_t) - w_tl_t - i_t \label{eq:pset_2023_24_ps1_q4_golosov_firm_dividends} \\
    \text{and } i_t = k_{t+1} - k_t + \delta k_t \label{eq:pset_2023_24_ps1_q4_golosov_firm_investment}
\end{align}

\samesection{Part C Solution}

Competitive equilibrium in this economy can be 
characterized by the 
sequence of prices, $\{R_t, w_t\}_{t \geq 0}$, 
and allocations, $\{c_t, b_t, d_t, \hat{l}_t, \hat{k}_{t}\}_{t \geq 0}$,
such that

\begin{enumerate}
    \item $\{c_t, b_t\}_{t \geq 0}$ solves the below consumer/household problem,
    taking $\{R_t, w_t, d_t\}_{t \geq 0}$ as given.

\begin{align}
    &\max_{\{c_t, b_t\}_{t \geq 0}} \sum_{t \geq 0} \beta^t u\left(c_t\right) \\
    &\text{s.t. } c_t + b_{t+1} \leq w_t + R_tb_t + d_t \text{ for all } t \geq 0 \\
    &\text{and } \underset{t \rightarrow \infty}{\text{lim}} Q_tb_t \geq 0
\end{align}

    \item $\{d_t, \hat{l}_t, \hat{k}_t\}_{t \geq 0}$ solves the firm's problem defined in Part B, taking $\{R_t, w_t\}_{t \geq 0}$ as given.
    \item Markets clear: that is, $\hat{l}_t = 1$, $b_t = 0$, $c_t + i_t = F(k_t, 1)$ $\forall t$.
\end{enumerate}

Our goal is to prove that the competitive equilibrium is efficient, i.e.,
that it aligns with the allocation that solves the social planner's problem,
and to analyze the dividends.

Consider the Lagrangian for the firm's problem:

\begin{align}
    \mathcal{L} &= \sum_{t \geq 0} Q_t[F(k_t, l_t) - w_tl_t - k_{t+1} + k_t - \delta k_t] \label{eq:pset_2023_24_ps1_q4_golosov_firm_lagrangian}
\end{align}

Then consider the FOCs for \eqref{eq:pset_2023_24_ps1_q4_golosov_firm_lagrangian} wrt $k_t$ and $k_{t+1}$:

\begin{align}
    \frac{\partial \mathcal{L}}{\partial k_{t+1}} &= Q_t = 0 \label{eq:pset_2023_24_ps1_q4_golosov_firm_foc_ktp1_first} \\
    \frac{\partial \mathcal{L}}{\partial k_{t}} &= Q_t[F_k(k_t, l_t) + (1 - \delta)] = 0 \label{eq:pset_2023_24_ps1_q4_golosov_firm_foc_kt}
\end{align}

where \eqref{eq:pset_2023_24_ps1_q4_golosov_firm_foc_kt} also gives:

\begin{align}
    \frac{\partial \mathcal{L}}{\partial k_{t+1}} &= Q_{t+1}[F_k(k_{t+1}, l_{t+1}) + (1 - \delta)] = 0 \label{eq:pset_2023_24_ps1_q4_golosov_firm_foc_ktp1_second}
\end{align}

Thus, from \eqref{eq:pset_2023_24_ps1_q4_golosov_firm_foc_ktp1_first} and \eqref{eq:pset_2023_24_ps1_q4_golosov_firm_foc_ktp1_second}, we have:

\begin{align}
    Q_{t+1}[F_k(k_{t+1}, l_{t+1}) + (1 - \delta)] = Q_t \label{eq:pset_2023_24_ps1_q4_golosov_firm_foc_result}
\end{align}

Then, consider the consumer's Lagrangian:

\begin{align}
    \mathcal{L} = \sum_{t \geq 0}[ \beta^t u(c_t) - \lambda_t(c_t + b_{t+1} - w_t - R_tb_t - d_t)]
\end{align}

Consider the FOCs for $c_t$ and $b_{t+1}$:

\begin{align}
    &\frac{\partial \mathcal{L}}{\partial c_t} = \beta^t u'(c_t) - \lambda_t = 0 \Rightarrow \beta^t u'(c_t) = \lambda_t \label{eq:pset_2023_24_ps1_q4_golosov_consumer_foc_ct} \\
    &\frac{\partial \mathcal{L}}{\partial b_{t+1}} = -\lambda_t R_t + \lambda_{t+1} = 0 \Rightarrow \lambda_t = R_t \lambda_{t+1} \label{eq:pset_2023_24_ps1_q4_golosov_consumer_foc_bt1}
\end{align}

From \eqref{eq:pset_2023_24_ps1_q4_golosov_consumer_foc_ct}, we can also get:

\begin{align}
    \beta^{t+1} u'(c_{t+1}) = \lambda_{t+1} \label{eq:pset_2023_24_ps1_q4_golosov_consumer_foc_ctp1}
\end{align}

Then we have:

\begin{align}
    &\frac{\beta^t u'(c_t)}{\beta^{t+1} u'(c_{t+1})} = \frac{\lambda_t}{\lambda_{t+1}} && \text{By \eqref{eq:pset_2023_24_ps1_q4_golosov_consumer_foc_ct} and \eqref{eq:pset_2023_24_ps1_q4_golosov_consumer_foc_ctp1}} \\
    \Rightarrow &\frac{u'(c_t)}{\beta u'(c_{t+1})} = \frac{\lambda_t}{\lambda_{t+1}} \\
    \Rightarrow &u'(c_t)\lambda_{t+1} = \beta u'(c_{t+1})\lambda_t \\
    \Rightarrow &u'(c_t)\lambda_{t+1} = \beta u'(c_{t+1})R_{t+1}\lambda_{t+1} && \text{By \eqref{eq:pset_2023_24_ps1_q4_golosov_consumer_foc_bt1}} \\
    \Rightarrow &u'(c_t) = \beta u'(c_{t+1})R_{t+1} && \text{Divide both sides by $\lambda_{t+1}$} \label{eq:pset_2023_24_ps1_q4_golosov_consumer_foc_result}
\end{align}

Moreover, since $\frac{Q_t}{Q_{t+1}} = R_{t+1}$, we 
can substitute \eqref{eq:pset_2023_24_ps1_q4_golosov_consumer_foc_result}
into \eqref{eq:pset_2023_24_ps1_q4_golosov_firm_foc_result} to get
the Euler equation:

\begin{align}
    u'(c_t) = \beta u'(c_{t+1})[f'(k_t) + (1 - \delta)]
\end{align}

The resource constraint from the Social Planner's Problem can 
then be obtained by substituting 
$b_t = 0$, $l_t = 0$, $d_t = f(k_t) - w_t - k_{t+1} - k_t + \delta k_t$
in the consumer's budget constraint.

Regarding the dividends, let's first 
compute $w_t$. By the firm's problem, we have

\begin{align}
    w_t = F_l(k_t, l_t) = f(k_t) - f'(k_t)k_t
\end{align}

If we take, $l = 1$, then we have 

\begin{align}
    0 = f(k_t) - f'(k_t)k_t
    \Rightarrow f(k_t) = f'(k_t)k_t
\end{align}

and hence

\begin{align}
    F(k_t, l_t) = F(k_t, 1) = f(k_t) = f'(k_t)k_t \label{eq:pset_2023_24_ps1_q4_golosov_prod_equals}
\end{align}

Then, we can manipulate the expression for dividends, \eqref{eq:pset_2023_24_ps1_q4_golosov_firm_dividends}, to get:

\begin{align}
    d_t &= F(k_t, l_t) - w_tl_t - i_t \\
    &= f(k_t) - w_t - i_t && \text{taking $l = 1$} \\
    &= f'(k_t)k_t - f(k_t) - i_t && \text{by \eqref{eq:pset_2023_24_ps1_q4_golosov_prod_equals}} \\
    &= [f'(k_t) + 1 - \delta]k_t - k_{t+1} \\
    &= R_tk_t - k_{t+1}
\end{align}

Then total present value of profits is:

\begin{align}
    \sum_{t \geq 0} Q_t\left(R_t k_t-k_{t+1}\right) & =\sum_{t \geq 0}\left(Q_t R_t k_t-Q_t k_{t+1}\right) \\
    & =\sum_{t \geq 0}\left(Q_{t-1} k_t-Q_t k_{t+1}\right) \\
    & =k_0+\sum_{t \geq 1} Q_{t-1} k_t-\sum_{t \geq 0} Q_t k_{t+1} \\
    & =k_0+\sum_{t \geq 0} Q_t k_{t+1}-\sum_{t \geq 0} Q_t k_{t+1} \\
    & =k_0>0
\end{align}