(Marx, Piketty, and the neoclassical growth model)

Consider a one-sector neoclassical growth model. The production side is standard: there is a CRS technology the produces capital and investments:
$$
\begin{aligned}
C_t+I_t & =F\left(K_t, X_t L_t\right) \\
\dot{K}_t & =I_t-\delta K_t
\end{aligned}
$$

Technological TFP $X$ grows at a constant growth rate.
There are two types of consumers. There is measure $\pi^c$ of "capitalists" who are born with initial capital stock $k_0$. They do not work and supply capital to the firms on competitive rental markets. There is also measure $\pi^w$ of "workers" who have no initial capital stock and supply labor inelastically. To make math easier, each worker supplied $1 / \pi^w$ units of labor. Workers and capitalists can freely borrow and lend with each other. Both types of agents have preferences
$$
\int_0^{\infty} \exp (-\rho t) \ln C_t^i d t \text { for } i \in\{c, w\} .
$$

Aggregate consumption $C$ is naturally given by
$$
\pi^c C_t^c+\pi^w C_t^w=C_t
$$

1. Define competitive equilibrium in this economy.

2. Show that there is the level of capital stock $k_0^*$ such that if $k_0=k_0^*$ then this economy is on the balanced growth path.

3. What happens to the ratio of wage income of workers to the rental income of capitalists over time on the balanced growth path?

4. Is the assumption that workers can trade assets with capitalists is important for this conclusion?

In his best-selling book "Capital in the Twenty-First Century" Thomas Piketty documents that the growth rate of wages $g$ has been systematically below interest rates $r$ for most of the last century, $r>g$. He argues that this is the reason for widening inequality in many countries between the rich (who rely mainly on interest income) and the poor (who rely mainly on wage income).

5. Show that in the balance growth path we must necessarily have $r>g$.

6. For concreteness, let use Gini as a measure of inequality. What happens to inequality over time on the balanced growth path?

\deepersection{Solution}

\deepersection{Part 1}

Take the following terms

\begin{align}
    &A_t: \text{households' saving stock} \\
    &S_t: \text{saving flow} \\
    &r^s: \text{interest rate for savings} \\
    &K^s: \text{supply of capital} \\
    &L^s = \frac{1}{\pi^s}: \text{supply of labor} \\
    &K^d: \text{demand of capital} \\
    &L^d: \text{demand of labor} \\
\end{align}

The competitive equilibrium can be defined as the 
set of prices $\{r^k_t, r^A_t, w_t\}$ and quantities \\
$\{C^c_t, C^w_t, I_t, K^d_t, K^s_t, L_t^d, L^s, S_t^c, S_t^w\}$ such that

\begin{enumerate}
    \item $\{C_t^c, S_t^c, I_t\}$ solves the capitalist's problem
    \item $\{C_t^w, S_t^w\}$ solves the worker's problem
    \item $\{K_t, L_t\}$ solves the firm's problem
    \item All markets clear
\end{enumerate}

in which these criteria are clarified below.

The capitalist's problem is:

\begin{align}
    \underset{C^c_t, S^c_t, I_t}{\max} \int_0^{\infty} \exp (-\rho t) \ln C_t^c d t \\
    \quad \\
    \text{s.t.} \quad I_t + C_t^c + S_t^c = r_t^k K_t^c+ r_t^A A_t^c \label{eq:pset_2023_24_ps2_q1_golosov_cp_bc1} \\
    \dot{A}_t^c = S_t^c \\
    \dot{K}_t^c = I_t - \delta K_t^c \label{eq:pset_2023_24_ps2_q1_golosov_cp_kdot} \\
    A^c(0) = A^c_0 \\
    K^s(0) = K_0 \\
    K_t^s \geq 0
\end{align}

The worker's problem is:

\begin{align}
    \underset{C^w_t, S^w_t}{\max} \int_0^{\infty} \exp (-\rho t) \ln C_t^w d t \\
    \quad \\
    \text{s.t.} \quad C_t^w + S_t^w = \frac{w_t}{\pi^w} + r_t^A A_t^w \\
    \dot{A}_t^w = S_t^w \\
    A^w(0) = A^w_0
\end{align}

The firm's problem is:

\begin{align}
    \underset{K_t, L_t}{\max} F(K^d_t, X_t L^d_t) - w_t L^d_t - r_t^k K^d_t \label{eq:pset_2023_24_ps2_q1_golosov_firm_problem} \\
\end{align}

The market clearing conditions are:

\begin{align}
    &\text{Labor market clears:} \quad L_t^d = \pi^w L^s_t = 1 \\
    &\text{Capital market clears:} \quad K_t^d = \pi^c K_t^s \\
    &\text{Goods market clears:} \quad \pi^c I_t + \pi^c C_t^c + \pi^w C_t^w = Y_t \label{eq:pset_2023_24_ps2_q1_golosov_goods_market_clears} \\
    &\text{Bond market clears:} \quad \pi^w A_t^w + \pi^c A_t^c = 0 \label{eq:pset_2023_24_ps2_q1_golosov_bond_market_clears} \\
\end{align}

Let's consider the necessary conditions for each of the above criteria. 

For simplicity, we will denote $K^s \equiv K$ and $K^d \equiv \pi^c K$.

The current-value Hamiltonian for the capitalist's problem is:

\begin{align}
    \mathcal{H}(K_t, A_t^c, C_t^c, S_t^c, \lambda_t, \mu_t)
    &= \ln(C_t^c) + \lambda_t (r_t^k K_t + r_t^A A_t^c - C_t^c - S_t^c - \delta K_t) + \mu_tS_t^c \\
    &= \ln(C_t^c) + \lambda_t [(r_t^k - \delta) K_t + r_t^A A_t^c - C_t^c - S_t^c] + \mu_tS_t^c \\
\end{align}

The necessary conditions for the capitalists' problem are then:

\begin{align}
    \frac{\partial \mathcal{H}}{\partial C_t^c} &= \frac{1}{C_t^c} - \lambda_t = 0 \\
    &\Rightarrow \lambda_t = \frac{1}{C_t^c} \label{eq:pset_2023_24_ps2_q1_golosov_c_foc1} \\
    \frac{\partial \mathcal{H}}{\partial S_t^c} &= -\lambda_t + \mu_t = 0 \\
    &\Rightarrow \lambda_t = \mu_t \label{eq:pset_2023_24_ps2_q1_golosov_c_foc2} \\
    \dot{\lambda}_t - \rho \lambda_t &= -\frac{\partial \mathcal{H}}{\partial K_t} = -\lambda_t(r_t^k - \delta) \\
    &\Rightarrow \frac{\dot{\lambda}_t}{\lambda} = \delta -r_t^k - \rho \label{eq:pset_2023_24_ps2_q1_golosov_c_foc3} \\
    \dot{\mu}_t - \rho \mu_t &= -\frac{\partial \mathcal{H}}{\partial A_t^c} = 
    -\lambda_t r_t^A \\ 
    &\Rightarrow \dot{\mu}_t = \rho \mu_t - \lambda_t r_t^A \label{eq:pset_2023_24_ps2_q1_golosov_c_foc4} \\
    \dot{K} &= \frac{\partial \mathcal{H}}{\partial \lambda} =  (r_t^k - \delta) K_t + r_t^A A_t^c - C_t^c - S_t^c \label{eq:pset_2023_24_ps2_q1_golosov_c_foc5} \\
    \dot{A}_t^c &= \frac{\partial \mathcal{H}}{\partial \mu} = S_t^c \label{eq:pset_2023_24_ps2_q1_golosov_c_foc6} \\
    \underset{t \rightarrow \infty}{\lim} e^{-\rho t} \lambda_t K_t &= 0 \\
    \underset{t \rightarrow \infty}{\lim} e^{-\rho t} \lambda_t &\geq 0 \\
    \underset{t \rightarrow \infty}{\lim} e^{-\rho t} \mu_t A_t^c &= 0 \\
    \underset{t \rightarrow \infty}{\lim} e^{-\rho t} \mu_t &\geq 0 \\
\end{align}

The current-value Hamiltonian for the worker's problem is:

\begin{align}
    \mathcal{H}(A_t^w, S_t^w, \kappa_t) = \ln(\frac{w_t}{\pi^w} + r_t^A A_t^w - S_t^w) + \kappa_t S_t^w
\end{align}

The necessary conditions for the worker's problem are then:

\begin{align}
    \frac{\partial \mathcal{H}}{\partial S_t^w} &= \frac{-1}{\frac{w_t}{\pi^w} + r_t^A A_t^w - S_t^w} + \kappa_t = 0 \\
    &\Rightarrow \kappa_t = \frac{1}{\frac{w_t}{\pi^w} + r_t^A A_t^w - S_t^w} \\
    &\Rightarrow \kappa_t = \frac{1}{C_t^w} \label{eq:pset_2023_24_ps2_q1_golosov_w_foc1} \\
    \dot{\kappa} - \rho \kappa &= - \frac{\partial \mathcal{H}}{\partial A_t^w} 
    = -\frac{r_t^A}{\frac{w_t}{\pi^w} + r_t^A A_t^w - S_t^w} \\
    &\Rightarrow \dot{\kappa} = \rho \kappa - \frac{r_t^A}{\frac{w_t}{\pi^w} + r_t^A A_t^w - S_t^w} \\
    &\Rightarrow \dot{\kappa} = \rho \kappa - \frac{r_t^A}{C_t^w} \label{eq:pset_2023_24_ps2_q1_golosov_w_foc2} \\
    \dot{A}_t^w &= \frac{\partial \mathcal{H}}{\partial \kappa} = S_t^w \label{eq:pset_2023_24_ps2_q1_golosov_w_foc3} \\
    \underset{t \rightarrow \infty}{\lim} e^{-\rho t} \kappa_t A_t^w &= 0 \\
    \underset{t \rightarrow \infty}{\lim} e^{-\rho t} \kappa_t &\geq 0 \\
\end{align}

For the firm's problem, the rates should correspond to the marginal product, and we require:

\begin{align}
    F_K = r_t^k \\
    X_t F_L = w_t
\end{align}

Market clearing conditions:

\begin{align}
    & \pi^c I_t+\pi^c C_t^c+\pi^w C_t^w=Y_t \\
    & \pi^w A_t^w+\pi^c A_t^c=0
\end{align}

\samesection{Part 2}

We will begin by showing that if a BGP exists,
then we must have \\
$g = g_{c^w} = g_{c^c} = g_{k} = g_{y} = g_{A^w} = g_{A^c} = \frac{\dot{x}}{X}$.
We will then proceed to show that a unique $k^*_0$ exists such that 
the economy begins on this BGP.

We begin with the capitalist's problem to get a value for $g_{c^c}$.

We can make use of the capitalist's FOCs to get:

\begin{align}
    \frac{\dot{\lambda}_t}{\lambda} &= \frac{\dot{\mu}_t}{\mu} && \text{by \eqref{eq:pset_2023_24_ps2_q1_golosov_c_foc2}} \label{eq:pset_2023_24_ps2_q1_golosov_c_eq1} \\
    \frac{\dot{\mu}_t}{\mu} &= \rho - r_t^A && \text{by \eqref{eq:pset_2023_24_ps2_q1_golosov_c_foc2} and \eqref{eq:pset_2023_24_ps2_q1_golosov_c_foc4}} \label{eq:pset_2023_24_ps2_q1_golosov_c_eq2} \\
    \rho+\delta-r_t^k=\frac{\dot{\lambda}_t}{\lambda_t}&=\frac{\dot{\mu}_t}{\mu_t}=\rho-r_t^A && \text{by \eqref{eq:pset_2023_24_ps2_q1_golosov_c_foc3}, \eqref{eq:pset_2023_24_ps2_q1_golosov_c_eq1}, and \eqref{eq:pset_2023_24_ps2_q1_golosov_c_eq2}} \\
    \Rightarrow r_t^A &= r_t^k - \delta \quad \forall t
\end{align}

Note that 

\begin{align}
    \lambda_t C_t^c &= 1 && \text{by \eqref{eq:pset_2023_24_ps2_q1_golosov_c_foc1}} \\
    % differentiate wrt t
    \Rightarrow \dot{\lambda}_t C_t^c + \lambda_t \dot{C}_t^c &= 0 && \text{differentiate wrt t} \\
    % get rate of change on each side
    \Rightarrow \frac{\dot{C}_t^c}{C_t^c}&= -\frac{\dot{\lambda}_t}{\lambda_t} \\
    &= -\left(\rho+\delta-r_t^k\right) && \text{by \eqref{eq:pset_2023_24_ps2_q1_golosov_c_foc3}} \\
\end{align}

Supposing a BGP exists, then 
$\frac{\dot{C}^c}{C^c}=-\frac{\dot{\lambda}}{\lambda}=-\left(\rho+\delta-r_t^k\right)$ 
should be constant. Thus, $r_t^k = r^k$ should be constant. Thus, 
since $r_t^A$ is a function of $r_t^k$ and the constant $\delta$, 
$r_t^A$ is also constant. Thus, we denote 
$R = r^k - \delta = r_t^A$

Then, note that 

\begin{align}
    -\left(\rho+\delta-r_t^k\right) &= R - \rho \\
\end{align}

Thus,

\begin{align}
    g_{C^c} = \frac{\dot{C}_t^c}{C_t^c} = R - \rho \\
\end{align}

Now, we consider the worker's problem to get a value for $g_{C^w}$.

Due to the bond market clearing conditions\footnote{Note to self: R}, 
we have that the interest rate earned by workers on saving is $R$, i.e., $r_A = R$.
From there, we have 

\begin{align}
    \dot{\kappa} &= \rho \kappa - r_t^A \kappa && \text{by combining \eqref{eq:pset_2023_24_ps2_q1_golosov_w_foc1} and \eqref{eq:pset_2023_24_ps2_q1_golosov_w_foc2}} \\
    \Rightarrow \frac{\dot{\kappa}}{\kappa} &= \rho - r_t^A = \rho - R \label{eq:pset_2023_24_ps2_q1_golosov_w_eq1} \\
\end{align}


Additionally, we have 

\begin{align}
    &\kappa_t C_t^w = 1 && \text{by \eqref{eq:pset_2023_24_ps2_q1_golosov_w_foc1}} \\
    % differentiate wrt t
    \Rightarrow &\dot{\kappa}_t C_t^w + \kappa_t \dot{C}_t^w = 0 && \text{differentiate wrt t} \\
    % get rate of change on each side
    \Rightarrow &\frac{\dot{C}_t^w}{C_t^w}= -\frac{\dot{\kappa}_t}{\kappa_t} \\
    % apply pset_2023_24_ps2_q1_golosov_w_eq1
    \Rightarrow &\frac{\dot{C}_t^w}{C_t^w}= R - \rho && \text{by \eqref{eq:pset_2023_24_ps2_q1_golosov_w_eq1}} \\
\end{align}

Thus, we have 

\begin{align}
    g_{C^w} = \frac{\dot{C}_t^w}{C_t^w} = R - \rho = g_{C^c} \\
\end{align}

Now, we consider the firm's problem.

Consider that from \eqref{eq:pset_2023_24_ps2_q1_golosov_firm_problem}, we can 
take the partial derivative wrt to k of 

\begin{align}
    F(K_t^d\pi_c, X_t L_t^d) - w_tL^d_t - r^k_t K_t^d \pi_c
\end{align}

to get

\begin{align}
    \pi_c F_k(K_t^d\pi_c, X_t L_t^d) - r^k_t \pi_c = 0 \\
    \Rightarrow \pi_c F_k(K_t^d\pi_c, X_t L_t^d) = \pi_c r^k_t \\ 
    \Rightarrow F_k(K_t^d\pi_c, X_t L_t^d) = r^k_t \\
    \Rightarrow F_k(\frac{K_t^d\pi_c}{X_t},\frac{X_t L_t^d}{X_t}) = r^k_t && \text{since $F_k$ is H0} \\
    \Rightarrow F_k(\frac{K_t^d\pi_c}{X_t}, L_t^d) = r^k_t \\
    \Rightarrow F_k(\frac{K_t^d\pi_c}{X_t}, 1) = r^k_t && \text{since $L_t^d = 1$}  \\
    \Rightarrow F_k(\frac{K_t^d\pi_c}{X_t}, 1) = r^k && \text{since $r^k_t = r^k$ is constant} \\
\end{align}

Thus, $\frac{K_t^d\pi_c}{X_t}$ must be constant, since $r^k$ is.

Thus, since $g = \frac{\dot{X}}{X}$ and $\frac{K_t^d\pi_c}{X_t}$ is constant, we must have

\begin{align}
    g_{k} = g
\end{align}

Additionally, note that\footnote{Note to self: R}

\begin{align}
    Y &= X_t F(\frac{\pi_c K}{X_t}, 1) \\
    \Rightarrow &F(\frac{\pi_c K_t}{X_t}, 1) = \frac{Y_t}{K_t}
\end{align}

Thus, $\frac{Y_t}{K_t}$ and hence, $\frac{\dot{Y}}{Y}$ must be constant and 
equal to $\frac{\dot{X}}{X}$.

That is, 

\begin{align}
    g_k = g_y = g
\end{align}

\begin{align}
    &Y_t = \pi^c I_t+\pi^c C_t^c+\pi^w C_t^w \\
    \Rightarrow &Y_t = \pi^c (\dot{K} + \delta K_t^s) + \pi^c C_t^c+\pi^w C_t^w && \text{substituting \eqref{eq:pset_2023_24_ps2_q1_golosov_cp_kdot} into \eqref{eq:pset_2023_24_ps2_q1_golosov_goods_market_clears}} \\
    \Rightarrow &Y_t = \pi^c (g + \delta)K_t^s + \pi^c C_t^c+\pi^w C_t^w && \text{since $\dot{K} = gK^s_t$} \\
    \Rightarrow &Y_0 \exp(gt) = C_0 \exp(g_ct) + \pi^c (g + \delta)\exp(gt) K_0^s && \text{since $\frac{\dot{x}}{x} = g \Leftrightarrow \exp(gt)x_t$} \\
\end{align}

Given $\pi^c(g+\delta), Y_0, C_0>0$, and $g_k = g_y = g$ and $g_{C^c} = g_{C^w} = g_C = \frac{\dot{C}_t}{C}$, this gives us the solution $g_c = g$; 
this follows from Uzawa, but we have provided some direction above. 

We will now look at debt.

Consider the capitalist's budget constraints:

\begin{align}
    &\dot{K} = \left(r^k-\delta\right) K^s_t+r_t^A A_t^C-C_t^c-S_t^c && \text{by \eqref{eq:pset_2023_24_ps2_q1_golosov_cp_bc1} and \eqref{eq:pset_2023_24_ps2_q1_golosov_cp_kdot}} \\
    \Rightarrow &\dot{K}=R K^s_t+R A_t^C-C_t^c-S_t^c && \text{since } R = \left(r^k-\delta\right) = r_t^A \\
    \Rightarrow &g K^s=R\left(K^s_t+A_t^C\right)-C_t^c-S_t^c && \text{since $\dot{K} = gK^s_t$ and $r_t^A = $} \\
    \Rightarrow &g K^s=R\left(K^s_t+A_t^C\right)-C_t^c-\dot{A}_t^c && \text{by \eqref{eq:pset_2023_24_ps2_q1_golosov_c_foc6}} \\
\end{align}

From Uzawa's Theorem, we then have $g_{A^c} = g$.

This comes from taking $A_t^c$ to grow at a constant rate, $g_{A^c}$, on the BGP,
then we have from the above:

\begin{align}
        &0= (R-g) K_0 e^{g t}-C_0 e^{g t}+\left(R-g_{A^c}\right) A_0^c e^{g_A c t} \\
        \Rightarrow & \left(g_{A^c}-R\right) A_0^c=e^{\left(g-g_{A^c}\right) t}\left[(R-g) K_0-C_0\right] \\
        \Rightarrow &g=g_{A^c} && \text {differentiate wrt time} 
\end{align}

To identify the growth rate of worker's assets on this prospective BGP,
we have:

\begin{align}
    A_t^c &= -\frac{\pi^w}{\pi^c} A_t^w && \text{by \eqref{eq:pset_2023_24_ps2_q1_golosov_bond_market_clears}} \\
\end{align}

Thus, we need 

\begin{align}
    A_0^c=-\frac{\pi^w}{\pi^c} A_0^w
\end{align}

and from the BGP, we have:

\begin{align}
    A_0^c=-\frac{\pi^w}{\pi^c} A_0^w e^{\left(g_A w-g\right) t}
\end{align}

Differentiating wrt time yields:

\begin{align}
    g_{A^w}=g
\end{align}

Thus, we have 

\begin{align}
    g=g_{c^w}=g_{c^c}=g_k=g_y=g_{A^w}= g_{A^c}
\end{align}

Having shown the equivalence of all growth rates under 
the supposed BGP, we want to show that a $k_o^*$ exists such that 
the economy begins on this BGP.

Consider that 

\begin{align}
    \frac{\dot{C}^c}{C^c}=g=r_t^k-(\rho+\delta)
\end{align}

gives 

\begin{align}
    r_t^k=g+\delta+\rho
\end{align}

which implies that 

\begin{align}
    F_k\left(\pi^c \frac{K_t}{X_t}, 1\right)=g+\delta+\rho
\end{align}

is fixed.

Moreover, applying aforementioned growth rate facts, we have 

\begin{align}
    K_t&=k_0 e^{g t}\\
    X_t&=x_0 e^{g t}
\end{align}


We define 

\begin{align}
    f(k) \equiv F_k(k, 1)
\end{align}

Then

\begin{align}
    f\left(\pi^c \frac{k_0}{x_0}\right)=g+\delta+\rho
\end{align}

Note that $f$ is strictly positive and monotonically decreasing,
and thus, there exists a unique $k_0^*$ to satisfy the above equation.
For this $k_0^*$, the economy begins on the BGP.


\samesection{Part 3}
The ratio is constant on the BGP.
At equilibrium, 
\begin{align} 
    w_t&=X_t F_L\left(\pi^c K_t, X_t L_t\right) \\
       &=X_t F_L\left(\pi^c \frac{K_t}{X_t}, 1\right) & \text{homogenous degree }0 \\
\end{align}
which is constant since $\pi^c \frac{K_t}{X_t}$. 
Therefore, $\frac{\dot{w}}{w}=\frac{\dot{X}}{X}=g$ and the ratio in question is
$$
\frac{w}{r K}=\frac{w_0}{k_0 r} e^{(g-g) t}=\frac{w_0}{k_0 r}, \quad \forall t
$$

\samesection{Part 4}
No, there is no trading in assets along the BGP, so our conclusion does not depend on the existence of this market.
Our conclusion depends only constant interest rates, the FOCs for the firm's problem holding, and the market clearing conditions for $L$ and $K$.

\samesection{Part 5}
Recall that for the BGP $g=r-\delta-\rho$. We know $\delta,\rho > 0$, so $r > g$.

\samesection{Part 6}
The Gini coefficient is 
$
G=\frac{1}{2 \mu} \int_{-\infty}^{\infty} \int_{-\infty}^{\infty} p(x) p(y)|x-y| d x d y
$.
where $p(x)$ is the density and $\mu$ is the mean of income.
Income is $\frac{w_t}{\pi^w}$ with probability $\pi^w$ and $r_t K_t$ with probability $\pi^c$, so we can rewrite the expression above to get
$$
G_t=\frac{\pi^w \pi^C\left|r_t K_t-\left(w_t / \pi^w\right)\right|}{\pi^C r_t K_t+w_t}
$$

On the BGP, $r_t$ is constant and $K_t$ and $w_t$ grow at the same rate.
Hence, inequality as measured by the Gini coefficient is constant over time.
