(Marx, Piketty, and the neoclassical growth model)

Consider a one-sector neoclassical growth model. The production side is standard: there is a CRS technology the produces capital and investments:
$$
\begin{aligned}
C_t+I_t & =F\left(K_t, X_t L_t\right) \\
\dot{K}_t & =I_t-\delta K_t
\end{aligned}
$$

Technological TFP $X$ grows at a constant growth rate.
There are two types of consumers. There is measure $\pi^c$ of "capitalists" who are born with initial capital stock $k_0$. They do not work and supply capital to the firms on competitive rental markets. There is also measure $\pi^w$ of "workers" who have no initial capital stock and supply labor inelastically. To make math easier, each worker supplied $1 / \pi^w$ units of labor. Workers and capitalists can freely borrow and lend with each other. Both types of agents have preferences
$$
\int_0^{\infty} \exp (-\rho t) \ln C_t^i d t \text { for } i \in\{c, w\} .
$$

Aggregate consumption $C$ is naturally given by
$$
\pi^c C_t^c+\pi^w C_t^w=C_t
$$

1. Define competitive equilibrium in this economy.

2. Show that there is the level of capital stock $k_0^*$ such that if $k_0=k_0^*$ then this economy is on the balanced growth path.

3. What happens to the ratio of wage income of workers to the rental income of capitalists over time on the balanced growth path?

4. Is the assumption that workers can trade assets with capitalists is important for this conclusion?

In his best-selling book "Capital in the Twenty-First Century" Thomas Piketty documents that the growth rate of wages $g$ has been systematically below interest rates $r$ for most of the last century, $r>g$. He argues that this is the reason for widening inequality in many countries between the rich (who rely mainly on interest income) and the poor (who rely mainly on wage income).

5. Show that in the balance growth path we must necessarily have $r>g$.

6. For concreteness, let use Gini as a measure of inequality. What happens to inequality over time on the balanced growth path?