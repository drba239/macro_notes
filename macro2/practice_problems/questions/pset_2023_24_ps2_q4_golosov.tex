(Potatoes and the fall of aristocracy)

Consider a two-sector version of the neoclassical growth model without capital. The output in the manufacturing is given by a constant returns to scale technology

$$
Y^M=X^M L^M,
$$
where $L^M$ is supply of labor in manufacturing.

The output in the aggricultural sector is given by technology
$$
Y^A=X^A\left(L^A\right)^\alpha(\bar{Z})^{1-\alpha},
$$
where $X$ is agricultural productivity, $L^A$ is labor supply in agriculture, and $\bar{Z}$ is supply of land, available in fixed supply.

Suppose there are two sets of agents: workers who supply 1 unit of labor inelastically and aristocrats, who supply no labor but own land and receive land rents. Intratemporal preferences of all agents are given by
$$
\left[\left(c^A\right)^{(\sigma-1) / \sigma}+\left(c^M\right)^{(\sigma-1) / \sigma}\right]^{\sigma /(\sigma-1)}
$$
with $\sigma<1$.

Since we abstract from capital, we will focus on the static economy.

1. Define competitive equilibrium in this economy.

2. What happens to labor income if agricultural productivity $X^A$ increases? What happens to land rents? What happens labor allocation $L^M$ and $L^A$ ?

In a series of papers, Nathan Nunn, Nancy Qian and co-authors study an exogenous agricultural technical change: introduction of potatoes to Europe during the Columbian exchange. Potatoes are much superior in their nutritional characteristics to native European staples such as turnips. By exploiting variation in European regional variation in suitability for cultivating potatoes, the authors provide causal evidence that introduction of potatoes in Europe increase urbanization and reduced the incidence of European military conflict.

3. Explain how these findings can be rationalized by the standard two-sector growth model?