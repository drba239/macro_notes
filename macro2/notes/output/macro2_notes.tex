\documentclass[10pt]{article}
\usepackage{amsmath}
\usepackage{amsthm}
\usepackage{amsfonts}
\usepackage{amssymb}
%\usepackage[version=4]{mhchem}
\usepackage{amssymb}
%\\usepackage{stmaryrd}
\setlength\parindent{0pt}
\usepackage[margin=1.2in]{geometry}
\usepackage{enumitem}
\usepackage{xcolor}
\usepackage{hyperref}
\setcounter{tocdepth}{4}

% Set the length of \parskip to add a line between paragraphs
\setlength{\parskip}{1em}

\usepackage{mathtools}
\mathtoolsset{showonlyrefs=true}


\DeclareMathSymbol{\Perp}{\mathrel}{symbols}{"3F}

\newtheorem{theorem}{Theorem}[section]  % Numbered within sections
\newtheorem{definition}[theorem]{Definition} % Definitions share numbering with theorems
\newtheorem{proposition}[theorem]{Proposition}  % Propositions share numbering with theorems



\newcounter{example}

\newenvironment{example}
{
  \stepcounter{example}
  \noindent\textbf{Example \thesection.\theexample.}
}
{
  \par
}


% deeper section command
% This will let you go one level deeper than whatever section level you're on.
\makeatletter
\newcommand{\deepersection}[1]{%
  \ifnum\value{subparagraph}>0
    % Already at the deepest standard level (\subparagraph), cannot go deeper
    \subparagraph{#1}
  \else
    \ifnum\value{paragraph}>0
      \subparagraph{#1}
    \else
      \ifnum\value{subsubsection}>0
        \paragraph{#1}
      \else
        \ifnum\value{subsection}>0
          \subsubsection{#1}
        \else
          \ifnum\value{section}>0
            \subsection{#1}
          \else
            \section{#1}
          \fi
        \fi
      \fi
    \fi
  \fi
}
\makeatother



% same section command
% This will let create a section at the same level as whatever section level you're on.
\makeatletter
\newcommand{\samesection}[1]{%
  \ifnum\value{subparagraph}>0
    \subparagraph{#1}
  \else
    \ifnum\value{paragraph}>0
      \paragraph{#1}
    \else
      \ifnum\value{subsubsection}>0
        \subsubsection{#1}
      \else
        \ifnum\value{subsection}>0
          \subsection{#1}
        \else
          \ifnum\value{section}>0
            \section{#1}
          \else
            % Default to section if outside any sectioning
            \section{#1}
          \fi
        \fi
      \fi
    \fi
  \fi
}
\makeatother






\title{Macro 2 Notes}

\author{}
\date{}

\begin{document}
\maketitle

\tableofcontents

\section{Introduction}

Much of this is directly quoted from Golosov's notes, slides, Ragini's notes,
or the notes of past students (Jordan Rosenthal-Kay, Jingoo Kwon).

\section{Lecture 0: Neoclassical Growth Model without Growth}

This section pulls from Golosov's Lecture 0.

\subsection{Terms}

\begin{itemize}
    \item $t$: period
    \item $\beta$: discount factor
    \item $c_t$: consumption in period $t$
    \item $u(c_t)$: utility derived from consumption in period $t$
    \item $k_t$: capital in period $t$
    \item $f(k_t)$: production function
    \item $\delta$: depreciation rate
\end{itemize}

\subsection{Setup}

\subsubsection{Preferences}

Continuum of identical, infinitely lived consumers with preferences

$$
\sum_{t=0}^{\infty} \beta^t u\left(c_t\right),
$$

where $c_t \geq 0$ is consumption in period $t$.

\subsubsection{Technology}

Technology
Output produced with production function $f\left(k_t\right)$, where $k_t \geq 0$ is capital with initial $k_0>0$ given. Output can be costlessly transferred between consumption and capital for next period:

$$
    \begin{aligned}
    c_t+k_{t+1} & \leq f\left(k_t\right)+(1-\delta) k_t \\
    k_0 & >0 \text { is given. }
    \end{aligned}
$$

for depreciation rate $\delta \in(0,1)$.

\subsubsection{Assumptions}

\begin{enumerate}
    \item $u, f$ are strictly increasing, differentiable, $u$ is strictly concave, $f$ is concave;
    \item $u, f$ are "nice"\footnote{Notes from earlier in Lecture 0 on niceness: There are multiple ways to assume niceness: bounded $u ; u$ bounded from below and $F$ is such that feasible $x$ are bounded; $u$ is CRRA and some assumption on the speed of change in derivatives of $F$ around $x=0$. The formal arguments are a bit tedious and not that insightful beyond the intuition that I gave here, so we will not talk about them.};
    \item $u, f$ satisfy Inada conditions $\lim _{c \rightarrow 0} u^{\prime}(c)=\lim _{k \rightarrow 0} f^{\prime}(k)=\infty$.
\end{enumerate}

\subsection{Model}

\subsubsection{Social Planner Problem}

$$
\max _{\left\{c_t, k_t\right\}_t} \sum_{t=0}^{\infty} \beta^t u\left(c_t\right)
$$
s.t.
$$
c_t+k_{t+1} \leq f\left(k_t\right)+(1-\delta) k_t,
$$
and $c_t \geq 0, k_t \geq 0, k_0$ is given.

\subsubsection{Key Optimality Theorem}

\begin{theorem}
    
    Suppose the assumptions above hold.
(necessity) If $\left\{c_t^*, k_t^*\right\}_t$ solves (3) then $\left\{c_t^*, k_t^*\right\}_t$ satisfies
$$
\begin{gathered}
c_t^*+k_{t+1}^*=f\left(k_t^*\right)+(1-\delta) k_t^*, \\
u^{\prime}\left(c_t^*\right)=\beta\left[1+f^{\prime}\left(k_{t+1}^*\right)-\delta\right] u^{\prime}\left(c_{t+1}^*\right), \\
\lim _{T \rightarrow \infty} \beta^T u^{\prime}\left(c_T^*\right) k_{T+1}^* \leq 0 .
\end{gathered}
$$
(sufficiency) If $\left\{c_t^*, k_t^*\right\}_t$ satisfies (4), (5), and (6), then it is a solution to (3).

\end{theorem}

\section{Lecture 1}

\subsection{Neoclassical Growth Model}

\subsubsection{Terms}

\begin{itemize}
    \item $t$: period
    \item $C_t$: consumption in period $t$
    \item $I_t$: investment in period $t$
    \item $K_t$: capital in period $t$
    \item $Y_t$: output in period $t$, sum of factor income
    \item $F_t$: production function
    \item $X_t$: (labor-augmenting) technology in period $t$
    \item $n$: growth rate of population
    \item $\rho$: discount factor
\end{itemize}

\subsubsection{Basic Accounting Definitions}

\begin{align}
    &C_t+I_t=Y_t \\
    &K_{t+1}=I_t+(1-\delta) K_t\\
    &\text{$Y_t=$ sum of factor income}
\end{align}

\subsubsection{More Relationships}

\begin{align}
    &\dot{K}(t)=Y(t)-C(t)-\delta K(t) && \text{Feasibility}\\
    &Y(t)=F(K(t), X(t) L(t)) \\
    &L(t) =1 && \text{Feasibility: inelastic labor}
\end{align}

\subsubsection{Assumptions}

\begin{itemize}
    \item Perfectly competitive firms
    \item $Y_t$ is produced by CRS technology $F_t$ (DRS is a CRS with a fixed factor, IRS is hard to model parsimoniously).
    \item Two factors: capital and labor.
    \item Inelastic Labor
\end{itemize}

\subsubsection{Setup}

\textbf{Household}

Infinitely lived representative household with preferences
$$
\int_0^{\infty} e^{-\rho t} \frac{C(t)^{1-\sigma}}{1-\sigma} d t
$$
and inelastic labor supply (for now)

\subsubsection{Useful Normalization}

Re-normalize everything per unit of $X$ :
$$
\begin{aligned}
k(t) & \equiv \frac{K(t)}{X(t)} \\
c(t) &\equiv \frac{C(t)}{X(t)} \\
y(t) & \equiv \frac{Y(t)}{X(t)}=F(k(t), 1) \\
\tilde{\rho} &\equiv \rho-(1-\sigma) g_X
\end{aligned}
$$

In this case, the model becomes isomorphic to the neoclassical growth model
without growth. Thus, we have 

\begin{itemize}
    \item Competitive equilibrium is efficient.
    \item $k(t), c(t), y(t)$ converge to the steady state $k^{ss}, c^{ss}, y^{ss}$.
\end{itemize}

\subsubsection{Neoclassical Growth Model and Kaldor Facts}

Steady state of the neoclassical growth model is consistent with Kaldor
facts (presented just below)

\begin{enumerate}
    \item $y(t)=y^{ss}$ implies that $Y(t)$ grows at rate $g_X$.
    \item Capital-output ratio is constant: $K(t) / Y(t)=k^{ss} / y^{ss}$.
    \item Since consumption growth rate is constant, so are interest rates.
    \item Factor shares are constant by labor-augmenting technical change + constant interest rate.
\end{enumerate}


\subsection{Kaldor Facts}

\begin{enumerate}
    \item Output per capita grows at a constant rate.
    \item Capital-output ratio is roughly constant.
    \item Interest rate is roughly constant.
    \item Distribution of income between capital and labor is roughly constant.
\end{enumerate}

\subsection{Constant Growth}

\begin{itemize}
    \item $\frac{\dot{Y}(t)}{Y(t)}=g_Y>0$
    \item $\frac{\dot{K}(t)}{K(t)}=g_K>0$
    \item $\frac{\dot{C}(t)}{C(t)}=g_C>0$
    \item $\frac{\dot{L}(t)}{L(t)}=n$
\end{itemize}

\subsection{Uzawa Theorem}

With constant growth and CRS technology, we have

\begin{enumerate}
    \item Balanced growth: $g_Y=g_C=g_K \equiv g$
    \item Labor-augmenting technical change: $\tilde{F}$ can be represented as $\tilde{F}(K(t), L(t), \tilde{X}(t))=F(K(t), X(t) L(t))$ for some CRS $F$ with $\frac{\dot{X}(t)}{X(t)}=g-n$
\end{enumerate}

\subsubsection{Implications of Uzawa}

Some implications from Uzawa's Theorem:
\begin{itemize}
    \item With CRS, all constant growth must be balanced, i.e., all variables grow at the same rate. Moreover, per capita growth is driven by technology.
    \item Technology must be either purely labor-augmenting or the elasticity of substitution between $K$ and $L$ equals 1.
\end{itemize}

\subsection{Uzawa Theorem - Part 2}

With constant growth, CRS technology, 
and constant factor shares\footnote{Jingoo's notes also 
mention perfect competition, not sure if that's 
implicit in Golosov's statement}, we have
\begin{itemize}
    \item Constant interest rate: $R(t)=R^* \quad \forall t$
    \item Constant wage growth rate at the rate of technological growth: $\frac{\dot{w}(t)}{w(t)}=g_X=g_Y-n$
\end{itemize}

\subsection{Constant Interest Rates, Balanced Growth, and U Theorem}

Constant interest rates and balanced growth implies that $U(C)$ must be, up to a linear tranformation,
$$
U(C)=\frac{C^{1-\sigma}}{1-\sigma}
$$

\subsection{Useful Facts}

\subsubsection{Re-Expressing Growth Rates}

If any variable $Z$ grows with rate $g$, $\frac{\dot{Z}(t)}{Z(t)}=g \Longleftrightarrow Z(t)=e^{(t-\tau) g} Z(\tau)$ for all $t, \tau$

\section{Lecture 2: Structural Change - Demand Side}

\subsection{Model}

\subsubsection{Terms}

\begin{itemize}
    \item $t$: period
    \item $c_t$: aggregate consumption in period $t$
    \item $I(t)$: investment at time $t$
    \item $K(t)$: capital at time $t$
    \item $r(t)$: rental rate of capital at time $t$
    \item $w(t)$: wage rate at time $t$
    \item $\rho$: discount factor
    \item $U_0$: Utility beginning at period 0
    \begin{itemize}
        \item $c^A(t) \in [\gamma^A, \infty)$ is the agricultural consumption at time $t$.
        \item $c^M(t) \geq 0$ is the manufacturing consumption at time $t$.
        \item $c^S(t) \geq 0$ is the services consumption at time $t$.
    \end{itemize}
    \item $\gamma^A < 0$: constant establishing a subsistence level of agricultural consumption
    \begin{itemize}
        \item The household must consume at least this much agricultural production (food) to survive
    \end{itemize}
    \item $\gamma^S >0$: constant establishing that consumption of services can be zero or negative
    \item $\eta^i$: long-run share of consumption in sector $i$
    \item $p^i(t)$ is the price of one unit of $c^i(t)$ for $i \in\{A, M, S\}$
    \begin{itemize}
        \item In general, we normalize s.t. $p^M(t)=1$, but we can choose any sector to normalize to 1 if useful
    \end{itemize}
    \item $Y^i(t)$: Output of sector $i$ at time $t$
    \item $B^i$: Hicks-neutral productivity term for sector $i \in\{A, M, S\}$
    \item $X(t)$ : Labor-augmenting productivity term affecting all sectors.
    \item $g = \frac{\dot{X}(t)}{X(t)}$: growth rate of labor-augmenting productivity
\end{itemize}

\subsubsection{Model Setup}

\paragraph{Preferences}

\begin{align}
    &U_0=\int_0^{\infty} \exp (-\rho t) \frac{c(t)^{1-\sigma}-1}{1-\sigma} d t \\
    &\text{with} \\
    &c(t)=\left(c^A(t)+\gamma^A\right)^{\eta^A} c^M(t)^{\eta^M}\left(c^S(t)+\gamma^S\right)^{\eta^S} \\
    &\eta^i  >0, \sum_{i \in\{A, M, S\}} \eta^i=1, \\
    &\gamma^A  <0, \gamma^S>0
\end{align}

Budget Constraint:

\begin{align}
    \sum_{i \in\{A, M, S\}} p^i(t) c^i(t)+\dot{K}(t)=w(t)+(r(t)-\delta) K(t)
\end{align}

\paragraph{Technology}

Technology $F$ is CRS with

\begin{align}
    &Y^i(t) =B^i F\left(K^i(t), X(t) L^i(t)\right), \\
    &\dot{X}(t) / X(t) =g .
\end{align}

with capital goods produced by sector $M$

\subsubsection{Firm's Problem}

$$
\max p^i(t) Y^i(t)-w(t) L^i(t)-r(t) K^i(t)
$$
s.t.
$$
Y^i(t)=B^i F\left(K^i(t), X(t) L^i(t)\right)
$$

\paragraph{Optimality Conditions for Firm}

Capital:

$$
p^i(t) B^i F_K\left(K^i(t), X(t) L^i(t)\right)=r(t)
$$

Labor:

$$
p^i(t) B^i F_L\left(K^i(t), X(t) L^i(t)\right) X(t)=w(t)
$$


\subsubsection{Market Clearing}

\paragraph{Market Clearing for Labor and Capital}

\begin{align}
    &K^A(t)+K^M(t)+K^S(t)=K(t) \\
    &L^A(t)+L^M(t)+L^S(t)=1
\end{align}

\paragraph{Market Clearing for Agricultural and Service Goods}

\begin{align}
    &c^A(t)=Y^A(t) \\
    &c^S(t)=Y^S(t)
\end{align}

\paragraph{Manufacturing good is used in production of investment good}

\begin{align}
    &I(t)+c^M(t) =Y^M(t) \\ 
    &\dot{K}(t) =I(t)-\delta K
\end{align}

\subsubsection{Competitive Equilibrium}

Given initial $K_0$, collection of prices and quantities, such that
\begin{enumerate}
    \item Consumers choose their quantities optimally given prices.
    \item Firms choose their quantities optimally given prices.
    \item All markets clear.
\end{enumerate}



\subsubsection{Variable/Parameter Relationships}

\paragraph{Nonhomothetic Preferences}

Generally, allowing for nonhomothetic preferences:

\begin{align}
    \frac{p^i c^i}{p^M c^M}=\frac{\eta^i}{\eta^M}-\frac{p^i}{p^M} \frac{\gamma^i}{c^M}
\end{align}

Note that holding prices fixed, $p^i c^i$ growths faster (slower) than $p^M c^M$ if $\gamma^i>0\left(\right.$ if $\left.\gamma^i<0\right)$.

\begin{itemize}
    \item $\gamma^A < 0$: Consumption share of $A$ grows slower than $M$.
    \item $\gamma^S > 0$: Consumption share of $S$ grows faster than $M$.
\end{itemize}

This is consistent with cross-sectional patterns in spending.

\subsection{Useful Facts}

\subsubsection{HD1 F}

In our context, $F$ is HD1. That is,

\begin{align}
    F(\lambda K, \lambda L)=\lambda F(K, L)
\end{align}

\subsubsection{HD1 F Implications; HD0 Partials}

If $F(K, L)$ is HD1 then

\begin{align}
    F(K, L)=F_K(K, L) K+F_L(K, L) L
\end{align}

and $F_K(K, L), F_L(K, L)$ are HD0.

That is,

\begin{align}
    F_K(\lambda K,\lambda L) = F_K(K, L)\\
    F_L(\lambda K,\lambda L) = F_L(K, L)
\end{align}
 
\end{document}