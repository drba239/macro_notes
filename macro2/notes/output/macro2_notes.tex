\documentclass[10pt]{article}
\usepackage{amsmath}
\usepackage{amsthm}
\usepackage{amsfonts}
\usepackage{amssymb}
%\usepackage[version=4]{mhchem}
\usepackage{amssymb}
%\\usepackage{stmaryrd}
\setlength\parindent{0pt}
\usepackage[margin=1.2in]{geometry}
\usepackage{enumitem}
\usepackage{xcolor}
\usepackage{hyperref}
\setcounter{tocdepth}{4}

% Set the length of \parskip to add a line between paragraphs
\setlength{\parskip}{1em}

\usepackage{mathtools}
\mathtoolsset{showonlyrefs=true}


\DeclareMathSymbol{\Perp}{\mathrel}{symbols}{"3F}

\newtheorem{theorem}{Theorem}[section]  % Numbered within sections
\newtheorem{definition}[theorem]{Definition} % Definitions share numbering with theorems
\newtheorem{proposition}[theorem]{Proposition}  % Propositions share numbering with theorems



\newcounter{example}

\newenvironment{example}
{
  \stepcounter{example}
  \noindent\textbf{Example \thesection.\theexample.}
}
{
  \par
}


% deeper section command
% This will let you go one level deeper than whatever section level you're on.
\makeatletter
\newcommand{\deepersection}[1]{%
  \ifnum\value{subparagraph}>0
    % Already at the deepest standard level (\subparagraph), cannot go deeper
    \subparagraph{#1}
  \else
    \ifnum\value{paragraph}>0
      \subparagraph{#1}
    \else
      \ifnum\value{subsubsection}>0
        \paragraph{#1}
      \else
        \ifnum\value{subsection}>0
          \subsubsection{#1}
        \else
          \ifnum\value{section}>0
            \subsection{#1}
          \else
            \section{#1}
          \fi
        \fi
      \fi
    \fi
  \fi
}
\makeatother



% same section command
% This will let create a section at the same level as whatever section level you're on.
\makeatletter
\newcommand{\samesection}[1]{%
  \ifnum\value{subparagraph}>0
    \subparagraph{#1}
  \else
    \ifnum\value{paragraph}>0
      \paragraph{#1}
    \else
      \ifnum\value{subsubsection}>0
        \subsubsection{#1}
      \else
        \ifnum\value{subsection}>0
          \subsection{#1}
        \else
          \ifnum\value{section}>0
            \section{#1}
          \else
            % Default to section if outside any sectioning
            \section{#1}
          \fi
        \fi
      \fi
    \fi
  \fi
}
\makeatother






\title{Macro 2 Notes}

\author{}
\date{}

\begin{document}
\maketitle

\tableofcontents

\section{Introduction}

Much of this is directly quoted from Golosov's notes, slides, Ragini's notes,
or the notes of past students (Jordan Rosenthal-Kay, Jingoo Kwon).

\section{Lecture 0: Neoclassical Growth Model without Growth}

This section pulls from Golosov's Lecture 0.

\subsection{Terms}

\begin{itemize}
    \item $t$: period
    \item $\beta$: discount factor
    \item $c_t$: consumption in period $t$
    \item $u(c_t)$: utility derived from consumption in period $t$
    \item $k_t$: capital in period $t$
    \item $f(k_t)$: production function
    \item $\delta$: depreciation rate
\end{itemize}

\subsection{Setup}

\subsubsection{Preferences}

Continuum of identical, infinitely lived consumers with preferences

$$
\sum_{t=0}^{\infty} \beta^t u\left(c_t\right),
$$

where $c_t \geq 0$ is consumption in period $t$.

\subsubsection{Technology}

Technology
Output produced with production function $f\left(k_t\right)$, where $k_t \geq 0$ is capital with initial $k_0>0$ given. Output can be costlessly transferred between consumption and capital for next period:

$$
    \begin{aligned}
    c_t+k_{t+1} & \leq f\left(k_t\right)+(1-\delta) k_t \\
    k_0 & >0 \text { is given. }
    \end{aligned}
$$

for depreciation rate $\delta \in(0,1)$.

\subsubsection{Assumptions}

\begin{enumerate}
    \item $u, f$ are strictly increasing, differentiable, $u$ is strictly concave, $f$ is concave;
    \item $u, f$ are "nice"\footnote{Notes from earlier in Lecture 0 on niceness: There are multiple ways to assume niceness: bounded $u ; u$ bounded from below and $F$ is such that feasible $x$ are bounded; $u$ is CRRA and some assumption on the speed of change in derivatives of $F$ around $x=0$. The formal arguments are a bit tedious and not that insightful beyond the intuition that I gave here, so we will not talk about them.};
    \item $u, f$ satisfy Inada conditions $\lim _{c \rightarrow 0} u^{\prime}(c)=\lim _{k \rightarrow 0} f^{\prime}(k)=\infty$.
\end{enumerate}

\subsection{Model}

\subsubsection{Social Planner Problem}

$$
\max _{\left\{c_t, k_t\right\}_t} \sum_{t=0}^{\infty} \beta^t u\left(c_t\right)
$$
s.t.
$$
c_t+k_{t+1} \leq f\left(k_t\right)+(1-\delta) k_t,
$$
and $c_t \geq 0, k_t \geq 0, k_0$ is given.

\subsubsection{Key Optimality Theorem}

\begin{theorem}
    
    Suppose the assumptions above hold.
(necessity) If $\left\{c_t^*, k_t^*\right\}_t$ solves (3) then $\left\{c_t^*, k_t^*\right\}_t$ satisfies
$$
\begin{gathered}
c_t^*+k_{t+1}^*=f\left(k_t^*\right)+(1-\delta) k_t^*, \\
u^{\prime}\left(c_t^*\right)=\beta\left[1+f^{\prime}\left(k_{t+1}^*\right)-\delta\right] u^{\prime}\left(c_{t+1}^*\right), \\
\lim _{T \rightarrow \infty} \beta^T u^{\prime}\left(c_T^*\right) k_{T+1}^* \leq 0 .
\end{gathered}
$$
(sufficiency) If $\left\{c_t^*, k_t^*\right\}_t$ satisfies (4), (5), and (6), then it is a solution to (3).

\end{theorem}

\section{Lecture 1}

\subsection{Neoclassical Growth Model}

\subsubsection{Terms}

\begin{itemize}
    \item $t$: period
    \item $C_t$: consumption in period $t$
    \item $I_t$: investment in period $t$
    \item $K_t$: capital in period $t$
    \item $Y_t$: output in period $t$, sum of factor income
    \item $F_t$: production function
    \item $X_t$: (labor-augmenting) technology in period $t$
    \item $n$: growth rate of population
    \item $\rho$: discount factor
\end{itemize}

\subsubsection{Basic Accounting Definitions}

\begin{align}
    &C_t+I_t=Y_t \\
    &K_{t+1}=I_t+(1-\delta) K_t\\
    &\text{$Y_t=$ sum of factor income}
\end{align}

\subsubsection{More Relationships}

\begin{align}
    &\dot{K}(t)=Y(t)-C(t)-\delta K(t) && \text{Feasibility}\\
    &Y(t)=F(K(t), X(t) L(t)) \\
    &L(t) =1 && \text{Feasibility: inelastic labor}
\end{align}

\subsubsection{Assumptions}

\begin{itemize}
    \item Perfectly competitive firms
    \item $Y_t$ is produced by CRS technology $F_t$ (DRS is a CRS with a fixed factor, IRS is hard to model parsimoniously).
    \item Two factors: capital and labor.
    \item Inelastic Labor
\end{itemize}

\subsubsection{Setup}

\textbf{Household}

Infinitely lived representative household with preferences
$$
\int_0^{\infty} e^{-\rho t} \frac{C(t)^{1-\sigma}}{1-\sigma} d t
$$
and inelastic labor supply (for now)

\subsubsection{Useful Normalization}

Re-normalize everything per unit of $X$ :
$$
\begin{aligned}
k(t) & \equiv \frac{K(t)}{X(t)} \\
c(t) &\equiv \frac{C(t)}{X(t)} \\
y(t) & \equiv \frac{Y(t)}{X(t)}=F(k(t), 1) \\
\tilde{\rho} &\equiv \rho-(1-\sigma) g_X
\end{aligned}
$$

In this case, the model becomes isomorphic to the neoclassical growth model
without growth. Thus, we have 

\begin{itemize}
    \item Competitive equilibrium is efficient.
    \item $k(t), c(t), y(t)$ converge to the steady state $k^{ss}, c^{ss}, y^{ss}$.
\end{itemize}

\subsubsection{Neoclassical Growth Model and Kaldor Facts}

Steady state of the neoclassical growth model is consistent with Kaldor
facts (presented just below)

\begin{enumerate}
    \item $y(t)=y^{ss}$ implies that $Y(t)$ grows at rate $g_X$.
    \item Capital-output ratio is constant: $K(t) / Y(t)=k^{ss} / y^{ss}$.
    \item Since consumption growth rate is constant, so are interest rates.
    \item Factor shares are constant by labor-augmenting technical change + constant interest rate.
\end{enumerate}


\subsection{Kaldor Facts}

\begin{enumerate}
    \item Output per capita grows at a constant rate.
    \item Capital-output ratio is roughly constant.
    \item Interest rate is roughly constant.
    \item Distribution of income between capital and labor is roughly constant.
\end{enumerate}

\subsection{Constant Growth}

\begin{itemize}
    \item $\frac{\dot{Y}(t)}{Y(t)}=g_Y>0$
    \item $\frac{\dot{K}(t)}{K(t)}=g_K>0$
    \item $\frac{\dot{C}(t)}{C(t)}=g_C>0$
    \item $\frac{\dot{L}(t)}{L(t)}=n$
\end{itemize}

\subsection{Uzawa Theorem}

With constant growth and CRS technology, we have

\begin{enumerate}
    \item Balanced growth: $g_Y=g_C=g_K \equiv g$
    \item Labor-augmenting technical change: $\tilde{F}$ can be represented as $\tilde{F}(K(t), L(t), \tilde{X}(t))=F(K(t), X(t) L(t))$ for some CRS $F$ with $\frac{\dot{X}(t)}{X(t)}=g-n$
\end{enumerate}

\subsubsection{Implications of Uzawa}

Some implications from Uzawa's Theorem:
\begin{itemize}
    \item With CRS, all constant growth must be balanced, i.e., all variables grow at the same rate. Moreover, per capita growth is driven by technology.
    \item Technology must be either purely labor-augmenting or the elasticity of substitution between $K$ and $L$ equals 1.
\end{itemize}

\subsection{Uzawa Theorem - Part 2}

With constant growth, CRS technology, 
and constant factor shares\footnote{Jingoo's notes also 
mention perfect competition, not sure if that's 
implicit in Golosov's statement}, we have
\begin{itemize}
    \item Constant interest rate: $R(t)=R^* \quad \forall t$
    \item Constant wage growth rate at the rate of technological growth: $\frac{\dot{w}(t)}{w(t)}=g_X=g_Y-n$
\end{itemize}

\subsection{Constant Interest Rates, Balanced Growth, and U Theorem}

Constant interest rates and balanced growth implies that $U(C)$ must be, up to a linear tranformation,
$$
U(C)=\frac{C^{1-\sigma}}{1-\sigma}
$$

\subsection{Useful Facts}

\subsubsection{Re-Expressing Growth Rates}

If any variable $Z$ grows with rate $g$, $\frac{\dot{Z}(t)}{Z(t)}=g \Longleftrightarrow Z(t)=e^{(t-\tau) g} Z(\tau)$ for all $t, \tau$

\section{Lecture 2: Structural Change - Demand Side}

\subsection{Model}

\subsubsection{Terms}

\begin{itemize}
    \item $t$: period
    \item $c_t$: aggregate consumption in period $t$
    \item $I(t)$: investment at time $t$
    \item $K(t)$: capital at time $t$
    \item $r(t)$: rental rate of capital at time $t$
    \item $w(t)$: wage rate at time $t$
    \item $\rho$: discount factor
    \item $U_0$: Utility beginning at period 0
    \begin{itemize}
        \item $c^A(t) \in [\gamma^A, \infty)$ is the agricultural consumption at time $t$.
        \item $c^M(t) \geq 0$ is the manufacturing consumption at time $t$.
        \item $c^S(t) \geq 0$ is the services consumption at time $t$.
    \end{itemize}
    \item $\gamma^A < 0$: constant establishing a subsistence level of agricultural consumption
    \begin{itemize}
        \item The household must consume at least this much agricultural production (food) to survive
    \end{itemize}
    \item $\gamma^S >0$: constant establishing that consumption of services can be zero or negative
    \item $\eta^i$: long-run share of consumption in sector $i$
    \item $p^i(t)$ is the price of one unit of $c^i(t)$ for $i \in\{A, M, S\}$
    \begin{itemize}
        \item In general, we normalize s.t. $p^M(t)=1$, but we can choose any sector to normalize to 1 if useful
    \end{itemize}
    \item $Y^i(t)$: Output of sector $i$ at time $t$
    \item $B^i$: Hicks-neutral productivity term for sector $i \in\{A, M, S\}$
    \item $X(t)$ : Labor-augmenting productivity term affecting all sectors.
    \item $g = \frac{\dot{X}(t)}{X(t)}$: growth rate of labor-augmenting productivity
\end{itemize}

\subsubsection{Model Setup}

\paragraph{Preferences}

\begin{align}
    &U_0=\int_0^{\infty} \exp (-\rho t) \frac{c(t)^{1-\sigma}-1}{1-\sigma} d t \\
    &\text{with} \\
    &c(t)=\left(c^A(t)+\gamma^A\right)^{\eta^A} c^M(t)^{\eta^M}\left(c^S(t)+\gamma^S\right)^{\eta^S} \\
    &\eta^i  >0, \sum_{i \in\{A, M, S\}} \eta^i=1, \\
    &\gamma^A  <0, \gamma^S>0
\end{align}

Budget Constraint:

\begin{align}
    \sum_{i \in\{A, M, S\}} p^i(t) c^i(t)+\dot{K}(t)=w(t)+(r(t)-\delta) K(t)
\end{align}

\paragraph{Technology}

Technology $F$ is CRS with

\begin{align}
    &Y^i(t) =B^i F\left(K^i(t), X(t) L^i(t)\right), \\
    &\dot{X}(t) / X(t) =g .
\end{align}

with capital goods produced by sector $M$

\subsubsection{Firm's Problem}

$$
\max p^i(t) Y^i(t)-w(t) L^i(t)-r(t) K^i(t)
$$
s.t.
$$
Y^i(t)=B^i F\left(K^i(t), X(t) L^i(t)\right)
$$

\paragraph{Optimality Conditions for Firm}

Capital:

\begin{align}
    p^i(t) B^i F_K\left(K^i(t), X(t) L^i(t)\right)=r(t) \label{eq:l2_firm_capital_optimality}
\end{align}

Labor:

\begin{align}
    p^i(t) B^i F_L\left(K^i(t), X(t) L^i(t)\right) X(t)=w(t) \label{eq:l2_firm_labor_optimality}
\end{align}

Interpretation: 
The rental rate of capital must be the marginal value 
of capital to production multiplied by price 
and the productivity term. 
The wage rate must be the 
marginal value of labor to production multiplied by price 
and the productivity term and the labor-augmenting 
technology. 


\subsubsection{Market Clearing}

\paragraph{Market Clearing for Labor and Capital}

\begin{align}
    &K^A(t)+K^M(t)+K^S(t)=K(t) \\
    &L^A(t)+L^M(t)+L^S(t)=1
\end{align}

\paragraph{Market Clearing for Agricultural and Service Goods}

\begin{align}
    &c^A(t)=Y^A(t) \\
    &c^S(t)=Y^S(t)
\end{align}

\paragraph{Manufacturing good is used in production of investment good}

\begin{align}
    &I(t)+c^M(t) =Y^M(t) \\ 
    &\dot{K}(t) =I(t)-\delta K
\end{align}

\subsubsection{Competitive Equilibrium}

Given initial $K_0$, collection of prices and quantities, such that
\begin{enumerate}
    \item Consumers choose their quantities optimally given prices.
    \item Firms choose their quantities optimally given prices.
    \item All markets clear.
\end{enumerate}



\subsubsection{Variable/Parameter Relationships}

\paragraph{Nonhomothetic Preferences}

Generally, allowing for nonhomothetic preferences:

\begin{align}
    \frac{p^i c^i}{p^M c^M}=\frac{\eta^i}{\eta^M}-\frac{p^i}{p^M} \frac{\gamma^i}{c^M}
\end{align}

Note that holding prices fixed, $p^i c^i$ growths faster (slower) than $p^M c^M$ if $\gamma^i>0\left(\right.$ if $\left.\gamma^i<0\right)$.

\begin{itemize}
    \item $\gamma^A < 0$: Consumption share of $A$ grows slower than $M$.
    \item $\gamma^S > 0$: Consumption share of $S$ grows faster than $M$.
\end{itemize}

This is consistent with cross-sectional patterns in spending.

\paragraph{Equalization of capital-labor ratios}

Since 

\begin{align}
    \frac{r(t)}{w(t)} = \frac{p^i(t) B_i F_K\left(K^i(t), X(t) L^i(t)\right)}{p^i(t) B_i F_L\left(K^i(t), X(t) L^i(t)\right) X(t)}
\end{align}

by \eqref{eq:l2_firm_capital_optimality} and \eqref{eq:l2_firm_labor_optimality},
we have (by re-arranging and cancellation):

\begin{align}
    X(t) \frac{r(t)}{w(t)}=\frac{F_K\left(K^i(t), X(t) L^i(t)\right)}{F_L\left(K^i(t), X(t) L^i(t)\right)}
\end{align}

Notably:

\begin{itemize}
    \item Since $F_K, F_L$ are HD0, the RHS is a function of $\frac{K^i(t)}{X(t) L^i(t)}$.
        \begin{itemize}
            \item To be hyper-clear, I am saying that since (1) and HD0 
            function maps to the same output for inputs with the same ratio 
            and (2) since the RHS is a function of two functions with this quality 
            and with the same 
            inputs, it must be the case that the whole RHS is 
            just a function of the ratio of these two inputs.
        \end{itemize}
    \item Since $r(t) / w(t)$ does not depend on $i$, there is some $k(t)$ s.t.
\end{itemize}
\begin{align}
    \frac{K^i(t)}{X(t) L^i(t)}=k(t) \quad \text{for all $i$} \label{eq:l2_capital_labor_ratio}
\end{align}
\begin{itemize}
    \item[]
        \begin{itemize}
            \item I am not totally sure why they don't reference whole LHS,
            which also don't depend on $i$, but essentially, I believe they 
            are just saying that since the LHS doesn't depend on $i$,
            there must be some constant capital-labor ratio that characterizes 
            the RHS across sectors.
        \end{itemize}
\end{itemize}

\paragraph{Constant Relative Prices}

From \eqref{eq:l2_capital_labor_ratio}, we have that:

\begin{align}
    F_K\left(K^i(t), X(t) L^i(t)\right)=F_K\left(K^j(t), X(t) L^j(t)\right) \equiv \bar{F}(t) \quad \forall i, j \in\{A, M, S\}
\end{align}

From \eqref{eq:l2_firm_capital_optimality} with $p^M = 1$, this gives 

\begin{align}
    B^M \bar{F}(t)=r(t) &\Rightarrow B_M = \frac{r(t)}{\bar{F}(t)} \\
    p^A(t) B^A \bar{F}(t)=r(t) &\Rightarrow p^A = \frac{r(t)}{\bar{F}(t)} \frac{1}{B^A} = \frac{B^M}{B^A} \\
    p^S(t) B^S \bar{F}(t)=r(t) &\Rightarrow p^S = \frac{r(t)}{\bar{F}(t)} \frac{1}{B^S} = \frac{B^M}{B^S}
\end{align}

where the last equality in the second and third lines follows from the first line.

This leads to the conclusion that:

\begin{itemize}
    \item In CE prices determined by technology, not preferences
        \begin{itemize}
            \item same growth rate in all sectors $\Longleftrightarrow$ same relative prices
        \end{itemize}
\end{itemize}

\paragraph{Optimality Conditions for Consumers}

In Pset 2 or 3, we derive the following from the consumer optimality conditions:

\begin{align}
    \frac{1}{\sigma}(r-\delta-\rho)=\frac{\dot{c}^M(t)}{c^M(t)}=\frac{\dot{c}(t)}{c(t)} \\
    \frac{\dot{c}^M}{c^M}=\frac{\dot{c}^A}{c^A+\gamma^A}=\frac{\dot{c}^S}{c^S+\gamma^S}
\end{align}

which implies sectoral reallocation (since $\gamma^A<0, \gamma^S>0$)

\begin{align}
    \frac{\dot{c}^A}{c^A}<\frac{\dot{c}^M}{c^M}<\frac{\dot{c}^S}{c^S}
\end{align}

\paragraph{Structural Change}

Since $c^A(t)=Y^A(t)$ and $c^S(t)=Y^S(t)$, we must have
$$
\frac{\dot{Y}^A}{Y^A}<\frac{\dot{Y}^S}{Y^S}
$$

\paragraph{Aggregation}

We want to aggregate up our economy:

Start with three feasibility conditions

\begin{align}
    c^A(t) & =B^A F\left(K^A(t), X(t) L^A(t)\right) \\
    c^S(t) & =B^S F\left(K^S(t), X(t) L^S(t)\right) \\
    c^M(t)+\dot{K}(t) & =B^M F\left(K^M(t), X(t) L^M(t)\right)-\delta K(t)
\end{align}

We can then apply \eqref{eq:l2_hd1_lemma} to get:

\begin{align}
    c^A(t)= & B^A\left\{F_K^A(t) K^A(t)+F_L^A X(t) L^A(t)\right\} \\ 
    c^S(t)= & B^S\left\{F_K^S(t) K^S(t)+F_L^S(t) X(t) L^S(t)\right\} \\ 
    c^M(t)+\dot{K}(t)= & B^M\left\{F_K^M(t) K^M(t)+F_L^M(t) X(t) L^M(t)\right\} -\delta K(t) \\ 
\end{align}

Multiplying each $c^i$ by $p^i$ then gives:

\begin{align}
    p^A c^A(t)= & p^A B^A\left\{F_K^A(t) K^A(t)+F_L^A X(t) L^A(t)\right\} \\ 
    p^S c^S(t)= & p^S B^S\left\{F_K^S(t) K^S(t)+F_L^S(t) X(t) L^S(t)\right\} \\ 
    p^M c^M(t)+p^M \dot{K}(t)= & p^M B^M\left\{F_K^M(t) K^M(t)+F_L^M(t) X(t) L^M(t)\right\} -\delta p^M K(t) \\
\end{align}

Then, summing gives:

\begin{align}
    p^A c^A(t)+p^S c^S(t)+p^M c^M(t)= & p^A B^A\left\{F_K^A(t) K^A(t)+F_L^A X(t) L^A(t)\right\} \\
    & +p^S B^S\left\{F_K^S(t) K^S(t)+F_L^S(t) X(t) L^S(t)\right\} \\
    & +p^M B^M\left\{F_K^M(t) K^M(t)+F_L^M(t) X(t) L^M(t)\right\} -\delta p^M K(t) \\
    = & r(t)(K^A(t)+K^S(t)+K^M(t)) + && \text{by \eqref{eq:l2_firm_capital_optimality}} \\
    &w(t)(X(t) L^A(t)+X(t) L^S(t)+X(t) L^M(t)) -\delta p^M K(t) && \text{by \eqref{eq:l2_firm_labor_optimality}} \\
    = & r(t) K(t) + w(t) \bar{L} -\delta p^M K(t) && \text{by market clearing}
\end{align}

We want to get rid of the $r(t)$ and $w(t)$. 
We will do that in the next subsection.

\subsection{CGP Growth Rates}

Note that $\frac{K^i(t)}{X(t) L^i(t)}=k(t)$ for all $i$ implies
$$
\frac{K(t)}{X(t) \bar{L}}=k(t)
$$

Since $F_K$ and $F_L$ are HDO:
$$
\begin{aligned}
r(t) & =B^M F_K\left(K^M(t), X(t) L^M(t)\right)=B^M F_K(k(t), 1) \\
& =B^M F_K(K(t), X(t) \bar{L})
\end{aligned}
$$
and same for $w(t)$

Therefore, the sum of feasibility constraints is

\begin{align}
    p^A c^A(t)+c^M(t)+p^S c^S(t)+\dot{K}(t)=B^M F(K(t), X(t) \bar{L})-\delta K(t) \label{eq:l2_cgp_sum_i_prod}
\end{align}

\subsubsection{Constant Capital-Labor Output Ratio}

If CGP exists, then $r(t)=r$ and therefore $k(t)=k$ so that all sectorial capital-labor ratios are constant
$$
K^i(t)=k \cdot X(t) L^i(t)
$$

Sum up across $i$:

\begin{align}
    K(t)=k \cdot X(t) \bar{L} \label{l2:cgp_sum_i_kl}
\end{align}

Therefore

\begin{align}
    &\frac{\dot{K}(t)}{K(t)}=g \\
    \Rightarrow &\dot{K} = g K(t) \label{l2:cgp_k_dot}
\end{align}

since the only value changing on the RHS of \eqref{l2:cgp_sum_i_kl} is $X(t)$,
and it's multiplied by the other two terms.

Then plugging \eqref{l2:cgp_k_dot} into \eqref{eq:l2_cgp_sum_i_prod} gives the aggregate feasibility equation:

\begin{align}
    p^A c^A(t)+c^M(t)+p^S c^S(t)=B^M F(K(t), X(t) \bar{L})-(\delta+g) K(t) \label{eq:l2_agg_feas_eq_new_kdot}
\end{align}

\subsubsection{Existence of CGP}

Then re-writing \eqref{eq:l2_agg_feas_eq_new_kdot} as 

\begin{align}
    & p^A\left(c^A(t)+\gamma^A\right)+c^M(t)+p^S\left(c^S(t)+\gamma^S\right) \\ 
    & -\left[p^A \gamma^A+p^S \gamma^S\right] \\ 
    = & B^M F(K(t), X(t) \bar{L})-(\delta+g) K(t)
\end{align}

implies that $p^A\left(c^A(t)+\gamma^A\right), c^M(t), p^S\left(c^S(t)+\gamma^S\right)$
grow at the same rates.

Apply Uzawa's arguments: can have balanced growth only if
$$
p^A \gamma^A+p^S \gamma^S=0
$$


\begin{proposition}
    In the above-described economy a CGP exists if and only if

    $$
    \frac{\gamma^A}{B^A}+\frac{\gamma^S}{B^S}=0
    $$

    In a CGP $k(t)=k$ for all $t$, and moreover
    $$
    \frac{\dot{c}^A}{c^A}=g \frac{c^A+\gamma^A}{c^A}, \frac{\dot{c}^M}{c^M}=g, \frac{\dot{c}^S}{c^S}=g \frac{c^S+\gamma^S}{c^S}
    $$

    \begin{itemize}
        \item Growth rate in $S$ starts high and asymptotes to $g$ as $c^S \rightarrow \infty$
        \item Growth rate in $A$ starts low and asymptotes to $g$ as $c^A \rightarrow \infty$
    \end{itemize}

\end{proposition}

\subsubsection{Labor Transition}

We have
$$
c^i(t)=X(t) L^i(t) B^i F(k, 1) \text { for } i \in\{A, S\}
$$

This implies
$$
\frac{\dot{c}^i}{c^i}=\frac{\dot{X}}{X}+\frac{\dot{L}^i}{L^i}
$$

In Pset 2:

We showed that this + previous equations imply in CGP
$$
\frac{\dot{L}^M}{L^M}=0, \frac{\dot{L}^A}{L^A}<0, \frac{\dot{L}^S}{L^S}>0 .
$$

\subsection{Useful Facts}

\subsubsection{HD1 F}

In our context, $F$ is HD1. That is,

\begin{align}
    F(\lambda K, \lambda L)=\lambda F(K, L)
\end{align}

\subsubsection{HD1 F Implications; HD0 Partials}

If $F(K, L)$ is HD1 then

\begin{align}
    F(K, L)=F_K(K, L) K+F_L(K, L) L \label{eq:l2_hd1_lemma}
\end{align}

and $F_K(K, L), F_L(K, L)$ are HD0.

That is,

\begin{align}
    F_K(\lambda K,\lambda L) = F_K(K, L)\\
    F_L(\lambda K,\lambda L) = F_L(K, L)
\end{align}

\section{Lecture 3: Structural Change - Supply Side}

\subsection{Terms}

\begin{itemize}
    \item $\rho$: discount rate 
    \item $t$: time 
    \item $c(t):$ consumption at time $t$
    \item $i \in \{A, M, S\}$: sector
    \item $c^i(t):$ consumption of good $i$ at time $t$
    \item $\eta_i$: long-run share of consumption in sector $i$
    \item $K^i(t)$: capital in sector $i$ at time $t$
    \item $L^i(t)$: labor in sector $i$ at time $t$
    \item $Y^i(t)$: output of sector $i$ at time $t$
    \item $p^i(t)$: price of one unit of $c^i(t)$ at time $t$
    \item $X^i(t)$: labor-augmenting technology in sector $i$ at time $t$
    \item $g_i$: growth rate of labor-augmenting technology, $X$, in sector $i$
    \item $\sigma$: price elasticity
\end{itemize}

\subsection{Setup}

\subsubsection{Preferences}

\begin{align}
    \int_0^{\infty} \exp (-\rho t) \frac{c(t)^{1-\theta}-1}{1-\theta} dt
\end{align}

with 

\begin{align}
    c(t)=\left(\sum_{i \in\{A, S, M\}} \eta^i c^i(t)^{(\sigma-1) / \sigma}\right)^{\sigma /(\sigma-1)}
\end{align}

%%%%%%%%%%%%%%%%%%%%%%%%%%%%%%%%%%%%%%%%%%%%%%%%%%%%%%%%%%%%%%%%%%%%%%%%%%%%%%%%%%%%%%%

\subsubsection{Technology with Unequal Growth}

\begin{align}
    Y^i(t) & =X^i(t) K^i(t)^\alpha L^i(t)^{1-\alpha} \\
    \dot{X}^i(t) / X^i(t) & =g^i
\end{align}

%%%%%%%%%%%%%%%%%%%%%%%%%%%%%%%%%%%%%%%%%%%%%%%%%%%%%%%%%%%%%%%%%%%%%%%%%%%%%%%%%%%%%%%

\subsubsection{Additional Assumption}

\begin{itemize}
    \item Inelastic labor
    \item $M$ produces all capital
    \item price elasticity, $\sigma$, assumed to be same for all goods
    \item Income elasticity is 1 for all goods
\end{itemize}

%%%%%%%%%%%%%%%%%%%%%%%%%%%%%%%%%%%%%%%%%%%%%%%%%%%%%%%%%%%%%%%%%%%%%%%%%%%%%%%%%%%%%%%

\subsubsection{Intratempotal optimality condition}

\begin{align}
    \frac{c^i}{c^j}=\left(\frac{\eta^i}{\eta^j}\right)^\sigma\left(\frac{p^i}{p^j}\right)^{-\sigma} \label{eq:l3_intratemporal_optimality}
\end{align}

If $p^i / p^j$ increases then relatively consumption shares $p^i c^i / p^j c^j$

\begin{itemize}
    \item decreases if $\sigma<1$
    \item constant if $\sigma=1$
    \item increases if $\sigma>1$
\end{itemize}

\color{red}
I'm not sure that I understand why this is this way. 
\color{black}

%%%%%%%%%%%%%%%%%%%%%%%%%%%%%%%%%%%%%%%%%%%%%%%%%%%%%%%%%%%%%%%%%%%%%%%%%%%%%%%%%%%%%%%

\subsubsection{Feasibility Constraint}

\begin{align}
    c^M+\dot{K} & =X^M\left(K^M\right)^\alpha\left(L^M\right)^{1-\alpha}-\delta K \\
    c^A & =X^A\left(K^A\right)^\alpha\left(L^A\right)^{1-\alpha} \\
    c^S & =X^S\left(K^S\right)^\alpha\left(L^S\right)^{1-\alpha}
\end{align}

Multiply by $p^i$ and sum to get
$$
C+\dot{K}=X^M(K)^\alpha(\bar{L})^{1-\alpha}-\delta K
$$
where
$$
C \equiv \sum_{i \in\{A, M, S\}} p^i c^i
$$

To make progress, lets express dynamic conditions in terms of C

%%%%%%%%%%%%%%%%%%%%%%%%%%%%%%%%%%%%%%%%%%%%%%%%%%%%%%%%%%%%%%%%%%%%%%%%%%%%%%%%%%%%%%%

\subsubsection{More Intertemporal Optimality}

Let $\lambda$ be multiplier on the consumer's budget constraint and
$$
C(t) \equiv \sum_{i \in\{A, M, S\}} p^i(t) c^i(t)
$$
be total consumption expenditures. Show that optimality requires\footnote{
    \color{red}
    Not sure that I understand what little $c$ is with no superscript. 
    \color{black}
}
$$
\frac{\dot{\lambda}}{\lambda}+\frac{\dot{C}}{C}=(1-\theta) \frac{\dot{c}}{c}
$$
and
$$
\frac{\dot{\lambda}}{\lambda}=-\left(\alpha X^M K^{\alpha-1} \bar{L}^{1-\alpha}-\delta-\rho\right)
$$

%%%%%%%%%%%%%%%%%%%%%%%%%%%%%%%%%%%%%%%%%%%%%%%%%%%%%%%%%%%%%%%%%%%%%%%%%%%%%%%%%%%%%%%

\subsubsection{Euler Equation}

So we have two conditions

\begin{align}
    &C+\dot{K}=X^M K^\alpha \bar{L}^{1-\alpha}-\delta K \\
    &\frac{\dot{C}}{C}-(1-\theta) \frac{\dot{c}}{c}=\left(\alpha X^M K^{\alpha-1} \bar{L}^{1-\alpha}-\delta-\rho\right)
\end{align}

Note that if $\theta=1$, these are the optimality conditions of the neoclassical growth model

\begin{itemize}
    \item $\theta=1$ is a sufficient condition to deliver Kaldor facts
    \item it also turns out to be a necessary condition
\end{itemize}


%%%%%%%%%%%%%%%%%%%%%%%%%%%%%%%%%%%%%%%%%%%%%%%%%%%%%%%%%%%%%%%%%%%%%%%%%%%%%%%%%%%%%%%

\subsection{Results}

\subsubsection{Production efficiency}

Firm's FOCS:

\begin{align} 
    p^i(t) X^i(t) \alpha\left(\frac{K^i(t)}{L^i(t)}\right)^{\alpha-1} & =r(t) \\ 
    p^i(t) X^i(t)(1-\alpha)\left(\frac{K^i(t)}{L^i(t)}\right)^\alpha & =w(t)
\end{align}

From firm's optimization, capital-labor ratios are equalized across sectors
$$
\frac{K^i(t)}{L^i(t)}=k(t) \text { for all } i
$$

Relative prices reflect relative productivities

\begin{align}
    \frac{p^i(t)}{p^j(t)}=\frac{X^j(t)}{X^i(t)} \text { for } i, j \in\{A, S, M\} \label{eq:l3_rel_prices}
\end{align}

Notice, relative prices fall in sectors with higher productivity growth.

%%%%%%%%%%%%%%%%%%%%%%%%%%%%%%%%%%%%%%%%%%%%%%%%%%%%%%%%%%%%%%%%%%%%%%%%%%%%%%%%%%%%%%%

\subsubsection{Consumption Side}

Plug these into intratremporal optimality for consumers -- 
that is combine \eqref{eq:l3_intratemporal_optimality} and \eqref{eq:l3_rel_prices} --
to get:

\begin{align}
    \frac{p^i(t) c^i(t)}{p^j(t) c^j(t)}=\left(\frac{\eta^i}{\eta^j}\right)^\sigma\left(\frac{X^j(t)}{X^i(t)}\right)^{1-\sigma} \label{eq:l3_rel_lab_tech}
\end{align}


Consumption share $p^i c^i$ for the more stagnant sector

\begin{itemize}
    \item increases if $\sigma<1$ (empirically relevant case)
    \item constant if $\sigma=1$
    \item decreases if $\sigma>1$
\end{itemize}

That is, if demand is elastic, then 
$1- \sigma$ is negative and an increase in the productivity of $j$
means that the RHS is getting smaller, i.e., the share of 
consumption of $i$ relative to $j$ is decreasing. This is sensible 
as elastic demand would indicate that the increase in demand 
for $j$ would be proportionally larger than its decrease in price. 
The reverse is true for inelastic demand.

%%%%%%%%%%%%%%%%%%%%%%%%%%%%%%%%%%%%%%%%%%%%%%%%%%%%%%%%%%%%%%%%%%%%%%%%%%%%%%%%%%%%%%%

\subsubsection{Labor Allocation}

For $i, j \neq M$,

$$
c^i(t)=X^i(t) k(t)^\alpha L^i(t)
$$

Then, we have:

\begin{align}
    \frac{c^i(t)}{c^j(t)}&=\frac{X^i(t) k(t)^\alpha L^i(t)}{X^j(t) k(t)^\alpha L^j(t)} \\
    &=\frac{X^i(t)}{X^j(t)}\frac{L^i(t)}{L^j(t)} \\
    &= \frac{P^j(t)}{P^i(t)} \frac{L^i(t)}{L^j(t)} && \text{by \eqref{eq:l3_rel_prices}} \\
    \Rightarrow \frac{c^i(t)}{c^j(t)}\frac{P^i(t)}{P^j(t)}&=\frac{L^i(t)}{L^j(t)} \\
    \Rightarrow \frac{L^i(t)}{L^j(t)}&=\left(\frac{\eta^i}{\eta^j}\right)^\sigma\left(\frac{X^j(t)}{X^i(t)}\right)^{1-\sigma} && \text{by \eqref{eq:l3_rel_lab_tech}}
\end{align}

This gives\footnote{\color{red}
Would need to think more about this 
\color{black}}
$$
\frac{\dot{L}_i(t)}{L_i(t)}-\frac{\dot{L}_j(t)}{L_j(t)}=(1-\sigma)\left(g^j-g^i\right) \text { for } i \in\{A, S\}
$$

\subsubsection{Takeaways}

\begin{itemize}
    \item Suppose demand is inelastic $(\sigma<1)$
        \begin{itemize}
            \item prices of faster growing sector fall
            \item consumption share of that sector falls
            \item labor outflows from that sector
        \end{itemize}
    \item Same logic extends to arbitrary number of sectors
        \begin{itemize}
            \item asymptotically, everyone works in the most stagnant sector 
            \item ``Baumol's cost disease''
        \end{itemize}
\end{itemize}

%%%%%%%%%%%%%%%%%%%%%%%%%%%%%%%%%%%%%%%%%%%%%%%%%%%%%%%%%%%%%%%%%%%%%%%%%%%%%%%%%%%%%%%
%%%%%%%%%%%%%%%%%%%%%%%%%%%%%%%%%%%%%%%%%%%%%%%%%%%%%%%%%%%%%%%%%%%%%%%%%%%%%%%%%%%%%%%
%%%%%%%%%%%%%%%%%%%%%%%%%%%%%%%%%%%%%%%%%%%%%%%%%%%%%%%%%%%%%%%%%%%%%%%%%%%%%%%%%%%%%%%

\section{Lecture 8}

\subsection{Terms}

\begin{itemize}
    \item $Y_i$: output from firm $i$
    \item $K_i$: capital in firm $i$
    \item $L_i$: labor in firm $i$
    \item $A_i$: technology for firm $i$
    \item $P_i$: price for firm $i$
    \item $\bar{K}$: Consumers inelastic endowment of capital 
    \item $\bar{L}$: Consumers inelastic endowment of labor
    \item $U(C)$: Utility derived from capital 
    \item $\sigma$: Elasticity of substitution\footnote{\color{red} Not sure if this is right} 
    \item $\Pi_i$: Profit for firm $i$
    \item $r$: rental rate of capital
    \item $w$: wage rate
\end{itemize}

\subsection{Setup}

\subsubsection{Intermediate Sector}

\begin{itemize}
    \item Measure one of firms
    \item Firm $i \in[0,1]$ produces differentiated product $Y_i$ with technology
\end{itemize}

\begin{align}
    Y_i=A_i K_i^\alpha L_i^{1-\alpha}
\end{align}

\subsubsection{Final Sector}

\begin{itemize}
    \item Competitive market 
    \item We assume $\sigma > 1$ in the production function as there's no equilibrium otherwise 
\end{itemize}

Production Function:

\begin{align}
    Y=\left(\int Y_i^{\frac{\sigma-1}{\sigma}} d i\right)^{\frac{\sigma}{\sigma-1}}
\end{align}

\paragraph{Maximization Problem}

The final sector solves:

\begin{align}
    &\max _{\left\{Y_i\right\}} Y-\int P_i Y_i d i\\
    &\text{s.t.}\\
    &Y=\left(\int Y_i^{\frac{\sigma-1}{\sigma}} d i\right)^{\frac{\sigma}{\sigma-1}}
\end{align}

FOC for $i$:

\begin{align}
    Y_i=Y \times P_i^{-\sigma}
\end{align}

\subsubsection{Household}

\begin{itemize}
    \item Consumers buy final goods
    \item Inelastic endowment of capital and labor
\end{itemize}

\subsubsection{Equilibrium}

\paragraph{Consumers}


Prices $\left\{P_i\right\}_i, r, w$, allocations $\left\{Y_i, \Pi_i, K_i, L_i\right\}_i, C, Y$ such that:

Consumers own firms and get profits $\int \Pi_i d i$, supply labor $\bar{L}$ and capital $\bar{K}$ inelastically at $w$ and $r$ and solve

\begin{align}
    &\max _C U(C) \\
    &\text { s.t. } \\
    &C=w \bar{L}+r \bar{K}+\int \Pi_i d i
\end{align}

\paragraph{Final Firms}

Final goods firms take prices $\left\{P_i\right\}$ as given and solve

\begin{align}
    &\max _{\left\{Y_i\right\}_i, Y} Y-\int P_i Y_i d i \\
    &\text { s.t. } \\
    &Y=\left(\int Y_i^{\frac{\sigma-1}{\sigma}} d i\right)^{\frac{\sigma}{\sigma-1}}
\end{align}

which produces demand for good $i$ as $Y_i\left(P_i\right)=Y P_i^{-\sigma}$

\paragraph{Intermediate Firms}

Intermediate firms take $w$ and $Y_i\left(P_i\right)$ as given and solve
$$
\Pi_i=\max _{P_i, Y_i, L_i, K_i} P_i Y_i-w L_i-R K_i
$$
s.t.
$$
\begin{aligned}
& Y_i=Y P_i^{-\sigma} \\
& Y_i=A_i K_i^\alpha L_i^{1-\alpha}
\end{aligned}
$$

\paragraph{Market Clearing}

\begin{align}
    C=Y, \int K_i d i=\bar{K}, \int L_i d i=\bar{L}
\end{align}


\subsubsection{Social Planner's}

Social planner chooses allocations subject to feasibilities:
$$
\max _{Y,\left\{Y_i, K_i, L_i\right\}_i} U(Y)
$$
s.t.
$$
\begin{gathered}
Y=\left(\int Y_i^{\frac{\sigma-1}{\sigma}} d i\right)^{\frac{\sigma}{\sigma-1}}, \\
Y_i=A_i K_i^\alpha L_i^{1-\alpha}, \\
\int K_i d i=\bar{K}, \quad \int L_i d i=\bar{L}
\end{gathered}
$$


\subsection{Results}

\subsubsection{Optimal Price Equation}

\begin{align}
    P_i=\underbrace{\frac{\sigma}{\sigma-1}}_{\text {mark up }>1} \times \underbrace{\frac{1}{A_i}\left(\frac{R}{\alpha}\right)^\alpha\left(\frac{w}{1-\alpha}\right)^{1-\alpha}}_{\text {marginal cost, } \lambda_i}
\end{align}

\begin{itemize}
    \item All firms charge the same mark up $\frac{\sigma}{\sigma-1}$ over marginal costs
        \begin{itemize}
            \item perfect competition limit as $\sigma \rightarrow \infty$
        \end{itemize}
\end{itemize}

%%%%%%%%%%%%%%%%%%%%%%%%%%%%%%%%%%%%%%%%%%%%%%%%%%%%%%%%%%%%%%%%%%%%%%%%%%%%%%%%%%%%%%%
%%%%%%%%%%%%%%%%%%%%%%%%%%%%%%%%%%%%%%%%%%%%%%%%%%%%%%%%%%%%%%%%%%%%%%%%%%%%%%%%%%%%%%%
%%%%%%%%%%%%%%%%%%%%%%%%%%%%%%%%%%%%%%%%%%%%%%%%%%%%%%%%%%%%%%%%%%%%%%%%%%%%%%%%%%%%%%%

\section{Lecture 9}

\subsection{Terms}

\subsection{Setup}

\subsection{Results}

\subsubsection{TFP of Intermediate Goods Sector}

- Value added of the intermediate good sector is
$$
\int P_i Y_i d i=Y
$$
- We would measure sectoral TFP in the data from
$$
Y=T F P \times L
$$
- What is this TFP?
$$
Y=\left(\int\left(A_i L_i\right)^{\frac{\sigma-1}{\sigma}} d i\right)^{\frac{\sigma}{\sigma-1}}=\underbrace{\left(\int\left[A_i \frac{L_i}{L}\right]^{\frac{\sigma-1}{\sigma}} d i\right)^{\frac{\sigma}{\sigma-1}}}_{=T F P} \times L
$$

%%%%%%%%%%%%%%%%%%%%%%%%%%%%%%%%%%%%%%%%%%%%%%%%%%%%%%%%%%%%%%%%%%%%%%%%%%%%%%%%%%%%%%%
%%%%%%%%%%%%%%%%%%%%%%%%%%%%%%%%%%%%%%%%%%%%%%%%%%%%%%%%%%%%%%%%%%%%%%%%%%%%%%%%%%%%%%%
%%%%%%%%%%%%%%%%%%%%%%%%%%%%%%%%%%%%%%%%%%%%%%%%%%%%%%%%%%%%%%%%%%%%%%%%%%%%%%%%%%%%%%%

\section{Lecture 9}

\subsection{Terms}

\begin{itemize}
    \item $A_i$: technology for firm $i$
    \item $a_i$: $\ln A_i-\mathbb{E} \ln A_i$: Log of technology for firm $i$ minus the expected log of technology across firms
    \item $\bar{a}=\mathbb{E} \ln A_i$: Expected log of technology across firms
    \item $\tau_i$: Distortion faced by firm $i$
    \item $t_i:=\ln \left(1+\tau_i\right)-\mathbb{E} \ln \left(1+\tau_i\right)$
\end{itemize}

%%%%%%%%%%%%%%%%%%%%%%%%%%%%%%%%%%%%%%%%%%%%%%%%%%%%%%%%%%%%%%%%%%%%%%%%%%%%%%%%%%%%%%%
%%%%%%%%%%%%%%%%%%%%%%%%%%%%%%%%%%%%%%%%%%%%%%%%%%%%%%%%%%%%%%%%%%%%%%%%%%%%%%%%%%%%%%%
%%%%%%%%%%%%%%%%%%%%%%%%%%%%%%%%%%%%%%%%%%%%%%%%%%%%%%%%%%%%%%%%%%%%%%%%%%%%%%%%%%%%%%%

\section{Lecture 10}

\subsection{Terms}

\begin{itemize}
    \item $x_{ij}$: inputs from firm $i$ used by firm $j$
    \item $Y$: Output of the final good sector
        \begin{itemize}
            \item Aggregates all of the N producers
        \end{itemize}
    \item $\mathcal{D}\left(c_1, \ldots, c_N\right)$: The production function for the final good sector
    \item $N$: Number of producers
    \item $p_1, \ldots, p_N$: Prices of products
    \item $F$: Number of factors
    \item $\bar{L}_f$: Inelastic supply of factors
    \item $w_1, \ldots, w_F$: Prices of factors
    \item $F_i$: Production function for producer $i$
    \item $A_i$: Technology for producer $i$
    \item $y_i$: Output of producer $i$
        \begin{itemize}
            \item $y_i=A_i F_i\left(\left\{x_{i j}\right\}_{j=1}^{N+F}\right)$
        \end{itemize}
\end{itemize}

\subsection{Model}

\subsubsection{Producer's Problem}

Each firm is competitive and operates
$$
\pi_i=\max _{y_i,\left\{x_{i j}\right\}_{j=1}^{N+F}} p_i y_i-\sum_{j=1}^N p_j x_{i j}-\sum_{f=1}^F w_f x_{i f}
$$
s.t.
$$
y_i=A_i F_i\left(\left\{x_{i j}\right\}_{j=1}^{N+F}\right)
$$
where $A_i$ is productivity and $F$ is CRS

\subsubsection{Final Good Sector Problem}

Final good sector is competitive and solves
$$
\Pi=\max _{Y,\left\{c_j\right\}_{j=1}^N} Y-\sum_{j=1}^N p_j c_j
$$
s.t.
$$
Y=\mathcal{D}\left(c_1, \ldots, c_N\right)
$$
where $\mathcal{D}$ is CRS.

\subsubsection{Consumer Problem}

Consumers solve
$$
\max _C U(C)
$$
s.t.
$$
C=\sum_{f=1}^F w_f \bar{L}_f
$$

\subsubsection{Competitive Equilibrium}

- CE is $C, Y,\left\{c_j, y_j\right\}_{j=1}^N,\left\{x_{i j}\right\}_{i=1 \ldots N, j=1 \ldots N+F}\left\{p_i\right\}_{i=1}^N,\left\{w_f\right\}_{f=1}^F$ such that
- Consumers, final good sector, all firms solve their problems
- Markets clear
$$
\begin{gathered}
C=Y, \\
y_i=c_i+\sum_{j=1}^N x_{j i} \text { for all } i=1, \ldots, N \\
\bar{L}_f=\sum_{j=1}^N x_{j f} \text { for all } f=1, \ldots, F
\end{gathered}
$$

\subsubsection{Social Planner's Problem}

$$
\max _{C,\left\{c_i, y_i, x_{i j}\right\}} U(C)
$$
s.t.
$$
\begin{gathered}
C=\mathcal{D}\left(c_1, \ldots, c_N\right), \\
y_i=A_i F_i\left(\left\{x_{i j}\right\}_{j=1}^{N+F}\right) \text { for all } i, \\
y_i=c_i+\sum_{j=1}^N x_{j i} \text { for all } i=1, \ldots, N, \\
\bar{L}_f=\sum_{j=1}^N x_{j f} \text { for all } f=1, \ldots, F .
\end{gathered}
$$

Note that $y_i=c_i+\sum_{j=1}^N x_{j i} \text { for all } i=1, \ldots, N$ 
is saying that $y_i$ must equal the amount of good $y_i$ consumed 
by the final good sector, $c_i$, and the 
amount of good $i$ consumed by other firms, $x_{j i}$ for each firm $j$.

$A_i F_i\left(\left\{x_{i j}\right\}_{j=1}^{N+F}\right)$ meanwhile 
says that $y_i$ must equal the amount of good $i$ produced by firm $i$
under their level of technology and production function.

%%%%%%%%%%%%%%%%%%%%%%%%%%%%%%%%%%%%%%%%%%%%%%%%%%%%%%%%%%%%%%%%%%%%%%%%%%%%%%%%%%%%%%%
%%%%%%%%%%%%%%%%%%%%%%%%%%%%%%%%%%%%%%%%%%%%%%%%%%%%%%%%%%%%%%%%%%%%%%%%%%%%%%%%%%%%%%%
%%%%%%%%%%%%%%%%%%%%%%%%%%%%%%%%%%%%%%%%%%%%%%%%%%%%%%%%%%%%%%%%%%%%%%%%%%%%%%%%%%%%%%%

\section{Lecture 11}

\subsection{Terms}

\begin{itemize}
    \item $\beta$: discount factor
    \item $U(.)$: utility function
    \item $C_t$: consumption in period $t$
    \item $L_t$: labor in period $t$
    \item $Y_t$: output in period $t$
    \item $K_t$: capital in period $t$
    \item $A_t$: technology in period $t$
    \item $s_t$: exogenous state in period $t$
    \item $s^t = (s_0, s_1, \ldots, s_t)$: history of exogenous states up to period $t$
    \item $\operatorname{Pr}\left(s^t\right)$: the probability of the history of realizations comprising $s_t$
    \item $\operatorname{Pr}\left(s^{t+1} \mid s^t\right)=\operatorname{Pr}\left(s^{t+1}\right) / \operatorname{Pr}\left(s^t\right)$: the conditional probability of $s^{t+1}$ given $s^t$
    \item $F_t$: production function at time $t$
        \begin{itemize}
            \item $F_t\left(K_t, L_t\right)=A_t K_t^\alpha L_t^{1-\alpha}$
        \end{itemize}
    \item $R_{t+1}^{r f}$: Risk free interest rate in period $t+1$
        \begin{itemize}
            \item $R_{t+1}^{r f} = \frac{1}{Q_t} - 1$
        \end{itemize}
\end{itemize}

%%%%%%%%%%%%%%%%%%%%%%%%%%%%%%%%%%%%%%%%%%%%%%%%%%%%%%%%%%%%%%%%%%%%%%%%%%%%%%%%%%%%%%%

\subsection{Setup}

\subsubsection{Household}

We have the representative consumer:

\begin{align}
    \mathbb{E}_0 \sum_{t=0}^{\infty} \beta^t U\left(C_t, L_t\right)
\end{align}



%%%%%%%%%%%%%%%%%%%%%%%%%%%%%%%%%%%%%%%%%%%%%%%%%%%%%%%%%%%%%%%%%%%%%%%%%%%%%%%%%%%%%%%

\subsubsection{Firm}

We have the representative firm:
\begin{align}
    Y_t=F_t\left(K_t, L_t\right)=A_t K_t^\alpha L_t^{1-\alpha}
\end{align}
our feasibility constraint is given by:
\begin{align}
    C_t+K_{t+1}=Y_t+(1-\delta) K_t
\end{align}

$A_t$ is the only exogenous stochastic process here.


%%%%%%%%%%%%%%%%%%%%%%%%%%%%%%%%%%%%%%%%%%%%%%%%%%%%%%%%%%%%%%%%%%%%%%%%%%%%%%%%%%%%%%%


\subsubsection{Market Qualities}

\begin{itemize}
    \item The market is complete (for simplicity)
\end{itemize}

%%%%%%%%%%%%%%%%%%%%%%%%%%%%%%%%%%%%%%%%%%%%%%%%%%%%%%%%%%%%%%%%%%%%%%%%%%%%%%%%%%%%%%%

\subsubsection{Conceptualization of Our Exogenous Shocks and Endogenous Variables}

Let $s_t$ be exogenous state ($s_t=A_t$ in our RBC example with one shock, but more generally it is a vector of all exogenous shocks)

Recall that we denote $s^t=\left(s_0, \ldots, s_t\right)$ to be a history of shocks up to period $t$

Then:

\begin{itemize}
    \item $\operatorname{Pr}\left(s^t\right)$ is the probability of the history of realizations comprising $s_t$
    \item $\operatorname{Pr}\left(s^{t+1} \mid s^t\right)=\operatorname{Pr}\left(s^{t+1}\right) / \operatorname{Pr}\left(s^t\right)$ is the conditional probability
\end{itemize}

Endogenous variables in period $t$ are functions of $s^t: X_t=X_t\left(s^t\right)$


%%%%%%%%%%%%%%%%%%%%%%%%%%%%%%%%%%%%%%%%%%%%%%%%%%%%%%%%%%%%%%%%%%%%%%%%%%%%%%%%%%%%%%%


\subsubsection{Social Planner's Problem}

The Social Planner's problem is given by:

\begin{align}
    &\max _{C_t, N_t, K_{t+1}} \sum_{t=0}^{\infty} \sum_{s^t} \operatorname{Pr}\left(s^t\right) \beta^t U\left(C_t\left(s^t\right), L_t\left(s^t\right)\right) \label{eq:l11_sp_problem_long} \\
    &\text{s.t.} \\
    &K_0 \text{ is given} \\
    &C_t\left(s^t\right)+K_{t+1}\left(s^t\right)=A_t K_t\left(s^{t-1}\right)^\alpha L_t\left(s^t\right)^{1-\alpha}+(1-\delta) K_t\left(s^{t-1}\right)
\end{align}

\begin{itemize}
    \item Let $\beta^t \operatorname{Pr}\left(s^t\right) \lambda_t\left(s^t\right)$ be the Lagrange multiplier on this constraint 
        \begin{itemize}
            \item Note that each Lagrange multiplier is specific to each possible history
        \end{itemize}
\end{itemize}

Then our FOCs are:\footnote{
    \color{red}
    I'm a bit confused what the full Lagrangian looks like this in case.
    \color{black}
}

\begin{align} 
    & U_C\left(s^t\right)=\lambda_t\left(s^t\right) \\ 
    & U_L\left(s^t\right)=-\lambda_t\left(s^t\right) F_L\left(s^t\right) \\ 
    & \lambda_t\left(s^t\right)=\beta \sum_{s^{t+1}} \operatorname{Pr}\left(s^{t+1} \mid s^t\right)\left(1+F_K\left(s^{t+1}\right)-\delta\right) \lambda_{t+1}\left(s^{t+1}\right)
\end{align}


%%%%%%%%%%%%%%%%%%%%%%%%%%%%%%%%%%%%%%%%%%%%%%%%%%%%%%%%%%%%%%%%%%%%%%%%%%%%%%%%%%%%%%%

\paragraph{Properties of SP's Problem}

Combining our FOC's yields:

\begin{align}
    U_L\left(s^t\right) & =-U_C\left(s^t\right) F_L\left(s^t\right) \\
    U_C\left(s^t\right) & =\beta \sum_{s^{t+1}} \operatorname{Pr}\left(s^{t+1} \mid s^t\right) U_C\left(s^{t+1}\right)\left(1+F_K\left(s^{t+1}\right)-\delta\right)
\end{align}

Together with feasibility (and TVC)
$$
C_t\left(s^t\right)+K_{t+1}\left(s^t\right)=A_t K_t\left(s^{t-1}\right)^\alpha L_t\left(s^t\right)^{1-\alpha}+(1-\delta) K_t\left(s^{t-1}\right)
$$
these three equations pin down (stochastic) solution $\left(C_t\left(s^t\right), K_{t+1}\left(s^t\right), L_t\left(s^t\right)\right)_{t, s^t}$


%%%%%%%%%%%%%%%%%%%%%%%%%%%%%%%%%%%%%%%%%%%%%%%%%%%%%%%%%%%%%%%%%%%%%%%%%%%%%%%%%%%%%%%

\paragraph{Shorthand Notation for the SP's Problem}

It is common to use shorthand notation for the SP problem in \eqref{eq:l11_sp_problem_long} and write it as:

\begin{align}
    \max _{C_t, L_t, K_{t+1}} \mathbb{E}_0 \sum_{t=0}^{\infty} \beta^t U\left(C_t, L_t\right)
\end{align}

s.t. $K_0$ given and
$$
C_t+K_{t+1}=A_t K_t^\alpha L_t^{1-\alpha}+(1-\delta) K_t
$$

The corresponding short-hand FOCs are then:
$$
\begin{aligned}
U_{L, t} & =-U_{C, t} F_{L, t} \\
U_{C, t} & =\beta \mathbb{E}_t U_{C, t+1}\left(1+F_{K, t+1}-\delta\right) .
\end{aligned}
$$

%%%%%%%%%%%%%%%%%%%%%%%%%%%%%%%%%%%%%%%%%%%%%%%%%%%%%%%%%%%%%%%%%%%%%%%%%%%%%%%%%%%%%%%

\paragraph{Recursive Formulation}

Suppose $A_t$ is an $\mathrm{AR}(1)$ process. Then the SP problem can be written as
$$
V(K, A)=\max _{C, L, K_{+}} U(C, L)+\beta\sum_{A_{+}} \operatorname{Pr}\left(A_{+} \mid A\right) V\left(K_{+}, A_{+}\right)
$$
s.t.
$$
C+K_{+}=A K^\alpha L^{1-\alpha}+(1-\delta) K
$$

Solution is policy functions $\widetilde{X}(K, A)=\widetilde{C}(K, A), \widetilde{N}(K, A), \widetilde{K}_{+}(K, A)$ and for any history of shocks $s^t=A^t$ we can find $X_t\left(s^t\right)$ by substituting these policy functions recursively, e.g.
$$
\begin{aligned}
& X_0\left(s_0\right)=\widetilde{X}\left(K_0, A_0\right) \\
& X_1\left(s_0, s_1\right)=\widetilde{X}\left(\widetilde{K}_{+}\left(K_0, A_0\right), A_1\right) \\
& \ldots
\end{aligned}
$$

%%%%%%%%%%%%%%%%%%%%%%%%%%%%%%%%%%%%%%%%%%%%%%%%%%%%%%%%%%%%%%%%%%%%%%%%%%%%%%%%%%%%%%%

\subsubsection{Competitive Equilibrium (Simplest Version)}

\paragraph{Setup}

\begin{itemize}
    \item The consumer
        \begin{itemize}
            \item own all capital
            \item rent capital and labor to firms
            \item trade risk-free bonds (a security that is bought in period $t$ at price $Q_t$ and pays 1 unit of consumption good in period $t+1$ )
        \end{itemize}
    \item Risk-free bond is in zero net supply
\end{itemize}

\paragraph{Competitive Equilibrium Characterization}

Competitive equilibrium are stochastic sequences $\left\{C_t, K_{t+1}, L_t, B_t\right\}_t$ and prices $\left\{W_t, R_t, R_t^{r f}\right\}_t$ such that

\paragraph{Household}

Consumers solve
$$
\max _{C_t, L_t, K_{t+1}} \mathbb{E}_0 \sum_{t=0}^{\infty} \beta^t U\left(C_t, L_t\right)
$$
s.t. $K_0$ given, $B_0=0$ and
$$
C_t+K_{t+1}+Q_t B_{t+1}=W_t L_t+\left(1+R_t-\delta\right) K_t+B_t
$$


\paragraph{Firm}

Firms solve
$$
\max _{K_t, L_t} A_t K_t^\alpha L_t^{1-\alpha}-W_t L_t-R_t K_t
$$

\paragraph{Market Clearing}

Markets clear
$$
B_t=0
$$


\paragraph{Risk Free Interest Rate}

Let risk-free interest rate $R_{t+1}^{r f}$ be defined by
$$
R_{t+1}^{r f}:=1 / Q_t-1
$$

\begin{itemize}
    \item Note that saving in period $t$ one unit of consumption in
        \begin{itemize}
            \item bonds pays interest rate $1+R_{t+1}^{r f}$ in period $t+1$
            \item capital pays interest rate $1+R_{t+1}-\delta$ in period $t+1$
            \item excess return of capital $r x_{t+1}:=\left(R_{t+1}-\delta\right)-R_{t+1}^{r f}$
        \end{itemize}
\end{itemize}

%%%%%%%%%%%%%%%%%%%%%%%%%%%%%%%%%%%%%%%%%%%%%%%%%%%%%%%%%%%%%%%%%%%%%%%%%%%%%%%%%%%%%%%

\subsection{Results}


%%%%%%%%%%%%%%%%%%%%%%%%%%%%%%%%%%%%%%%%%%%%%%%%%%%%%%%%%%%%%%%%%%%%%%%%%%%%%%%%%%%%%%%
%%%%%%%%%%%%%%%%%%%%%%%%%%%%%%%%%%%%%%%%%%%%%%%%%%%%%%%%%%%%%%%%%%%%%%%%%%%%%%%%%%%%%%%
%%%%%%%%%%%%%%%%%%%%%%%%%%%%%%%%%%%%%%%%%%%%%%%%%%%%%%%%%%%%%%%%%%%%%%%%%%%%%%%%%%%%%%%

\section{Lecture 12}

\subsection{Terms}

\begin{itemize}
    \item $P_t$: Price of final good in period $t$ (nominal -- don't normalize to 1)
    \item $P_{i,t}$: Price of product of firm $i$ in period $t$ (nominal)
    \item $A_{i,t}$: Technology of firm $i$ in period $t$
        \begin{itemize}
            \item $A = A_{i,t}$: Fix technology initially
            \item Later may consider stochastic technology
        \end{itemize}
    \item $L_{i, t}$: Labor of firm $i$ in period $t$
    \item $W_t$: Nominal wage in period t
    \item $\frac{W_t}{P_t}$: Real wage in period t
    \item $D_{i, t}$: Dividends of firm $i$ in period $t$
        \begin{itemize}
            \item $D_{i, t}=P_{i, t} Y_{i, t}-W_t L_{i, t}-\Phi_{i, t}$
        \end{itemize}
    \item $\Phi_{i, t}$: Cost of setting price $P_{i,t}$ in period $t$
        \begin{itemize}
            \item $\Phi_{i, t}=\frac{\theta}{2} P_t Y_t\left(\frac{P_{i, t}}{P_{i, t-1}}-1\right)^2$
        \end{itemize}
    \item $\theta \geq 0$: parameter capturing how costly it is for firm $i$ to change its price
    \item $Q_t$: Price of a nominal one period risk-free bond purchased at $t$.
    \item $I_t$: Nominal interest rates in period $t$ (controlled by Central Bank)
        \begin{itemize}
            \item $I_t=\frac{1}{Q_{t}}-1$
            \item Initially, we treat $I_t$ (or equivalently $Q_t$) as 
            an arbitrary stochastic process
        \end{itemize}
    \item $B_t$: Bonds held in period $t$
    \item $\Pi_t$: Inflation in period $t$
        \begin{itemize}
            \item $\Pi_t=P_t / P_{t-1}$
        \end{itemize}
\end{itemize}

\subsection{Model}

\subsubsection{Equilibrium}

\paragraph{Consumers}

Stochastic sequences $\left\{C_t, L_t, B_t, D_{i, t}, \Phi_{i, t}\right\}_{i, t}$, prices $\left\{P_t, P_{i, t}, W_t, Q_t, \Pi_t\right\}$ such that

Consumers solve

\begin{align}
    &\max _{\left\{C_t, L_t, B_t\right\}} \mathbb{E}_0 \sum_{t=0}^{\infty} \beta^t\left[\frac{C_t^{1-\gamma}}{1-\gamma}-\frac{L_t^{1+\varphi}}{1+\varphi}\right] \\
    & \text{s.t. } B_{-1}=0 \text{ and } \\
    & P_t C_t+Q_t B_t=W_t L_t+\int D_{i, t} d i+\int \Phi_{i, t} d i+B_{t-1}
\end{align}


\paragraph{Final Goods Market}

Final goods firms solve
$$
\max _{\left\{Y_i\right\}_i, Y} P_t Y_t-\int P_i Y_i d i
$$
s.t.
$$
Y_t=\left(\int Y_{i, t}^{\frac{\sigma-1}{\sigma}} d i\right)^{\frac{\sigma}{\sigma-1}}
$$

\paragraph{Intermediate Goods Market}

Easiest to write problem of intermediate firms in recursive form:
$$
V_t\left(P_{i, t-1}\right)=\max _{P_{i, t}, Y_{i, t}, L_{i, t}} P_{i, t} Y_{i, t}-W_t L_{i, t}-\Phi_{i, t}+Q_t \mathbb{E}_t V_{t+1}\left(P_{i, t}\right)
$$
s.t.
$$
\begin{gathered}
Y_{i, t}=A L_{i, t}, \quad Y_{i, t}=\left(\frac{P_{i, t}}{P_t}\right)^{-\sigma} Y_t \\
\Phi_{i, t}=\frac{\theta}{2} P_t Y_t\left(\frac{P_{i, t}}{P_{i, t-1}}-1\right)^2
\end{gathered}
$$
where dividends equal
$D_{i, t}=P_{i, t} Y_{i, t}-W_t L_{i, t}-\Phi_{i, t}$

\paragraph{Monetary Policy}

$Q_t$ is a given stochastic process

\paragraph{Market Clearing}

\begin{align}
    C_t=Y_t \\
    L_t=\int L_{i, t} d i \\
    B_t=0
\end{align}


\subsection{Optimality}

\subsubsection{Consumer Optimality}

\color{red}

It would be nice to 
get some intuition 
for the intratemporal 
and intertemporal 
optimality conditions.

\color{black}

\paragraph{Intratemporal Optimality}

\begin{align}
    C_t^\gamma L_t^{\varphi}=\frac{W_t}{P_t}
\end{align}

\paragraph{Intertemporal Optimality}

The inter-temporal optimality conditions
$$
\begin{aligned}
Q_t & =\beta \mathbb{E}_t \frac{P_t}{P_{t+1}}\left(\frac{C_{t+1}}{C_t}\right)^{-\gamma} \\
& =\beta \mathbb{E}_t \frac{1}{\Pi_{t+1}}\left(\frac{C_{t+1}}{C_t}\right)^{-\gamma}
\end{aligned}
$$

\subsubsection{Intermediate Firm Optimality}

Firm Problem:

\begin{align}
        V_t\left(P_{i, t-1}\right) & =\max _{P_{i, t}} P_{i, t}\left(\frac{P_{i, t}}{P_t}\right)^{-\sigma} Y_t-\frac{W_t}{A}\left(\frac{P_{i, t}}{P_t}\right)^{-\sigma} Y_t-\frac{\theta}{2} P_t Y_t\left(\frac{P_{i, t}}{P_{i, t-1}}-1\right)^2 
        +Q_t \mathbb{E}_t V_{t+1}\left(P_{i, t}\right)
\end{align}

FOC:

\begin{align}
    \begin{gathered}
        (1-\sigma)\left(\frac{P_{i, t}}{P_t}\right)^{-\sigma} Y_t+\sigma \frac{W_t}{A}\left(\frac{P_{i, t}}{P_t}\right)^{-\sigma} Y_t \frac{1}{P_{i, t}} \\
        -\theta P_t Y_t\left(\frac{P_{i, t}}{P_{i, t-1}}-1\right) \frac{1}{P_{i, t-1}}+Q_t \mathbb{E}_t \frac{\partial}{\partial P_{i, t}} V_{t+1}\left(P_{i, t}\right)=0
        \end{gathered}
\end{align}

How do we handle it when we have a partial derivative of the value function 
in our FOC? We use the envelope theorem to take the derivative 
of $V_t$ wrt our relevant state variable holding our choice variable 
fixed at the optimal choice.

Envelope Theorem:

\begin{align}
    \frac{\partial}{\partial P_{i, t}} V_{t+1}\left(P_{i, t}\right)=\theta P_{t+1} Y_{t+1}\left(\frac{P_{i, t+1}}{P_{i, t}}-1\right) \frac{P_{i, t+1}}{P_{i, t}^2}
\end{align}

All firms are identical, which implies that (see Lecture 8) that $P_{i, t}=P_t$ for all $i$
This simplifies our equation to
$$
(1-\sigma)+\sigma \frac{W_t / P_t}{A}-\theta \Pi_t\left(\Pi_t-1\right)+\theta Q_t \mathbb{E}_t\left(\Pi_{t+1}-1\right) \Pi_{t+1}^2 \frac{Y_{t+1}}{Y_t}=0
$$

\subsubsection{Equilibrium}

Combine previous equations and get rid of redundant variables. Then we get that 
$\left\{C_t, Q_t, L_t, W_t / P_t, \Pi_t\right\}_t$ are a competitive equilibrium if and only if they solve
$$
\begin{gathered}
(1-\sigma)+\sigma \frac{W_t / P_t}{A}-\theta \Pi_t\left(\Pi_t-1\right)+\theta Q_t \mathbb{E}_t\left(\Pi_{t+1}-1\right) \Pi_{t+1}^2 \frac{C_{t+1}}{C_t}=0 \\
C_t^\gamma L_t^{\varphi}=\frac{W_t}{P_t} \\
C_t=A L_t \\
Q_t=\beta \mathbb{E}_t \frac{1}{\Pi_{t+1}}\left(\frac{C_{t+1}}{C_t}\right)^{-\gamma} \\
\left\{Q_t\right\}_t \text { is given }
\end{gathered}
$$

\subsection{Sticky Prices}

\subsubsection{Steady State}

We say that 
$\left\{C_t, L_t, W_t / P_t, L_t, Q_t, \Pi_t\right\}_t$ 
is steady state if it is constant and 
independent of $t$

\subsubsection{Zero Inflation Steady State}

We have zero inflation if:

\begin{align}
    \Pi_t=1
\end{align}

Observe that by setting $Q_t=\bar{Q}:=\beta$ the central bank can attain such steady state:

$$
\begin{gathered}
\bar{Q}=\beta, \quad \bar{\Pi}=1, \\
\bar{C}^{\gamma+\varphi}=\frac{\sigma-1}{\sigma}, \bar{C}=A \bar{L}, \overline{W / P}=\frac{\sigma-1}{\sigma} .
\end{gathered}
$$


\section{Lecture 13}

\subsection{Terms}

\begin{itemize}
    \item $\beta$: discount factor
    \item $A_t$: Technology/productivity in period $t$
    \item $\epsilon_{a,t}$: Shocks to technology/productivity in period $t$
    \item $a_t = \ln(A_t)$: Log of technology/productivity in period $t$
        \begin{itemize}
            \item $a_t = \rho_a a_{t-1} + \epsilon_{a,t}$
        \end{itemize}
    \item $I_t$: (Nominal?) interest rate in period $t$
    \item $\Pi_t$: Inflation in period $t$
    \item $\pi_t$: Log inflation in period $t$
    \item $v_{v,t}$: Monetary policy shock in period $t$
    \item $\iota_t = \ln(1 + I_t)$: Log of 1 + interest rate
        \begin{itemize}
            \item $\iota_t$: $-\ln(\beta) + \eta \pi_t + v_t$
            \item $v_t = \rho_v v_{t-1} + \epsilon_{v,t}$
        \end{itemize}
    \item $x_t$: the vector of endogenous variables
    \item $z_t$: the vector of exogenous processes, $\left\{z_{i, t}\right\}_i$
    \item $(\bar{x}, \bar{z})$: Steady state of the system without aggregate shocks
    \item $\hat{x}_t:=x_t-\bar{x}$: The endogenous variables deviation from the steady state 
    \item $\hat{z}_t:=z_t-\bar{z}$: The exogenous variables deviation from the steady state
    \item $\sigma$: the elasticity of substitution for the final good between intermediate goods
    \item $\theta \geq 0$ : parameter capturing how costly it is for firm $i$ to change its price
    \item $\zeta = \frac{\sigma-1}{\theta}$: 
\end{itemize}

\subsection{Model}

\subsubsection{Equilibrium Conditions from Lecture 12}

$$
\begin{gathered}
(1-\sigma)+\sigma \frac{W_t / P_t}{A_t}-\theta \Pi_t\left(\Pi_t-1\right)+\theta Q_t \mathbb{E}_t\left(\Pi_{t+1}-1\right) \Pi_{t+1}^2 \frac{C_{t+1}}{C_t}=0 \\
C_t^{\varphi+\gamma} A_t^{-\varphi}=\frac{W_t}{P_t} \\
\frac{1}{1+I_t}=\beta \mathbb{E}_t \frac{1}{\Pi_{t+1}}\left(\frac{C_{t+1}}{C_t}\right)^{-\gamma}
\end{gathered}
$$
$\left\{I_t\right\}_t$ is given in previous slide

\subsubsection{Log Version of Equilibrium Conditions}

$\begin{gathered}
    (1-\sigma)+\sigma \exp \left(\left(w_t-p_t\right)-a_t\right)-\theta \exp \pi_t\left(\exp \pi_t-1\right) \\ 
    +\theta \exp \left(-l_t\right) \mathbb{E}_t \exp \left(2 \pi_{t+1}+c_{t+1}-c_t\right)\left(\exp \pi_{t+1}-1\right)=0 \\ 
    \\
    (\varphi+\gamma) c_t-\varphi a_t=w_t-p_t \\ 
    \\
    -l_t=\ln \beta+\ln \mathbb{E}_t \exp \left(-\pi_{t+1}-\gamma\left(c_{t+1}-c_t\right)\right), \\ 
    \iota_t=-\ln \beta+\eta \pi_t+v_t\end{gathered}$

\subsubsection{Stochastic Processes}

Stochastic processes follow an AR(1) process:
$$
\begin{aligned}
a_t & =\rho_a a_{t-1}+\varepsilon_{a, t} \\
v_t & =\rho_v v_{t-1}+\varepsilon_{v, t}
\end{aligned}
$$

\end{document}