\documentclass[10pt]{article}
\usepackage{amsmath}
\usepackage{amsthm}
\usepackage{amsfonts}
\usepackage{amssymb}
%\usepackage[version=4]{mhchem}
\usepackage{amssymb}
%\\usepackage{stmaryrd}
\setlength\parindent{0pt}
\usepackage[margin=1.2in]{geometry}
\usepackage{enumitem}
\usepackage{xcolor}
\usepackage{hyperref}
\setcounter{tocdepth}{4}

% Set the length of \parskip to add a line between paragraphs
\setlength{\parskip}{1em}

\usepackage{mathtools}
\mathtoolsset{showonlyrefs=true}


\DeclareMathSymbol{\Perp}{\mathrel}{symbols}{"3F}

\newtheorem{theorem}{Theorem}[section]  % Numbered within sections
\newtheorem{definition}[theorem]{Definition} % Definitions share numbering with theorems
\newtheorem{proposition}[theorem]{Proposition}  % Propositions share numbering with theorems



\newcounter{example}

\newenvironment{example}
{
  \stepcounter{example}
  \noindent\textbf{Example \thesection.\theexample.}
}
{
  \par
}


% deeper section command
% This will let you go one level deeper than whatever section level you're on.
\makeatletter
\newcommand{\deepersection}[1]{%
  \ifnum\value{subparagraph}>0
    % Already at the deepest standard level (\subparagraph), cannot go deeper
    \subparagraph{#1}
  \else
    \ifnum\value{paragraph}>0
      \subparagraph{#1}
    \else
      \ifnum\value{subsubsection}>0
        \paragraph{#1}
      \else
        \ifnum\value{subsection}>0
          \subsubsection{#1}
        \else
          \ifnum\value{section}>0
            \subsection{#1}
          \else
            \section{#1}
          \fi
        \fi
      \fi
    \fi
  \fi
}
\makeatother



% same section command
% This will let create a section at the same level as whatever section level you're on.
\makeatletter
\newcommand{\samesection}[1]{%
  \ifnum\value{subparagraph}>0
    \subparagraph{#1}
  \else
    \ifnum\value{paragraph}>0
      \paragraph{#1}
    \else
      \ifnum\value{subsubsection}>0
        \subsubsection{#1}
      \else
        \ifnum\value{subsection}>0
          \subsection{#1}
        \else
          \ifnum\value{section}>0
            \section{#1}
          \else
            % Default to section if outside any sectioning
            \section{#1}
          \fi
        \fi
      \fi
    \fi
  \fi
}
\makeatother






\title{Macro 3 Notes}

\author{}
\date{}

\begin{document}
\maketitle

\tableofcontents

\section{Introduction}

\section{Overlapping Generations Economy}

\subsection{Terms}

\input{../input/olg_terms.tex}

\subsection{Pure Exchange Economy}

\subsubsection{Competitive Equilibrium}

The price vector $p$ is an element of $R^{\infty}$, so that

$$
p=\left(p_1, p_2, p_3, \ldots\right)
$$

The agent problem is

$$
\max _x u^i(x)
$$
subject to

$$
p x \leq p e^i
$$

which, since generation $i$ neither consume nor has endowments at time $t \neq i$ or $t \neq i+1$, can be specialized as

$$
\max _{x_i, x_{i+1}} v^i\left(x_i, x_{i+1}\right)
$$

subject to

$$
p_i x_i+p_{i+1} x_{i+1}=p_i e_i^i+p_{i+1} e_{i+1}^i
$$
and for generation $i=0$ as

$$
\max _{x_1} x_1 \text { subject to } p_1 x_1=p_1 e_1^0 \text {. }
$$

\subsubsection{No Trade}

\begin{proposition} 
      The only competitive equilibrium has

      $$
      x^i=e^i
      $$

      i.e. there is no trade in equilibrium.
\end{proposition}

\subsubsection{Equilibrium Prices}

Normalize

\begin{align}
    p_1=1
\end{align}

We have:

\begin{align}
    \frac{p_{i+1}}{p_i}=\frac{v_2^i\left(e_i^i, e_{i+1}^i\right)}{v_1^i\left(e_i^i, e_{i+1}^i\right)}
\end{align}

for all $i \geq 1$

With $r_t$ net interest rate,
\begin{align}
    \frac{1}{1+r_t}=\frac{p_{t+1}}{p_t}
\end{align}

for all $t$. From our previous condition we have
$$
r_t=\frac{v_1^t\left(e_t^t, e_{i+1}^t\right)}{v_2^t\left(e_t^t, e_{t+1}^t\right)}-1
$$
for all $t>1$ and
$$
p_t=\frac{1}{\left(1+r_1\right)\left(1+r_2\right) \cdots\left(1+r_{t-1}\right)} .
$$

\subsubsection{Equilibrium Prices Under Specified Utility Function}

Taking utility function

$$
v^i\left(c_y, c_0\right)=(1-\beta) \log c_y+\beta \log c_0
$$

we have:

\begin{align}
    r_t \equiv \bar{r}=\frac{(1-\beta)}{\beta} \frac{\alpha}{1-\alpha}-1=\frac{\alpha-\beta}{\beta(1-\alpha)} 
\end{align}

or 

\begin{align}
    p_t=\left[\frac{\beta}{(1-\beta)} \frac{1-\alpha}{\alpha}\right]^{t-1} \text { for } t \geq 1
\end{align}

\subsubsection{Best Symmetric Allocation}

We will solve for the best feasible symmetric allocation, where best is for the point of view of the young. In particular, consider the problem
$$
\max _{c_y, c_o} v\left(c_y, c_o\right)=\max _{c_y, c_o}(1-\beta) \log c_y+\beta \log c_o
$$
subject to
$$
c_y+c_o=1
$$

Its sufficient first order condition is given by
$$
\frac{\beta}{1-\beta} \frac{c_y}{c_0}=1
$$
so the solution of this f.o.c. that also is feasible, i.e. the solution of the problem is
$$
c_y=1-\beta, c_o=\beta .
$$

The best symmetric allocation depends on $\beta$ in this way because for higher preference parameter $\beta$ agents give less weight to consumption when young and more weight to consumption when old.

\subsubsection{Comparison of CE and Best Symmetric Allocation}

We will compare the utility of the unique competitive equilibrium allocation
$$
\bar{c}_i^i=1-\alpha, \quad \bar{c}_{i+1}^i=\alpha \text { for } i \geq 1 \text { and } \bar{c}_1^0=\alpha
$$
with the one for the best symmetric allocation
$$
c_i^{* i}=1-\beta, c_{i+1}^{* i}=\beta \text { for } i \geq 1 \text { and } c_1^{* 0}=\beta
$$

Notice that, since the CE allocation has $x^i=e^i$, and since that allocation is a feasible symmetric allocation, then, unless $c^*=\bar{c}$-which happens only when $\alpha=\beta-$ the best symmetric feasible allocation is strictly preferred by the agents of generations $i=1,2, \ldots$. It only remains to compare the utility of the initial old, i.e. generation $i=0$, between the best symmetric and CE allocations.

\begin{itemize}
    \item Case 1: $\beta > \alpha$: All generations prefer Best Symmetric Allocation
    \item Case 2: $\beta = \alpha$: Indifferent between CE and Best Symmetric Allocation
    \item Case 3: $\beta < \alpha$: Original Generation prefers CE
\end{itemize}

\subsection{Social Security}

\subsubsection{After-Tax Endowments}

\begin{align}
    e_i^i & =(1-\alpha)-\tau \text { and } e_{i+1}^i=\alpha+\tau \text { for all } i \geq 1 \\
    e_1^0 & =\alpha+\tau
\end{align}

Notice that by suitable choice of $\tau$ we can make the after-tax endowments equal to the best symmetric allocations, the required $\tau$ is
$$
\tau=\beta-\alpha
$$

\subsection{Growing Economy}

\subsubsection{Setup}

We will now consider an economy with population and productivity growth. Let $N_t$ the number of young agents at time $t$. Let $n$ be the growth rate of population, so that
$$
N_{t+1}=(1+n) N_t \text { for } t \geq 1 \text { and } N_0=1 .
$$

Let $g$ denote the growth rate of productivity of the endowments of each cohort, so that
$$
e_{t+1}^{t+1}=(1+g) e_t^t \text { and } e_{t+2}^{t+1}=(1+g) e_{t+1}^t
$$
so that
$$
\begin{aligned}
e_t^t & =(1+g)^t(1-\alpha) \\
e_{t+1}^t & =(1+g)^t \alpha
\end{aligned}
$$
for all $t \geq 1$.

\subsubsection{Feasible Symmetric Allocation}


Define the feasible symmetric allocations as those solving
$$
N_t c_y^t+N_{t-1} c_o^t=N_t(1-\alpha)(1+g)^t+N_{t-1} \alpha(1+g)^{t-1}
$$
where each agent born at time $t$ and young at $t$ consumes
$$
c_y^t=\hat{c}_y(1+g)^t,
$$
and each agent born at time $t-1$ and old at $t$ consumes
$$
c_o^t=\hat{c}_o(1+g)^{t-1} .
$$

Notice that this constraint can be written as
$$
\hat{c}_y(1+g)(1+n)+\hat{c}_o=(1-\alpha)(1+g)(1+n)+\alpha
$$

\section{OLG Perpetual Youth Model}

\subsection{Terms}

\input{../input/olg_py_terms.tex}

\subsection{Setup}

Agents that die replaced by newborns. Thus, 
adding all cohort alive at time $t$ yields:

\begin{align}
    \int_{-\infty}^t N(s, t) d s=\int_{-\infty}^t p e^{-p(t-s)} d s=1 .
\end{align}

\subsection{Household Problem}

\begin{align}
    \max \mathbb{E}\left[\int_t^{\infty} u(c(z)) e^{-\theta(z-t)} d z\right]=\int_t^{\infty} \log (c(z)) e^{-(p+\theta)(z-t)} d z
\end{align}

subject to 

\begin{align}
    \int_t^{\infty}[c(z)-y(z)] R(t, z) d z=v(t)
\end{align}

We define human wealth as:

\begin{align}
    h(t)=\int_t^{\infty} y(z) R(t, z) d z
\end{align}

and find that the solution to our problem is:

\begin{align}
    c(t)=(\theta+p)(v(t)+h(t))
\end{align}

\end{document}