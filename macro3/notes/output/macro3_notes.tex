\documentclass[10pt]{article}
\usepackage{amsmath}
\usepackage{amsthm}
\usepackage{amsfonts}
\usepackage{amssymb}
%\usepackage[version=4]{mhchem}
\usepackage{amssymb}
%\\usepackage{stmaryrd}
\setlength\parindent{0pt}
\usepackage[margin=1.2in]{geometry}
\usepackage{enumitem}
\usepackage{xcolor}
\usepackage{hyperref}
\setcounter{tocdepth}{4}

% Set the length of \parskip to add a line between paragraphs
\setlength{\parskip}{1em}

\usepackage{mathtools}
\mathtoolsset{showonlyrefs=true}


\DeclareMathSymbol{\Perp}{\mathrel}{symbols}{"3F}

\newtheorem{theorem}{Theorem}[section]  % Numbered within sections
\newtheorem{definition}[theorem]{Definition} % Definitions share numbering with theorems
\newtheorem{proposition}[theorem]{Proposition}  % Propositions share numbering with theorems



\newcounter{example}

\newenvironment{example}
{
  \stepcounter{example}
  \noindent\textbf{Example \thesection.\theexample.}
}
{
  \par
}


% deeper section command
% This will let you go one level deeper than whatever section level you're on.
\makeatletter
\newcommand{\deepersection}[1]{%
  \ifnum\value{subparagraph}>0
    % Already at the deepest standard level (\subparagraph), cannot go deeper
    \subparagraph{#1}
  \else
    \ifnum\value{paragraph}>0
      \subparagraph{#1}
    \else
      \ifnum\value{subsubsection}>0
        \paragraph{#1}
      \else
        \ifnum\value{subsection}>0
          \subsubsection{#1}
        \else
          \ifnum\value{section}>0
            \subsection{#1}
          \else
            \section{#1}
          \fi
        \fi
      \fi
    \fi
  \fi
}
\makeatother



% same section command
% This will let create a section at the same level as whatever section level you're on.
\makeatletter
\newcommand{\samesection}[1]{%
  \ifnum\value{subparagraph}>0
    \subparagraph{#1}
  \else
    \ifnum\value{paragraph}>0
      \paragraph{#1}
    \else
      \ifnum\value{subsubsection}>0
        \subsubsection{#1}
      \else
        \ifnum\value{subsection}>0
          \subsection{#1}
        \else
          \ifnum\value{section}>0
            \section{#1}
          \else
            % Default to section if outside any sectioning
            \section{#1}
          \fi
        \fi
      \fi
    \fi
  \fi
}
\makeatother






\title{Macro 3 Notes}

\author{}
\date{}

\begin{document}
\maketitle

\tableofcontents

\section{Introduction}

\section{General Equilibrium Theory}

\subsection{Terms}

\input{../input/gen_eq_terms.tex}

\subsection{Feasible Allocations}

\begin{definition}[Feasible Allocation] 
    
    A feasible allocation, $\{x^i, y^j\}$, satisfies three conditions:

    \begin{enumerate}
        \item Each consumer can consume $x^i$, 
            \begin{align}
                x^i \in X^i, \forall i \in I
            \end{align}
        \item Each firm can produce $y^j$,
            \begin{align}
                y^j \in Y^j, \forall j \in J
            \end{align}
        \item Demand equals supply:
            \begin{align}
                \sum_{i \in I} x^i=\sum_{j \in J} y^j + \sum_{i \in I} e^i
            \end{align}
    \end{enumerate}

\end{definition}

\subsection{Key Concepts for Competitive Equilibrium}

Suppose there are $m$ commodities (i.e., $L = \mathbb{R}^m$).

Then the price vector $p$ is given by:

\begin{align}
    p=\left(p_1, \ldots, p_m\right)
\end{align}

and we have the following equation:

\begin{align}
    p \cdot x=p_1 x_1+p_2 x_2+\cdots+p_m x_m \equiv \sum_{s=1}^m p_s x_s
\end{align}

$p \cdot x$ is the value of the commodity vector $x$
in terms of the numeraire\footnote{By ``in terms of the numeraire,''
I just mean that we are normalizing the price of some good (usually the first 
element in $x$) to be 1 and 
then expressing the price of the other goods relative to that one.}

A quick useful quality to note:

\begin{align}
    p \cdot x^a+p \cdot x^b=p \cdot\left(x^a+x^b\right)
\end{align}

\subsection{Competitive Equilibrium}

\begin{definition}[Competitive Equilibrium] 
    We denote prices by a vector $p \in R^m$. 
    
    A competitive equilibrium is a price vector $p$, and a feasible allocation 
    $\left\{x^i, y^j\right\}$ such that

    \begin{enumerate}
        \item each firm $j \in J$ maximize its profits. We denote the profit of firm $j$ by $\pi^j$.
    \end{enumerate}

\end{definition}

\subsection{Welfare Theorems}

\begin{definition}[Local Non-Satiation] 
    Utility function $u^i: \mathbf{X}^i \rightarrow \mathbb{R}$ satisfies the 
    local non-satiation (LNS) property if, for any 
    $\mathbf{x} \in \mathbf{X}^i$ and any neighborhood of 
    $\mathbf{x}$, denote $B_{\varepsilon}(\mathbf{x})$, there 
    exists $\hat{\mathbf{x}} \in B_{\varepsilon}(\mathbf{x})$ 
    such that $u^i(\hat{\mathbf{x}})>u^i(\mathbf{x})$.
\end{definition}

\begin{theorem}[First Welfare Theorem]
    Suppose that $u^i$ satisfies local non-satiation for all $i \in \mathbf{I}$. 
    Let $\left\{\mathbf{p}, \overline{\mathbf{x}}^i, \overline{\mathbf{y}}^j\right\}$ 
    be a competitive equilibrium. 
    Then, $\left\{\overline{\mathbf{x}}^i, \overline{\mathbf{y}}^j\right\}$ is a Pareto optimal allocation.
\end{theorem}

\begin{definition}[Convexity] 
    A function $u^i: \mathbf{X}^i \rightarrow \mathbb{R}$ is strictly convex if, for any $\mathbf{x}, \mathbf{x}^{\prime} \in \mathbf{X}^i$ such that $\mathbf{x}^{\prime} \neq \mathbf{x}$, we have
    $$
    u^i\left(\theta \mathbf{x}+(1-\theta) \mathbf{x}^{\prime}\right)>\theta u^i(\mathbf{x})+(1-\theta) u^i\left(\mathbf{x}^{\prime}\right), \forall \theta \in(0,1) .
    $$
\end{definition}

\begin{definition}[Quasiconcavity] 
    A function $u^i: \mathbf{X}^i \rightarrow \mathbb{R}$ is (strictly) quasiconcave if its upper contour set $\left\{x \in \mathbf{X}^i: u^i(\mathbf{x}) \geq u^i(\overline{\mathbf{x}})\right\}$ is (strictly) convex for all $i \in \mathbf{I}$ and all $\overline{\mathbf{x}} \in \mathbf{X}^i$. Equivalently, for any $\mathbf{x} \neq \overline{\mathbf{x}}$, we must have
    $$
    u^i(\alpha \mathbf{x}+(1-\alpha) \tilde{\mathbf{x}})>\min \left\{u^i(\mathbf{x}), u^i(\overline{\mathbf{x}})\right\}, \forall \alpha \in(0,1)
    $$
\end{definition}

\begin{theorem}[Second Welfare Theorem]
    Assumption 1. (Assumption HH) Assume that $\mathbf{X}^i$ are convex for all $i \in \mathbf{I}$ and that $u^i: \mathbf{X}^i \rightarrow \mathbb{R}$ are continuous and strictly quasiconcave.

    Assumption 2. (Assumption FF) Assume that the aggregate production sumset of the economy is convex; i.e.
    $$
    \mathbf{Y}:=\left\{\mathbf{y} \in \mathbf{L}: \mathbf{y}=\sum_{j=1}^J \mathbf{y}^j, \mathbf{y}^j \in \mathbf{Y}^j, \forall j \in \mathbf{J}\right\} .
    $$

    Let $\left\{\overline{\mathbf{x}}^i, \overline{\mathbf{y}}^j\right\}$ be a Pareto optimal allocation. Then, there exists a price vector $\mathbf{p}$ such that:
    (i) all firms maximise profits such that, for all $j \in \mathbf{J}$,
    $$
    \mathbf{p} \cdot \overline{\mathbf{y}}^j \geq \mathbf{p} \cdot \mathbf{y}, \forall \mathbf{y} \in \mathbf{Y}^j
    $$
    (ii) given allocation $\left\{\overline{\mathbf{x}}^i\right\}$, consumers minimise expenditure subject to attaining at least the same utility obtained by consuming $\overline{\mathbf{x}}^i$; i.e.
    $$
    \overline{\mathbf{x}}^i \in \underset{\mathbf{x} \in \mathbf{X}^i}{\arg \min } \mathbf{p} \cdot \mathbf{x} \quad \text { s.t. } \quad u^i(\mathbf{x}) \geq u^i\left(\overline{\mathbf{x}}^i\right) .
    $$
\end{theorem}

\section{Aggregation}

\subsection{Terms}

\input{../input/aggregation_terms.tex}

\section{Overlapping Generations Economy}

\subsection{Terms}

\input{../input/olg_terms.tex}

\subsection{Pure Exchange Economy}

\subsubsection{Competitive Equilibrium}

The price vector $p$ is an element of $R^{\infty}$, so that

$$
p=\left(p_1, p_2, p_3, \ldots\right)
$$

The agent problem is

$$
\max _x u^i(x)
$$
subject to

$$
p x \leq p e^i
$$

which, since generation $i$ neither consume nor has endowments at time $t \neq i$ or $t \neq i+1$, can be specialized as

$$
\max _{x_i, x_{i+1}} v^i\left(x_i, x_{i+1}\right)
$$

subject to

$$
p_i x_i+p_{i+1} x_{i+1}=p_i e_i^i+p_{i+1} e_{i+1}^i
$$
and for generation $i=0$ as

$$
\max _{x_1} x_1 \text { subject to } p_1 x_1=p_1 e_1^0 \text {. }
$$

\subsubsection{No Trade}

\begin{proposition} 
      The only competitive equilibrium has

      $$
      x^i=e^i
      $$

      i.e. there is no trade in equilibrium.
\end{proposition}

\subsubsection{Equilibrium Prices}

Normalize

\begin{align}
    p_1=1
\end{align}

We have:

\begin{align}
    \frac{p_{i+1}}{p_i}=\frac{v_2^i\left(e_i^i, e_{i+1}^i\right)}{v_1^i\left(e_i^i, e_{i+1}^i\right)}
\end{align}

for all $i \geq 1$

With $r_t$ net interest rate,
\begin{align}
    \frac{1}{1+r_t}=\frac{p_{t+1}}{p_t}
\end{align}

for all $t$. From our previous condition we have
$$
r_t=\frac{v_1^t\left(e_t^t, e_{i+1}^t\right)}{v_2^t\left(e_t^t, e_{t+1}^t\right)}-1
$$
for all $t>1$ and
$$
p_t=\frac{1}{\left(1+r_1\right)\left(1+r_2\right) \cdots\left(1+r_{t-1}\right)} .
$$

\subsubsection{Equilibrium Prices Under Specified Utility Function}

Taking utility function

$$
v^i\left(c_y, c_0\right)=(1-\beta) \log c_y+\beta \log c_0
$$

we have:

\begin{align}
    r_t \equiv \bar{r}=\frac{(1-\beta)}{\beta} \frac{\alpha}{1-\alpha}-1=\frac{\alpha-\beta}{\beta(1-\alpha)} 
\end{align}

or 

\begin{align}
    p_t=\left[\frac{\beta}{(1-\beta)} \frac{1-\alpha}{\alpha}\right]^{t-1} \text { for } t \geq 1
\end{align}

\subsubsection{Best Symmetric Allocation}

We will solve for the best feasible symmetric allocation, where best is for the point of view of the young. In particular, consider the problem
$$
\max _{c_y, c_o} v\left(c_y, c_o\right)=\max _{c_y, c_o}(1-\beta) \log c_y+\beta \log c_o
$$
subject to
$$
c_y+c_o=1
$$

Its sufficient first order condition is given by
$$
\frac{\beta}{1-\beta} \frac{c_y}{c_0}=1
$$
so the solution of this f.o.c. that also is feasible, i.e. the solution of the problem is
$$
c_y=1-\beta, c_o=\beta .
$$

The best symmetric allocation depends on $\beta$ in this way because for higher preference parameter $\beta$ agents give less weight to consumption when young and more weight to consumption when old.

\subsubsection{Comparison of CE and Best Symmetric Allocation}

We will compare the utility of the unique competitive equilibrium allocation
$$
\bar{c}_i^i=1-\alpha, \quad \bar{c}_{i+1}^i=\alpha \text { for } i \geq 1 \text { and } \bar{c}_1^0=\alpha
$$
with the one for the best symmetric allocation
$$
c_i^{* i}=1-\beta, c_{i+1}^{* i}=\beta \text { for } i \geq 1 \text { and } c_1^{* 0}=\beta
$$

Notice that, since the CE allocation has $x^i=e^i$, and since that allocation is a feasible symmetric allocation, then, unless $c^*=\bar{c}$-which happens only when $\alpha=\beta-$ the best symmetric feasible allocation is strictly preferred by the agents of generations $i=1,2, \ldots$. It only remains to compare the utility of the initial old, i.e. generation $i=0$, between the best symmetric and CE allocations.

\begin{itemize}
    \item Case 1: $\beta > \alpha$: All generations prefer Best Symmetric Allocation
    \item Case 2: $\beta = \alpha$: Indifferent between CE and Best Symmetric Allocation
    \item Case 3: $\beta < \alpha$: Original Generation prefers CE
\end{itemize}

\subsection{Social Security}

\subsubsection{After-Tax Endowments}

\begin{align}
    e_i^i & =(1-\alpha)-\tau \text { and } e_{i+1}^i=\alpha+\tau \text { for all } i \geq 1 \\
    e_1^0 & =\alpha+\tau
\end{align}

Notice that by suitable choice of $\tau$ we can make the after-tax endowments equal to the best symmetric allocations, the required $\tau$ is
$$
\tau=\beta-\alpha
$$

\subsection{Growing Economy}

\subsubsection{Setup}

We will now consider an economy with population and productivity growth. Let $N_t$ the number of young agents at time $t$. Let $n$ be the growth rate of population, so that
$$
N_{t+1}=(1+n) N_t \text { for } t \geq 1 \text { and } N_0=1 .
$$

Let $g$ denote the growth rate of productivity of the endowments of each cohort, so that
$$
e_{t+1}^{t+1}=(1+g) e_t^t \text { and } e_{t+2}^{t+1}=(1+g) e_{t+1}^t
$$
so that
$$
\begin{aligned}
e_t^t & =(1+g)^t(1-\alpha) \\
e_{t+1}^t & =(1+g)^t \alpha
\end{aligned}
$$
for all $t \geq 1$.

\subsubsection{Feasible Symmetric Allocation}


Define the feasible symmetric allocations as those solving
$$
N_t c_y^t+N_{t-1} c_o^t=N_t(1-\alpha)(1+g)^t+N_{t-1} \alpha(1+g)^{t-1}
$$
where each agent born at time $t$ and young at $t$ consumes
$$
c_y^t=\hat{c}_y(1+g)^t,
$$
and each agent born at time $t-1$ and old at $t$ consumes
$$
c_o^t=\hat{c}_o(1+g)^{t-1} .
$$

Notice that this constraint can be written as
$$
\hat{c}_y(1+g)(1+n)+\hat{c}_o=(1-\alpha)(1+g)(1+n)+\alpha
$$

\section{OLG Perpetual Youth Model}

\subsection{Terms}

\input{../input/olg_py_terms.tex}

\subsection{Setup}

Agents that die replaced by newborns. Thus, 
adding all cohort alive at time $t$ yields:

\begin{align}
    \int_{-\infty}^t N(s, t) d s=\int_{-\infty}^t p e^{-p(t-s)} d s=1 .
\end{align}

\subsection{Insurance, Annuities}

Invest $v$ at $t$, gets $v \frac{1+\Delta r}{1-p \Delta}$ if alive at $t+\Delta$, and zero if dead.

Continuous time (as $\Delta \downarrow 0$ ) : $v \frac{1+\Delta r}{1-p \Delta}=v+v(r+p) \Delta+o(\Delta)$

\subsection{Household Problem}

\begin{align}
    \max \mathbb{E}\left[\int_t^{\infty} u(c(z)) e^{-\theta(z-t)} d z\right]=\int_t^{\infty} \log (c(z)) e^{-(p+\theta)(z-t)} d z
\end{align}

subject to 

\begin{align}
    \int_t^{\infty}[c(z)-y(z)] R(t, z) d z=v(t)
\end{align}

\subsection{Budget Constraint and Human Wealth}

The individual's dynamic budget constraint is:

\begin{align}
    \frac{d v(t)}{d t}=(r(t)+p) v(t)+y(t)-c(t)
\end{align}

We also have a no-Ponizi-game (NPG) condition:

\begin{align}
    \lim _{z \rightarrow \infty} v(z) R(t, z)=0
\end{align}

The price of a good in time $z$ in terms of goods in time $t$ is
given by:

\begin{align}
    R(t, z):=\exp \left[-\int_t^z(r(\mu)+p) d \mu\right]
\end{align}

We can also integrate the dynamic budget constraint to get
the intertemporal budget constraint:

\begin{align}
    v(t)=\int_t^{\infty}[c(z)-y(z)] R(t, z) d z
\end{align}

We define human wealth as the present value of 
all future income, i.e.,:

\begin{align}
    h(t)=\int_t^{\infty} y(z) R(t, z) d z
\end{align}

with the boundary condition

\begin{align}
    \lim _{z \rightarrow \infty} h(z) R(t, z)=0
\end{align}

which is equivalent to:

\begin{align}
    \frac{d h(z)}{d z}=[r(z)+p] h(z)-y(z) \text { with } \lim _{z \rightarrow \infty} R(t, z) h(z)=0
\end{align}

\subsection{Optimal Consumption}

We find that the solution to our problem is:

\begin{align}
    c(t)=(\theta+p)(v(t)+h(t))
\end{align}

The law of motion (or, Euler equation) for consumption is:

\begin{align}
    \frac{d c(t)}{d t}=(r(t)-\theta) c(t)
\end{align}

\begin{notes}
    A few miscellaneous points from Tak's notes:
    \begin{itemize}
        \item We will later find that in the steady state, the
            interest rate must be higher than our discount rate, i.e., $r > \theta$.
            We could re-frame this as the interest rate must be higher than 
            our impatience, hence savings will accumulate over time.
        \item Consumption is independent of the interest rate. This 
            is due to the assumption of log utility, which implies 
            that the substitution and income effect from changes in 
            the interest rate exactly offset each other.
    \end{itemize}
\end{notes}

\subsection{Aggregation}

\subsubsection{Distribution of Labor Income}

Given aggregate labor income at the time $t$, $Y(t)$, we have:

\begin{align}
    y(s, t)=\frac{\alpha+p}{p} Y(t) e^{-\alpha(t-s)}, \alpha \geq 0
\end{align}

as the expression for the labor income in time $t$ of a living agent born at time $s$.

This means that, at any particular point in time $t$, the share of $Y(t)$ endowed to generation $s$ falls at the rate $\alpha$ as $s$ increases.

\subsubsection{Aggregate Human Wealth}

The aggregate human wealth is defined as

\begin{align}
    H(t)&:=\int_{-\infty}^t h(s, t) N(s, t) d s \\
    &=\int_{-\infty}^t h(s, t) p e^{-p(t-s)} d s \\
    &= \int_{-\infty}^t a p\left[\int_t^{\infty} Y(z) e^{-\alpha(z-t)} R(t, z) d z\right] e^{-\alpha(t-s)} e^{-p(t-s)} d s \\
    &=\int_t^{\infty} Y(z) \exp \left\{-\int_t^z(\alpha+p+r(\mu)) d \mu\right\} d z
\end{align}

We can then get:

\begin{align}
    \frac{d H(t)}{d t}&=(r(t)+p+\alpha) H(t)-Y(t)
\end{align}

\subsubsection{Aggregate Non-Human Wealth}

The aggregate non-human wealth is defined as

\begin{align}
    V(t)&:=\int_{-\infty}^t N(s, t) v(s, t) d s \\
    &=\int_{-\infty}^t v(s, t) p e^{-p(t-s)} d s
\end{align}

From which we can get:

\begin{align}
    \frac{d V(t)}{d t}=r(t) V(t)+Y(t)-C(t)
\end{align}

\subsubsection{Aggregate Consumption}

Thus, aggregate consumption is defined as

$$
C(t):=\int_{-\infty}^t N(s, t) c(s, t) d s .
$$

We may also aggregate consumption as

$$
C(t)=(p+\theta)(H(t)+V(t)) .
$$

We can then get:

\begin{align}
    \frac{d C(t)}{d t}=(r(t)+\alpha-\theta) C(t)-(p+\theta)(p+\alpha) V(t)
\end{align}

\subsubsection{Aggregation Summary}

We have now derived the dynamics that describe the aggregate behaviour in the Perpetual Youth Model:

\begin{align}
    & \frac{d H(t)}{d t}=(r(t)+p+\alpha) H(t)-Y(t) \\
    & \frac{d V(t)}{d t}=r(t) V(t)+Y(t)-C(t) \\
    & \frac{d C(t)}{d t}=(r(t)+\alpha-\theta) C(t)-(p+\theta)(p+\alpha) V(t) \\
    &C(t)=(p+\theta)(H(t)+V(t))
\end{align}

We also need a no-Ponzi-game condition, which ensures that agents have finite wealth; i.e.

$$
\lim _{T \rightarrow \infty} Y(t) \exp \left[-\int_t^T(r(z)+\alpha+p) d z\right]=0 .
$$

\subsection{Pure Endowment}

\subsubsection{Labor Income and Consumption}

In a pure endowment economy, we have:

\begin{align}
    Y(t)=C(t)=Y
\end{align}

\subsubsection{Non-human Wealth}

In aggregate, we have:
$V(t)=0$, since some agents borrow and others lend:
$$
V(t)=\int_{-\infty}^t N(s, t) v(s, t) d s=0 .
$$

$v(s, t)$ value of cumulated savings (net assets) of cohort born at $s$ at $t$.

\subsubsection{Equilibrium Interest Rate}

Equilibrium interest rate $r(t)=\theta-\alpha$

\subsubsection{Individual Consumption}

\begin{align}
    \frac{d c(s, t)}{d t}=[r(t)-\theta] c(s, t)=-\alpha c(s, t)
\end{align}

\subsubsection{Equilibrium Outcome}

Thus equilibrium is autarky! $c(s, t)=y(s, t)$ and $v(s, t)=0$

\subsubsection{Results Summary}

\begin{align}
    Y(t) & =C(t)=Y, \\
    V(t) & =0 \\
    r(t) & =\theta-\alpha, \\
    c(s, t) & =y(s, t), \forall t \geq s, \\
    v(s, t) & =0, \forall t \geq s .
\end{align}

\subsection{Capital Accumulation, Technology}




\end{document}