\documentclass[10pt]{article}
\usepackage{amsmath}
\usepackage{amsthm}
\usepackage{amsfonts}
\usepackage{amssymb}
%\usepackage[version=4]{mhchem}
\usepackage{amssymb}
%\\usepackage{stmaryrd}
\setlength\parindent{0pt}
\usepackage[margin=1.2in]{geometry}
\usepackage{enumitem}
\usepackage{xcolor}
\usepackage{hyperref}
\setcounter{tocdepth}{4}

% Set the length of \parskip to add a line between paragraphs
\setlength{\parskip}{1em}

\usepackage{mathtools}
\mathtoolsset{showonlyrefs=true}


\DeclareMathSymbol{\Perp}{\mathrel}{symbols}{"3F}

\newtheorem{theorem}{Theorem}[section]  % Numbered within sections
\newtheorem{definition}[theorem]{Definition} % Definitions share numbering with theorems
\newtheorem{proposition}[theorem]{Proposition}  % Propositions share numbering with theorems



\newcounter{example}

\newenvironment{example}
{
  \stepcounter{example}
  \noindent\textbf{Example \thesection.\theexample.}
}
{
  \par
}


% deeper section command
% This will let you go one level deeper than whatever section level you're on.
\makeatletter
\newcommand{\deepersection}[1]{%
  \ifnum\value{subparagraph}>0
    % Already at the deepest standard level (\subparagraph), cannot go deeper
    \subparagraph{#1}
  \else
    \ifnum\value{paragraph}>0
      \subparagraph{#1}
    \else
      \ifnum\value{subsubsection}>0
        \paragraph{#1}
      \else
        \ifnum\value{subsection}>0
          \subsubsection{#1}
        \else
          \ifnum\value{section}>0
            \subsection{#1}
          \else
            \section{#1}
          \fi
        \fi
      \fi
    \fi
  \fi
}
\makeatother



% same section command
% This will let create a section at the same level as whatever section level you're on.
\makeatletter
\newcommand{\samesection}[1]{%
  \ifnum\value{subparagraph}>0
    \subparagraph{#1}
  \else
    \ifnum\value{paragraph}>0
      \paragraph{#1}
    \else
      \ifnum\value{subsubsection}>0
        \subsubsection{#1}
      \else
        \ifnum\value{subsection}>0
          \subsection{#1}
        \else
          \ifnum\value{section}>0
            \section{#1}
          \else
            % Default to section if outside any sectioning
            \section{#1}
          \fi
        \fi
      \fi
    \fi
  \fi
}
\makeatother






\title{Macro 3 Notes}

\author{}
\date{}

\begin{document}
\maketitle

\tableofcontents

\section{Introduction}

\section{General Equilibrium Theory}

\subsection{Terms}

\input{../input/gen_eq_terms.tex}

\subsection{Feasible Allocations}

\begin{definition}[Feasible Allocation] 
    
    A feasible allocation, $\{x^i, y^j\}$, satisfies three conditions:

    \begin{enumerate}
        \item Each consumer can consume $x^i$, 
            \begin{align}
                x^i \in X^i, \forall i \in I
            \end{align}
        \item Each firm can produce $y^j$,
            \begin{align}
                y^j \in Y^j, \forall j \in J
            \end{align}
        \item Demand equals supply:
            \begin{align}
                \sum_{i \in I} x^i=\sum_{j \in J} y^j + \sum_{i \in I} e^i
            \end{align}
    \end{enumerate}

\end{definition}

\subsection{Key Concepts for Competitive Equilibrium}

Suppose there are $m$ commodities (i.e., $L = \mathbb{R}^m$).

Then the price vector $p$ is given by:

\begin{align}
    p=\left(p_1, \ldots, p_m\right)
\end{align}

and we have the following equation:

\begin{align}
    p \cdot x=p_1 x_1+p_2 x_2+\cdots+p_m x_m \equiv \sum_{s=1}^m p_s x_s
\end{align}

$p \cdot x$ is the value of the commodity vector $x$
in terms of the numeraire\footnote{By ``in terms of the numeraire,''
I just mean that we are normalizing the price of some good (usually the first 
element in $x$) to be 1 and 
then expressing the price of the other goods relative to that one.}

A quick useful quality to note:

\begin{align}
    p \cdot x^a+p \cdot x^b=p \cdot\left(x^a+x^b\right)
\end{align}

\subsection{Competitive Equilibrium}

\begin{definition}[Competitive Equilibrium] 
    We denote prices by a vector $p \in R^m$. 
    
    A competitive equilibrium is a price vector $p$, and a feasible allocation 
    $\left\{x^i, y^j\right\}$ such that

    \begin{enumerate}
        \item each firm $j \in J$ maximize its profits. We denote the profit of firm $j$ by $\pi^j$.
    \end{enumerate}

\end{definition}

\subsection{Welfare Theorems}

\begin{definition}[Local Non-Satiation] 
    Utility function $u^i: \mathbf{X}^i \rightarrow \mathbb{R}$ satisfies the 
    local non-satiation (LNS) property if, for any 
    $\mathbf{x} \in \mathbf{X}^i$ and any neighborhood of 
    $\mathbf{x}$, denote $B_{\varepsilon}(\mathbf{x})$, there 
    exists $\hat{\mathbf{x}} \in B_{\varepsilon}(\mathbf{x})$ 
    such that $u^i(\hat{\mathbf{x}})>u^i(\mathbf{x})$.
\end{definition}

\begin{theorem}[First Welfare Theorem]
    Suppose that $u^i$ satisfies local non-satiation for all $i \in \mathbf{I}$. 
    Let $\left\{\mathbf{p}, \overline{\mathbf{x}}^i, \overline{\mathbf{y}}^j\right\}$ 
    be a competitive equilibrium. 
    Then, $\left\{\overline{\mathbf{x}}^i, \overline{\mathbf{y}}^j\right\}$ is a Pareto optimal allocation.
\end{theorem}

\begin{definition}[Convexity] 
    A function $u^i: \mathbf{X}^i \rightarrow \mathbb{R}$ is strictly convex if, for any $\mathbf{x}, \mathbf{x}^{\prime} \in \mathbf{X}^i$ such that $\mathbf{x}^{\prime} \neq \mathbf{x}$, we have
    $$
    u^i\left(\theta \mathbf{x}+(1-\theta) \mathbf{x}^{\prime}\right)>\theta u^i(\mathbf{x})+(1-\theta) u^i\left(\mathbf{x}^{\prime}\right), \forall \theta \in(0,1) .
    $$
\end{definition}

\begin{definition}[Quasiconcavity] 
    A function $u^i: \mathbf{X}^i \rightarrow \mathbb{R}$ is (strictly) quasiconcave if its upper contour set $\left\{x \in \mathbf{X}^i: u^i(\mathbf{x}) \geq u^i(\overline{\mathbf{x}})\right\}$ is (strictly) convex for all $i \in \mathbf{I}$ and all $\overline{\mathbf{x}} \in \mathbf{X}^i$. Equivalently, for any $\mathbf{x} \neq \overline{\mathbf{x}}$, we must have
    $$
    u^i(\alpha \mathbf{x}+(1-\alpha) \tilde{\mathbf{x}})>\min \left\{u^i(\mathbf{x}), u^i(\overline{\mathbf{x}})\right\}, \forall \alpha \in(0,1)
    $$
\end{definition}

\begin{theorem}[Second Welfare Theorem]
    Assumption 1. (Assumption HH) Assume that $\mathbf{X}^i$ are convex for all $i \in \mathbf{I}$ and that $u^i: \mathbf{X}^i \rightarrow \mathbb{R}$ are continuous and strictly quasiconcave.

    Assumption 2. (Assumption FF) Assume that the aggregate production sumset of the economy is convex; i.e.
    $$
    \mathbf{Y}:=\left\{\mathbf{y} \in \mathbf{L}: \mathbf{y}=\sum_{j=1}^J \mathbf{y}^j, \mathbf{y}^j \in \mathbf{Y}^j, \forall j \in \mathbf{J}\right\} .
    $$

    Let $\left\{\overline{\mathbf{x}}^i, \overline{\mathbf{y}}^j\right\}$ be a Pareto optimal allocation. Then, there exists a price vector $\mathbf{p}$ such that:
    (i) all firms maximise profits such that, for all $j \in \mathbf{J}$,
    $$
    \mathbf{p} \cdot \overline{\mathbf{y}}^j \geq \mathbf{p} \cdot \mathbf{y}, \forall \mathbf{y} \in \mathbf{Y}^j
    $$
    (ii) given allocation $\left\{\overline{\mathbf{x}}^i\right\}$, consumers minimise expenditure subject to attaining at least the same utility obtained by consuming $\overline{\mathbf{x}}^i$; i.e.
    $$
    \overline{\mathbf{x}}^i \in \underset{\mathbf{x} \in \mathbf{X}^i}{\arg \min } \mathbf{p} \cdot \mathbf{x} \quad \text { s.t. } \quad u^i(\mathbf{x}) \geq u^i\left(\overline{\mathbf{x}}^i\right) .
    $$
\end{theorem}

\section{Aggregation}

\subsection{Terms}

\input{../input/aggregation_terms.tex}

\section{Overlapping Generations Economy}

\subsection{Terms}

\input{../input/olg_terms.tex}

\subsection{Pure Exchange Economy}

\subsubsection{Competitive Equilibrium}

The price vector $p$ is an element of $R^{\infty}$, so that

$$
p=\left(p_1, p_2, p_3, \ldots\right)
$$

The agent problem is

$$
\max _x u^i(x)
$$
subject to

$$
p x \leq p e^i
$$

which, since generation $i$ neither consume nor has endowments at time $t \neq i$ or $t \neq i+1$, can be specialized as

$$
\max _{x_i, x_{i+1}} v^i\left(x_i, x_{i+1}\right)
$$

subject to

$$
p_i x_i+p_{i+1} x_{i+1}=p_i e_i^i+p_{i+1} e_{i+1}^i
$$
and for generation $i=0$ as

$$
\max _{x_1} x_1 \text { subject to } p_1 x_1=p_1 e_1^0 \text {. }
$$

\subsubsection{No Trade}

\begin{proposition} 
      The only competitive equilibrium has

      $$
      x^i=e^i
      $$

      i.e. there is no trade in equilibrium.
\end{proposition}

\subsubsection{Equilibrium Prices}

Normalize

\begin{align}
    p_1=1
\end{align}

We have:

\begin{align}
    \frac{p_{i+1}}{p_i}=\frac{v_2^i\left(e_i^i, e_{i+1}^i\right)}{v_1^i\left(e_i^i, e_{i+1}^i\right)}
\end{align}

for all $i \geq 1$

With $r_t$ net interest rate,
\begin{align}
    \frac{1}{1+r_t}=\frac{p_{t+1}}{p_t}
\end{align}

for all $t$. From our previous condition we have
$$
r_t=\frac{v_1^t\left(e_t^t, e_{i+1}^t\right)}{v_2^t\left(e_t^t, e_{t+1}^t\right)}-1
$$
for all $t>1$ and
$$
p_t=\frac{1}{\left(1+r_1\right)\left(1+r_2\right) \cdots\left(1+r_{t-1}\right)} .
$$

\subsubsection{Equilibrium Prices Under Specified Utility Function}

Taking utility function

$$
v^i\left(c_y, c_0\right)=(1-\beta) \log c_y+\beta \log c_0
$$

we have:

\begin{align}
    r_t \equiv \bar{r}=\frac{(1-\beta)}{\beta} \frac{\alpha}{1-\alpha}-1=\frac{\alpha-\beta}{\beta(1-\alpha)} 
\end{align}

or 

\begin{align}
    p_t=\left[\frac{\beta}{(1-\beta)} \frac{1-\alpha}{\alpha}\right]^{t-1} \text { for } t \geq 1
\end{align}

\subsubsection{Best Symmetric Allocation}

We will solve for the best feasible symmetric allocation, where best is for the point of view of the young. In particular, consider the problem
$$
\max _{c_y, c_o} v\left(c_y, c_o\right)=\max _{c_y, c_o}(1-\beta) \log c_y+\beta \log c_o
$$
subject to
$$
c_y+c_o=1
$$

Its sufficient first order condition is given by
$$
\frac{\beta}{1-\beta} \frac{c_y}{c_0}=1
$$
so the solution of this f.o.c. that also is feasible, i.e. the solution of the problem is
$$
c_y=1-\beta, c_o=\beta .
$$

The best symmetric allocation depends on $\beta$ in this way because for higher preference parameter $\beta$ agents give less weight to consumption when young and more weight to consumption when old.

\subsubsection{Comparison of CE and Best Symmetric Allocation}

We will compare the utility of the unique competitive equilibrium allocation
$$
\bar{c}_i^i=1-\alpha, \quad \bar{c}_{i+1}^i=\alpha \text { for } i \geq 1 \text { and } \bar{c}_1^0=\alpha
$$
with the one for the best symmetric allocation
$$
c_i^{* i}=1-\beta, c_{i+1}^{* i}=\beta \text { for } i \geq 1 \text { and } c_1^{* 0}=\beta
$$

Notice that, since the CE allocation has $x^i=e^i$, and since that allocation is a feasible symmetric allocation, then, unless $c^*=\bar{c}$-which happens only when $\alpha=\beta-$ the best symmetric feasible allocation is strictly preferred by the agents of generations $i=1,2, \ldots$. It only remains to compare the utility of the initial old, i.e. generation $i=0$, between the best symmetric and CE allocations.

\begin{itemize}
    \item Case 1: $\beta > \alpha$: All generations prefer Best Symmetric Allocation
    \item Case 2: $\beta = \alpha$: Indifferent between CE and Best Symmetric Allocation
    \item Case 3: $\beta < \alpha$: Original Generation prefers CE
\end{itemize}

\subsection{Social Security}

\subsubsection{After-Tax Endowments}

\begin{align}
    e_i^i & =(1-\alpha)-\tau \text { and } e_{i+1}^i=\alpha+\tau \text { for all } i \geq 1 \\
    e_1^0 & =\alpha+\tau
\end{align}

Notice that by suitable choice of $\tau$ we can make the after-tax endowments equal to the best symmetric allocations, the required $\tau$ is
$$
\tau=\beta-\alpha
$$

\subsection{Growing Economy}

\subsubsection{Setup}

We will now consider an economy with population and productivity growth. Let $N_t$ the number of young agents at time $t$. Let $n$ be the growth rate of population, so that
$$
N_{t+1}=(1+n) N_t \text { for } t \geq 1 \text { and } N_0=1 .
$$

Let $g$ denote the growth rate of productivity of the endowments of each cohort, so that
$$
e_{t+1}^{t+1}=(1+g) e_t^t \text { and } e_{t+2}^{t+1}=(1+g) e_{t+1}^t
$$
so that
$$
\begin{aligned}
e_t^t & =(1+g)^t(1-\alpha) \\
e_{t+1}^t & =(1+g)^t \alpha
\end{aligned}
$$
for all $t \geq 1$.

\subsubsection{Feasible Symmetric Allocation}


Define the feasible symmetric allocations as those solving
$$
N_t c_y^t+N_{t-1} c_o^t=N_t(1-\alpha)(1+g)^t+N_{t-1} \alpha(1+g)^{t-1}
$$
where each agent born at time $t$ and young at $t$ consumes
$$
c_y^t=\hat{c}_y(1+g)^t,
$$
and each agent born at time $t-1$ and old at $t$ consumes
$$
c_o^t=\hat{c}_o(1+g)^{t-1} .
$$

Notice that this constraint can be written as
$$
\hat{c}_y(1+g)(1+n)+\hat{c}_o=(1-\alpha)(1+g)(1+n)+\alpha
$$

\section{OLG Perpetual Youth Model}

\subsection{Terms}

\input{../input/olg_py_terms.tex}

\subsection{Setup}

Agents that die replaced by newborns. Thus, 
adding all cohort alive at time $t$ yields:

\begin{align}
    \int_{-\infty}^t N(s, t) d s=\int_{-\infty}^t p e^{-p(t-s)} d s=1 .
\end{align}

\subsection{Insurance, Annuities}

Invest $v$ at $t$, gets $v \frac{1+\Delta r}{1-p \Delta}$ if alive at $t+\Delta$, and zero if dead.

Continuous time (as $\Delta \downarrow 0$ ) : $v \frac{1+\Delta r}{1-p \Delta}=v+v(r+p) \Delta+o(\Delta)$

\subsection{Household Problem}

\begin{align}
    \max \mathbb{E}\left[\int_t^{\infty} u(c(z)) e^{-\theta(z-t)} d z\right]=\int_t^{\infty} \log (c(z)) e^{-(p+\theta)(z-t)} d z
\end{align}

subject to 

\begin{align}
    \int_t^{\infty}[c(z)-y(z)] R(t, z) d z=v(t)
\end{align}

\subsection{Budget Constraint and Human Wealth}

The individual's dynamic budget constraint is:

\begin{align}
    \frac{d v(t)}{d t}=(r(t)+p) v(t)+y(t)-c(t)
\end{align}

We also have a no-Ponizi-game (NPG) condition:

\begin{align}
    \lim _{z \rightarrow \infty} v(z) R(t, z)=0
\end{align}

The price of a good in time $z$ in terms of goods in time $t$ is
given by:

\begin{align}
    R(t, z):=\exp \left[-\int_t^z(r(\mu)+p) d \mu\right]
\end{align}

We can also integrate the dynamic budget constraint to get
the intertemporal budget constraint:

\begin{align}
    v(t)=\int_t^{\infty}[c(z)-y(z)] R(t, z) d z
\end{align}

We define human wealth as the present value of 
all future income, i.e.,:

\begin{align}
    h(t)=\int_t^{\infty} y(z) R(t, z) d z
\end{align}

with the boundary condition

\begin{align}
    \lim _{z \rightarrow \infty} h(z) R(t, z)=0
\end{align}

which is equivalent to:

\begin{align}
    \frac{d h(z)}{d z}=[r(z)+p] h(z)-y(z) \text { with } \lim _{z \rightarrow \infty} R(t, z) h(z)=0
\end{align}

\subsection{Optimal Consumption}

We find that the solution to our problem is:

\begin{align}
    c(t)=(\theta+p)(v(t)+h(t))
\end{align}

The law of motion (or, Euler equation) for consumption is:

\begin{align}
    \frac{d c(t)}{d t}=(r(t)-\theta) c(t)
\end{align}

\begin{notes}
    A few miscellaneous points from Tak's notes:
    \begin{itemize}
        \item We will later find that in the steady state, the
            interest rate must be higher than our discount rate, i.e., $r > \theta$.
            We could re-frame this as the interest rate must be higher than 
            our impatience, hence savings will accumulate over time.
        \item Consumption is independent of the interest rate. This 
            is due to the assumption of log utility, which implies 
            that the substitution and income effect from changes in 
            the interest rate exactly offset each other.
    \end{itemize}
\end{notes}

\subsection{Aggregation}

\subsubsection{Distribution of Labor Income}

Given aggregate labor income at the time $t$, $Y(t)$, we have:

\begin{align}
    y(s, t)=\frac{\alpha+p}{p} Y(t) e^{-\alpha(t-s)}, \alpha \geq 0
\end{align}

as the expression for the labor income in time $t$ of a living agent born at time $s$.

This means that, at any particular point in time $t$, the share of $Y(t)$ endowed to generation $s$ falls at the rate $\alpha$ as $s$ increases.

\subsubsection{Aggregate Human Wealth}

The aggregate human wealth is defined as

\begin{align}
    H(t)&:=\int_{-\infty}^t h(s, t) N(s, t) d s \\
    &=\int_{-\infty}^t h(s, t) p e^{-p(t-s)} d s \\
    &= \int_{-\infty}^t a p\left[\int_t^{\infty} Y(z) e^{-\alpha(z-t)} R(t, z) d z\right] e^{-\alpha(t-s)} e^{-p(t-s)} d s \\
    &=\int_t^{\infty} Y(z) \exp \left\{-\int_t^z(\alpha+p+r(\mu)) d \mu\right\} d z
\end{align}

We can then get:

\begin{align}
    \frac{d H(t)}{d t}&=(r(t)+p+\alpha) H(t)-Y(t)
\end{align}

\subsubsection{Aggregate Non-Human Wealth}

The aggregate non-human wealth is defined as

\begin{align}
    V(t)&:=\int_{-\infty}^t N(s, t) v(s, t) d s \\
    &=\int_{-\infty}^t v(s, t) p e^{-p(t-s)} d s
\end{align}

From which we can get:

\begin{align}
    \frac{d V(t)}{d t}=r(t) V(t)+Y(t)-C(t)
\end{align}

\subsubsection{Aggregate Consumption}

Thus, aggregate consumption is defined as

$$
C(t):=\int_{-\infty}^t N(s, t) c(s, t) d s .
$$

We may also aggregate consumption as

$$
C(t)=(p+\theta)(H(t)+V(t)) .
$$

We can then get:

\begin{align}
    \frac{d C(t)}{d t}=(r(t)+\alpha-\theta) C(t)-(p+\theta)(p+\alpha) V(t)
\end{align}

\subsubsection{Aggregation Summary}

We have now derived the dynamics that describe the aggregate behaviour in the Perpetual Youth Model:

\begin{align}
    & \frac{d H(t)}{d t}=(r(t)+p+\alpha) H(t)-Y(t) \\
    & \frac{d V(t)}{d t}=r(t) V(t)+Y(t)-C(t) \\
    & \frac{d C(t)}{d t}=(r(t)+\alpha-\theta) C(t)-(p+\theta)(p+\alpha) V(t) \\
    &C(t)=(p+\theta)(H(t)+V(t))
\end{align}

We also need a no-Ponzi-game condition, which ensures that agents have finite wealth; i.e.

$$
\lim _{T \rightarrow \infty} Y(t) \exp \left[-\int_t^T(r(z)+\alpha+p) d z\right]=0 .
$$

\subsection{Pure Endowment}

\subsubsection{Labor Income and Consumption}

In a pure endowment economy, we have:

\begin{align}
    Y(t)=C(t)=Y
\end{align}

\subsubsection{Non-human Wealth}

In aggregate, we have:
$V(t)=0$, since some agents borrow and others lend:
$$
V(t)=\int_{-\infty}^t N(s, t) v(s, t) d s=0 .
$$

$v(s, t)$ value of cumulated savings (net assets) of cohort born at $s$ at $t$.

\subsubsection{Equilibrium Interest Rate}

Equilibrium interest rate $r(t)=\theta-\alpha$

\subsubsection{Individual Consumption}

\begin{align}
    \frac{d c(s, t)}{d t}=[r(t)-\theta] c(s, t)=-\alpha c(s, t)
\end{align}

\subsubsection{Equilibrium Outcome}

Thus equilibrium is autarky! $c(s, t)=y(s, t)$ and $v(s, t)=0$

\subsubsection{Results Summary}

\begin{align}
    Y(t) & =C(t)=Y, \\
    V(t) & =0 \\
    r(t) & =\theta-\alpha, \\
    c(s, t) & =y(s, t), \forall t \geq s, \\
    v(s, t) & =0, \forall t \geq s .
\end{align}

\subsection{Capital Accumulation, Technology}

\section{Uncertainty}

\subsection{Terms}

\input{../input/uncer_terms.tex}

\subsection{Introducing Risk Under One Physically Different Good}

Suppose there is only one physically different good. Then:

\begin{align}
    u^i(\mathbf{x})=\sum_{s=1}^m v^i\left(x_s\right) \pi_s^i
\end{align}

\begin{definition}[Risk Averse] 
    We say that $u^i$ is risk averse if:
        \begin{align}
            v^i\left(\sum_{s=1}^m x_s \pi_s^i\right)>\sum_{s=1}^m v^i\left(x_s\right) \pi_s^i
        \end{align}
    for any random variable $x$.
\end{definition}

Notice that if $m=2$, this coincides with the definition of $v$ being strictly concave and the assumption that $x$, as a random variable, is not degenerate.

Intuition: Logical that risk aversion corresponds to 
utility over consumption across several possible states of the world 
being in some sense
concave: 

\begin{theorem}
    Fix an arbitrary vector of $\lambda$-weights.   
    The corresponding Pareto optimal allocation can be described by a set of strictly increasing functions $g^i$ of the aggregate endowment, i.e. the optimal allocation can be written as
    
    $$
    x_s^i=g^i\left(\bar{e}_s\right) \text { for all } i \in I
    $$
\end{theorem}

\begin{remark}
    Now we show that the $g^i$ functions are strictly increasing. Consider two states $s$ and $s^{\prime}$ with
    $$
    \bar{e}_s>\bar{e}_{s^{\prime}}
    $$

    It must be that for at least some $i$,
    $$
    x_s^i>x_{s^{\prime}}^i .
    $$
\end{remark}

\begin{remark}
    Remark: CE and State Prices. In a CE the budget constraint of agent $i$ is
    $$
    \sum_{s=1}^m p_s x_s^i=\sum_{s=1}^m p_s e_s^i
    $$
    where $p_s$ are also referred to as state prices, the price of a security that pays one unit of the numeraire in state $s$ and zero otherwise. Thus, agents can buy consumption contingent on the state, and they finance that by selling their endowment contingent on the state.
\end{remark}

\subsubsection{State Prices}

Using the foc of agent $i$ we obtain that in an equilibrium (or in its corresponding Pareto problem)
$$
p_s=\frac{\partial v^i\left(x_s^i\right)}{\partial x} \pi_s / \mu_i=\lambda_i \frac{\partial v^i\left(g^i\left(\bar{e}_s\right)\right)}{\partial x} \pi_s
$$
so that the state prices reflect the probability that the state $s$ be realized as well as the scarcity of the aggregate endowment in state $s$.

The state prices are lower if the probability is small or if the aggregate endowment in that state is large, so that goods are relatively plentiful, and hence its marginal value relatively smaller.

\subsection{Security Markets}

We now consider a securities market.
We assume that before the state $s$ is realized agents trade 
in competitive markets where they buy and sell securities $k=1, \ldots, K$.

\subsubsection{Value of Purchases}

The first equation is given by

$$
\sum_{k=1}^K h_k^i q_k=\sum_{k=1}^K \theta_k^i q_k
$$

This says that the value of purchases is 
limited by the value of 
the sales of the securities.

\subsubsection{Equation for each State}

The second equation, indeed one for each state $s$, is given by

$$
x_s^i=\sum_{k=1}^K h_k^i d_{k s}+\hat{e}_s^i
$$

for each $s=1, \ldots, m$,
which says that consumption in each state $s$
must equal purchases plus endowment.

\subsubsection{Summarizing Equations}

\paragraph{Budget Constraints}

\begin{align}
    \begin{aligned}
        \sum_{k=1}^K h_k^i q_k & =\sum_{k=1}^K \theta_k^i q_k \\
        x_s^i & =\sum_{k=1}^K h_k^i d_{k s}+\hat{e}_s^i \text { for each } s=1, \ldots, m
        \end{aligned}
\end{align}

\paragraph{Market Clearing}

\begin{align}
    \begin{aligned}
        \sum_{i=1}^I h_k^i & =\sum_{i=1}^I \theta_k^i \text { for each } k=1,2, \ldots K \\
        \sum_{i=1}^I x_s^i & =\sum_{i=1}^I\left[\hat{e}_s^i+\left(\sum_{k=1}^K d_{k s} \theta_k^i\right)\right] \text { for each } s=1,2 \ldots, m
    \end{aligned}
\end{align}

\subsubsection{More stuff}

\begin{definition}[Consistent with State Price] 
    We will say that security prices $q$ and payoffs $D$ are consistent with state price $p$, if
    $$
    q_k=\sum_{s=1}^m p_s d_{k s} \quad \text { for all securities } k=1, \ldots, K .
    $$
\end{definition}

\begin{definition}[Equivalent Endowments]
    
    The endowment $e^i$ and $\left(\hat{e}^i, \theta^i\right)$ are equivalent if
    $$
    \hat{e}_s^i+\sum_{k=1}^K d_{k s} \theta_k^i=e_s^i
    $$
    for all states $s=1,2 \ldots, m$.
     
\end{definition}

\begin{proposition} 
    Assume that prices $q$ and payoffs $D$ are consistent with state prices $p$, as in ( $q$ $=\mathrm{pv}$ using $\mathrm{AD}$ prices). Assume that the endowments are equivalent as in (equivalent endowments). Then:
    
    \begin{enumerate}
        \item If $(x, h)$ is budget feasible in the security market economy, then $x$ is budget feasible in the A-D economy.
        \item If $x$ is budget feasible in the A-D economy, then it must be budget feasible in the security market economy, provided that $D$ has full rank. 
    \end{enumerate}

\end{proposition}

\begin{proposition} 
    Assume that the endowments $e^i$ and $\left(\hat{e}^i, \theta^i\right)$ are equivalent, then
    
    \begin{enumerate}
        \item If $\left(x^i, h^i\right)$ clears the markets in the security market economy, then $\left(x^i\right)$ clears the markets in the A-D economy.
        \item Assume also that $D$ has full rank. 
        If $\left(x^i\right)$ clears the markets in the A-D economy, 
        the $\left(x^i, h^i\right)$ clears the markets in the security market economy.
    \end{enumerate}

\end{proposition}

\subsection{The Tilde Economy}

We will use objects with tildes to denote the A-D economy that 
corresponds to the security market economy. In this economy,
we define the utility as a function of the portfolio shares, 
so that $\tilde{L}=\mathbb{R}^K$. The utility is given by

\begin{align}
    \tilde{u}^i\left(\tilde{x}_1^i, \tilde{x}_2^i, \ldots, \tilde{x}_K^i\right):=u^i\left(\underbrace{\sum_{k=1}^K d_{k 1} \tilde{x}_k^i+e_1^i}_{=x_1^i}, \ldots, \underbrace{\sum_{k=1}^K d_{k s} \tilde{x}_k^i+e_s^i}_{=x_s^i}, \ldots, \underbrace{\sum_{k=1}^K d_{k m} \tilde{x}_k^i+e_m^i}_{=x_m^i}\right)
\end{align}

The budget constraint is:

\begin{align}
    \sum_{k=1}^K \tilde{p}_k \tilde{x}_k^i=\sum_{k=1}^K \tilde{p}_k \tilde{e}_k^i,
\end{align}

where 

\begin{align}
    \tilde{\mathbf{e}}^i=\boldsymbol{\theta}^i
\end{align}

LHS: Tilde utility is the utility derived from the $K$ 
objects that capture the number of shares bought or sold of each of the 
$K$ securities by agent $i$.

RHS: Non-tilde utility is the utility derived from the 
$m$ objects that capture the consumption of each of the 
$m$ distinguishable commodities, which in this case, since 
we're only considering one good,
reflects the $m$ possible states of the world.

Summation elaboration: Each of the $m$ elements in the RHS 
contains a summation over the $K$ securities. For each security, 
we multiply the payoff of that security in state $s$ (corresponding 
to entry $s$ in the input to $u^i$) by the number of shares 
of that security that the agent has bought or sold ($\tilde{x}_k^i$).
This gives the sum of the payoffs of the agent's bought or sold
securities in state $s$. We then add the agent's endowment
of the good in the state $s$. (Notice that there is only one subscript for $e$,
because there is only one good in the simplified model that we're 
considering.)

\subsection{Asset Pricing and the ``Equity Premium''}

Consider an economy with one good, $m$ states, and complete markets.

We are interested in understanding the price of two securities. Security $k=1$ is a riskless bond, i.e.

$$
d_{1 s}=1, \forall s=1,2, \ldots, m
$$

\begin{definition}[Expected Gross Return of Security $k$] 
    The expected (gross) return of any security $k$ is denoted as $r_k$ and defined as

    $$
    1+r_k:=\frac{\sum_{s=1}^m d_{k s} \pi_s}{q_k}=\frac{\mathbb{E}\left[d_k\right]}{q_k} ;
    $$
\end{definition}

\section{Risk Aversion and Portfolio Choice}

\subsection{Terms}

\input{../input/ra_pc_terms.tex}

\subsection{Coefficients of Risk Aversion}

\begin{definition}[Arrow-Pratt Coefficient of Risk Aversion] 
    This coefficient is a measure of the curvature of the utility function around the point $x$, and is given by
    
    $$
    r a(x)=-\frac{u^{\prime \prime}(x)}{u^{\prime}(x)} .
    $$

    The higher the coefficient, the greater is the curvature and, hence, the more risk averse the agent is.
        
\end{definition}

\begin{example}
    CRRA (Constant Relative Risk Aversion)
    $$
    \begin{aligned}
    u(x) & =\frac{x^{1-\gamma}-1}{1-\gamma} \\
    & \Rightarrow-\frac{u^{\prime \prime}(x)}{u^{\prime}(x)}=-\frac{-\gamma x^{-\gamma-1}}{x^{-\gamma}}=\gamma x^{-1}
    \end{aligned}
    $$
\end{example}

\begin{example}
    CARA (Constant Absolute Risk Aversion):
    Is there a utility function $u(x)$ such that $-\frac{u^{\prime \prime}(x)}{u^{\prime}(x)}=$ constant?
    Yes:
    $$
    \begin{aligned}
    u(x) & =-\frac{1}{a} e^{-a x} \\
    & \Rightarrow-\frac{u^{\prime \prime}(x)}{u^{\prime}(x)}=-\frac{-a e^{-a x}}{e^{-a x}}=a
    \end{aligned}
    $$

    See \autoref{fig:ra_pc_cara}.

\begin{figure}[!htb]
    \centering
        \includegraphics[width=0.8\textwidth]{../input/cara_graph.png}
    \caption{CARA Utility Function}
    \label{fig:ra_pc_cara}
\end{figure}

\end{example}

\begin{definition}[Absolute Insurance Premium] 
     

    Absolute insurance premium $p$ is the maximum amount that an agent is willing to pay to avoid a risk $\tilde{x}$ ( a random variable); i.e.\footnote{
        \color{red} Not sure why no tilde on LHS $x$
    }
    
    $$
    u(\mathbb{E}[x]-p)=\mathbb{E}[u(\tilde{x})],
    $$
    
    where $p$ is the premium and $\tilde{x}$ is the risk. The size of $p$ depends on the willingness of the agent to bear risk, as well as the size of the risk.

\end{definition}

For small risk, i.e., 
when the random variable $\tilde{x}$ has a small variance, 
$\sigma^2$, we can express the premium as:

\begin{align}
    p=-\frac{1}{2} \frac{u^{\prime \prime}(\bar{x})}{u^{\prime}(\bar{x})} \sigma^2
\end{align}

where 

\begin{align}
    \bar{x}=E[\tilde{x}]
\end{align}

\begin{definition}[Coefficient of Relative Risk Aversion]
    
        $$
        r r a(x)=-\frac{u^{\prime \prime}(x)}{u^{\prime}(x)} x .
        $$
     
\end{definition}

\begin{definition}[Relative Insurance Premium]
    Relative insurance premium is the the maximum proportion of certain consumption, $\bar{x}$, that an agent is willing to pay to avoid a risk $\tilde{x}=\bar{x}(1+\varepsilon)$; i.e.
    $$
    u((1-\rho) \bar{x})=\mathbb{E}[u(\bar{x}(1+\varepsilon))]
    $$
    where $\mathbb{E}[\varepsilon]=0$ and $\mathbb{E}\left[\varepsilon^2\right]=\sigma_{\varepsilon}^2$

\end{definition}

\begin{proposition}
    Relative insurance premium. Suppose that the risk is small. Then, the relative insurance premium $\rho$ is given by
    $$
    \rho=\frac{1}{2}\left(-\frac{u^{\prime \prime}(\bar{x}) \bar{x}}{u^{\prime}(\bar{x})}\right) \sigma_{\varepsilon}^2,
    $$
    where $-u^{\prime \prime}(\bar{x}) \bar{x} / u^{\prime}(\bar{x})$ measures the agent's willingness to bear relative risk, $\sigma_{\varepsilon}^2$ measures the size of the risk, and the utility function is evaluated at the expected value of the risk; i.e. $\bar{x}=\mathbb{E}[\tilde{x}]$.

\end{proposition}

We can then notice a relationship between the 
absolute and proportional risk premium:

\begin{align}
    p & =\bar{x} \rho \\
    \sigma^2(x) & =(\bar{x})^2 \sigma_{\varepsilon}^2
\end{align}

Thus using the expression for $p$ :
$$
p=-\frac{u^{\prime \prime}(\bar{x})}{u^{\prime}(\bar{x})} \sigma^2(x)=-\frac{u^{\prime \prime}(\bar{x})}{u^{\prime}(\bar{x})}(\bar{x})^2 \sigma_{\varepsilon}^2
$$
or
$$
\frac{p}{x}=\rho=-\bar{x} \frac{u^{\prime \prime}(\bar{x})}{u^{\prime}(\bar{x})} \sigma_{\varepsilon}^2
$$

\subsection{Certainty Equivalent}

\begin{definition}[Certainty Equivalent]
    
    A certainty equivalent of risk $\tilde{x}$, denoted $c_e(\tilde{x})$, is given by
    $$
    u\left(c_e\right)=\mathbb{E}[u(\tilde{x})] .
    $$

    Hence, $c_e$ is the sure (deterministic) amount of consumption that will be equivalent to a given risk $\tilde{x}$. 
    To draw parallel with earlier definitions:
    $$
    c_e=\bar{x}-p=\bar{x}(1-\rho) .
    $$
     
\end{definition}

\subsection{Arrow-Pratt Theorem}

\begin{theorem}
    (Arrow-Pratt) Let $u$ and $v$ be utility functions. The following statements are equivalent.
    
    \begin{enumerate}
        \item  If $u$ is an increasing and concave transformation of $v$; i.e. there exists a function $f$ such that
        $$
        u(x)=f(v(x)), \forall x,
        $$
        and,
        $$
        \begin{aligned}
        f^{\prime}(\cdot) & >0, \\
        f^{\prime \prime}(\cdot) & <0 .
        \end{aligned}
        $$
        \item $u$ has a higher insurance premium than v; i.e. for all random variables $\tilde{x}$, the insurance premium $p_u(\tilde{x}), p_v(\tilde{x})$ corresponding to the utility functions $u$ and $v$ are:
        $$
        p_u(\tilde{x})>p_v(\tilde{x}) .
        $$
        \item The absolute risk aversion coefficient of $u$ is higher than that of $v$ everywhere; i.e.
        $$
        -\frac{u^{\prime \prime}(x)}{u^{\prime}(x)}>-\frac{v^{\prime \prime}(x)}{v^{\prime}(x)}, \forall x
        $$
    \end{enumerate}

\end{theorem}

Intuition: 
The more concave the increasing utility function, the more risk averse the agent is, and the higher the insurance premium that the agent is willing to pay to avoid risk.
(Is that correct?)

\subsection{Portfolio Choice Problem}

\subsubsection{Maximization Problem}

$$
\max _{\left\{w_i\right\}} \mathbb{E}[u(\tilde{W})]=\max _{\left\{w_i\right\}}\sum_{s=1}^S u\left(W\left[\sum_{i=1}^N w_i\left(R_{i, s}-\bar{\mu}\right)+\bar{\mu}\right]\right) \pi_s
$$

First order conditions :
$$
\begin{aligned}
& \mathbb{E}\left[u^{\prime}(\tilde{W})\left(\tilde{R}_i-\bar{\mu}\right)\right]  \\
= & \mathbb{E}\left[u^{\prime}\left(W\left[\sum_{j=1}^N w_j\left(\tilde{R}_j-\bar{\mu}\right)+\bar{\mu}\right]\right)\left(\tilde{R}_i-\bar{\mu}\right)\right] \\
= & 0
\end{aligned}
$$
for $i=1,2, \ldots, N$.

\begin{lemma}
    (Properties of concavity).

    \begin{enumerate}
        \item Additive of concavity: let $f(x)$ and $g(x)$ be (strictly) concave, then $h(x)=f(x)+g(x)$ is also (strictly) concave.
        \item Preservation of concavity under strictly increasing and
            strictly concave transformation: 
            let $f(x)$ and $g(x)$ be strictly concave and $f(x)$ be a strictly increasing function, then $h(x)=$ $f(g(x))$ is also strictly concave.
    \end{enumerate}

\end{lemma}

\subsubsection{One Risky Asset Case}

In this case, we can denote the share of wealth 
invested in the risky asset as $w$
and the share invested in the risk-free asset as $1-w$.
We can denote the return on the risky asset as $\tilde{R}$.

Then the simplified problem is:

\begin{align}
    \max _w \mathbb{E}[u(W[w(\tilde{R}-\bar{\mu})+\bar{\mu}])]
\end{align}

and the first order condition becomes:

\begin{align}
    W \mathbb{E}\left[u^{\prime}\left(W\left[w_u^*(\tilde{R}-\bar{\mu})+\bar{\mu}\right]\right)(\tilde{R}-\bar{\mu})\right]=0
\end{align}

Intuition for the below propositions: 
(i) the investor will always invest in a risky asset with a higher return than the risk-free return, independent of the degree of his risk aversion; and (ii) the more risk averse the agent is, the smaller the size of the investment in the risky asset.

\begin{proposition} 
    The investor will always invest in a risky asset with a higher return than the risk-free return, no matter how risk averse the investor is. That is,
    $$
    w_u^*>0 \Leftrightarrow \mathbb{E}[\tilde{R}]>\bar{\mu} .
    $$ 
\end{proposition}

\begin{proposition} 
    Suppose $\mathbb{E}[\tilde{R}]>\bar{\mu}, u$ is strictly concave, and that $N=1$ (i.e. there is only one risky asset). Suppose $u$ is more risk averse than $v$; i.e.
    $$
    -\frac{u^{\prime \prime}(x)}{u^{\prime}(x)}>-\frac{v^{\prime \prime}(x)}{v^{\prime}(x)}, \forall x>0 .
    $$
    
    Then, the proportion of invested into the risky asset, $w_u$ and $w_v$ corresponding to $u$ and $v$ respectively, are such that
    $$
    w_u<w_v .
    $$ 
\end{proposition}

\section{Steady States, Neoclassical Growth Model, Deterministic Case}

\subsection{Terms}

\input{../input/neo_clas_terms.tex}

\subsection{Setup}

\subsubsection{Household}

We take there to be a representative household who
derives utility from a sequence of unique consumption
good and leisure. 

\begin{align}
    \left\{c_t, 1-n_t\right\}_{t=0}^{\infty}
\end{align}

with utility:

\begin{align}
    u\left(\left\{c_t, 1-n_t\right\}_{t=0}^{\infty}\right)=u\left(c_0, 1-n_0, c_1, 1-n_1, \ldots\right)
\end{align}

where the time endowment is normalized to one, 
so that leisure is given by $1-n_t$.

\subsubsection{Planning Problem}

We start with the following planning problem:

\begin{align}
        \max _{\left\{c_t, n_t, x_t, k_{t+1}\right\}_{t=0}^{\infty}} & u\left(c_0, 1-n_0, c_1, 1-n_1, \ldots\right) \\
        \text { s.t. } & c_t+x_t=F\left(k_t, n_t\right), \forall t \geq 0 \\
        & k_{t+1}=x_t+k_t(1-\delta), \forall t \geq 0 \\
        & k_0 \text { given, }
\end{align}

The first constraint is the resource constraint; it says 
that consumption and investment must equal the output of the
production function. 

The second constraint is the law of motion for capital; it says that
capital in the next period is equal to the sum of investment and the
depreciated value of capital from the current period.

We often use:

\begin{align}
    u\left(c_0, 1-n_0, c_1, 1-n_1, \ldots\right)=\sum_{t=0}^{\infty} \beta^t v\left(c_t, 1-n_t\right)
\end{align}

as our utility function.

\subsection{Version 1: Firms Own and Accumulate Capital}

\subsubsection{V1 Setup}

\begin{itemize}
    \item Commodity space: Let $\mathbb{R}^{2 T}$ for $T=\infty$. More formally,

        \begin{align}
            L:=\left\{\left\{c_t, n_t\right\}_{t=0}^{\infty}:\left(c_t, n_t\right) \in \mathbb{R}^2, \forall t \geq 0\right\}
        \end{align}
        
        i.e. the commodity space is the set of all pairs of real sequences.
    \item Production possibility set of the firm: 
        \begin{align}
            Y:=\left\{\left\{c_t, n_t\right\}_{t=1}^{\infty}: x_t+c_t \leq F\left(k_t, n_t\right), k_{t+1}=x_t+k_t(1-\delta), x_t \in \mathbb{R}, \forall t \geq 0, k_0 \text { given }\right\}
        \end{align}
        Constraints reflect the resource constraint and the law of motion for capital:
        \begin{itemize}
            \item $x_t+c_t \leq F\left(k_t, n_t\right)$: resource constraint
            \item $k_{t+1}=x_t+k_t(1-\delta)$: law of motion for capital
        \end{itemize}
    \item Consumption possibility set:
        \begin{align}
            X:=\left\{\left\{c_t, n_t\right\}_{t=0}^{\infty}: c_t \geq 0,0 \leq n_t \leq 1\right\}
        \end{align}
    \item Household budget constraint:
        \begin{align}
            \sum_{t=0}^{\infty} p_t\left(c_t+w_t \ell_t\right)=\pi+\sum_{t=0}^{\infty} p_t w_t
        \end{align}
        \begin{itemize}
            \item LHS:Purchases of consumption goods and leisure (over all $t$)
                \begin{itemize}
                    \item $p_t w_t \ell_t$: Weird term but $p_t w_t$ is 
                        the wage income in period $t$, and $\ell_t$ 
                        is an amount of leisure consumed in period $t$,
                        so the product is like the surrendered wage income.
                \end{itemize}
            \item RHS: Firm profit (over all $t$) plus wage income (over all $t$)  
        \end{itemize}
\end{itemize}

\subsubsection{Firm's Problem}

The firm's problem is:

\begin{align}
    \pi=\max _{\left\{\left\{c_t, n_t\right\} \in Y\right\}_{t=0}^{\infty}} \sum_{t=0}^{\infty} p_t\left(c_t-w_t n_t\right)
\end{align}

which we can re-write as:

\begin{align}
    \pi=\max _{\left\{k_{t+1}, n_t\right\}_{t=0}^{\infty}} \sum_{t=0}^{\infty} p_t\left(F\left(k_t, n_t\right)-w_t n_t-\left(k_{t+1}-k_t(1-\delta)\right)\right)
\end{align}

using the fact that the resource contraint will bind, i.e., 

\begin{align}
    x_t+c_t=F\left(k_t, n_t\right)
\end{align}

and the law of motion is:

\begin{align}
    k_{t+1}=x_t+k_t(1-\delta)
\end{align}

so consumption can be written as:

\begin{align}
    c_t=F\left(k_t, n_t\right)-k_{t+1}+k_t(1-\delta)
\end{align}

FOCs then give:

\begin{align}
    w_t & =F_n\left(k_t, n_t\right), \\
    p_t & =p_{t+1}\left[F_k\left(k_{t+1}, n_{t+1}\right)+(1-\delta)\right] \\
    \Rightarrow-(1-\delta)+\underbrace{\frac{p_t}{p_{t+1}}}_{:=1+r_t} & =F_k\left(k_{t+1}, n_{t+1}\right) \\
    \Rightarrow \delta+r_t & =F_k\left(k_{t+1}, n_{t+1}\right)
\end{align}

\paragraph{Simplifying $F$}

Since $F$ is CRS, by Euler's theorem, we have:

\begin{align}
    F\left(k_t, n_t\right)=F_n\left(k_t, n_t\right) n_t+F_k\left(k_t, n_t\right) k_t
\end{align}

and using the FOC wrt $n_t$ (i.e., $F_n\left(k_t, n_t\right)=w_t$):

\begin{align}
    F_k\left(k_t, n_t\right) k_t=F\left(k_t, n_t\right)-w_t n_t
\end{align}

Then note that since $F$ is CRS, it's partials 
are homogenous of degree zero. 
Define

\begin{align}
    \kappa_t:=k_t / n_t
\end{align}

which gives us:

\begin{align}
    F_k\left(\kappa_t, 1\right) k_t=F\left(k_t, n_t\right)-w_t n_t
\end{align}

Thus, we can re-write maximized profit as:

\begin{align}
    \pi=\sum_{t=0}^{\infty} p_t\left(F_k\left(\kappa_t, 1\right) k_t-\left(k_{t+1}-k_t(1-\delta)\right)\right)
\end{align}

where $\kappa_t$ solves $F_n\left(\kappa_t, 1\right)=w_t$.

\subsection{Version 2: Households Own Capital, which is Rented to the Firm}

\subsubsection{V2 Setup}

\begin{itemize}
    \item Commodity Space:
        Let $L=R^{4 T}$ for $T=\infty$, or more formally
        \begin{align}
            L=\left\{\left\{c_t, n_t, x_t, k_t\right\}_{t=0}^{\infty} \mid\left(c_t, n_t, k_t\right) \in R^4\right\}
        \end{align}
    \item Production possibility set of the firm:
        \begin{align}
            Y=\left\{\left\{c_t, x_t, n_t, k_t\right\}: c_t+x_t \leq F\left(n_t, k_t\right)\right\}
        \end{align}
    \item Consumption possibility set:
        \begin{align}
            X= & \left\{\left\{c_t, n_t, k_t\right\}: 0 \leq n_t \leq 1, k_{t+1}=x_t+k_t(1-\delta) \text { for all } t \geq 0, k_0 \text { given }\right\}
        \end{align}
        \begin{itemize}
            \item Notice that the law of motion for capital is now part of the consumption possibility set, 
                rather than the production possibility set as it was in Version 1.
        \end{itemize}
\end{itemize}

\subsubsection{Firm's Problem}

The firm's problem now becomes:

\begin{align}
    \pi=\max _{(y, n, k) \in Y} \sum_{t=0} p_t\left[y_t-w_t n_t-v_t k_t\right]
\end{align}


Notice the constraint in the production possibility set will be binding, 
so we can write:

\begin{align}
    \pi=\max _{\left\{\left\{n_t, k_t\right\} \in Y_t\right\}_{t=0}^{\infty} \in Y} \sum_{t=0}^{\infty} p_t\left(F\left(k_t, n_t\right)-w_t n_t-v_t k_t\right)
\end{align}






\end{document}